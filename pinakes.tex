\documentclass[a4paper,table]{report}
\input{preamble_ask.tex}
\input{definitions_ask.tex}


\geometry{top=2cm}
% \DeclareMathOperator{\r}{r}

\pagestyle{askhseis}

\begin{document}

\chapter*{Πίνακες}

\section*{Άλγεβρα Πινάκων}

\begin{enumerate}

\item Έστω $ A = 
  \begin{pmatrix*}[r]
    1 & 2 \\
    4 & -3
  \end{pmatrix*} $ και έστω $ f(x) = 2x^{3}-4x+5 $ και $ g(x) = x^{2}+2x+11 $. Να 
  υπολογίσετε τους πίνακες 
  \begin{enumerate}[i)]
    \item $ A^{2} $
    \item $ A^{3} $
    \item $ f(A) $
    \item $ g(A) $
  \end{enumerate}

  \hfill Απ: $ A^{2} = 
  \begin{pmatrix*}[r]
    9 & -4 \\
    -8 & 17
  \end{pmatrix*}, \; A^{3} = 
  \begin{pmatrix*}[r]
    -7 & 30 \\
    60 & -67
  \end{pmatrix*}, \; f(A) = 
  \begin{pmatrix*}[r]
    -13 & 52 \\
    104 & -117
  \end{pmatrix*}, g(A) = 
  \begin{pmatrix*}[r]
    0 & 0 \\
    0 & 0 
  \end{pmatrix*} $ 

  \item Δίνονται οι πίνακες $ A = 
    \begin{pmatrix}
      1 & 2 & 1 \\
      1 & 1 & 0 \\
      2 & 0 & -1
    \end{pmatrix} $, $ B= 
    \begin{pmatrix}
      2 & 1 & 3 \\
      4 & 3 & 0 \\
      3 & 2 & 1
    \end{pmatrix} $ και $ \mathbf{x} =  
    \begin{pmatrix*}[r] x \\ y \\ z \end{pmatrix*} $. Να υπολογιστούν τα γινόμενα:
    \begin{enumerate}[i)]
      \item $ A\cdot B $
      \item $ B\cdot A $
      \item $A \cdot \mathbf{x} $
      \item $ \mathbf{x}^{T} \cdot A$
    \end{enumerate} 
    \hfill Απ: $ \scriptstyle{AB= 
    \begin{pmatrix}
      13 & 9 & 4 \\
      6 & 4 & 3 \\
      1 & 0 & 5
    \end{pmatrix}, \;  BA = 
    \begin{pmatrix*}[r]
      9 & 5 & -1 \\
      7 & 11 & 4 \\
      7 & 8 & 2
    \end{pmatrix*}, \; A \mathbf{x} = 
    \begin{pmatrix*} x+2y+z \\ x+y \\ 2x-z \end{pmatrix*}, \;  \mathbf{x}^{T}A = 
    \begin{pmatrix*}
      x+y+2z & 2x+y & x-z
  \end{pmatrix*}} $  

\item Να βρεθούν τα $ x,y, z $ και $ w $ αν $ 3 
  \begin{pmatrix*}[r]
    x & y \\
    z &w
  \end{pmatrix*} = 
  \begin{pmatrix*}[r]
    x & 6 \\
    -1 & 2w
  \end{pmatrix*} + 
  \begin{pmatrix*}
    4 & x+y \\
    z+w & 3
  \end{pmatrix*} $ 

  \hfill Απ: $ x=2, \; y=2, \; z=1, \; w=3 $ 

\item Να βρείτε όλους τους, πραγματικούς, διαγώνιους $ 2 \times 2 $ πίνακες για τους 
  οποίους ισχύει ότι $ A^{2}=3A $.

  \hfill Απ: $\scriptstyle{ 
  \begin{pmatrix*}[r]
    0 & 0 \\
    0 &0
  \end{pmatrix*}, \; 
  \begin{pmatrix*}[r]
    0 & 0 \\
    0 & 3
  \end{pmatrix*}, \; 
  \begin{pmatrix*}[r]
    3 & 0 \\
    0 & 0
  \end{pmatrix*}, \; 
  \begin{pmatrix*}[r]
    3 & 0 \\
    0 & 3
\end{pmatrix*}} $
  

\item Αν $ A = 
  \begin{pmatrix*}[r]
    -2 & 5 \\
    -1 & 2
  \end{pmatrix*} $, να υπολογιστεί ο πινακας $ A^{2021} - 3A^{2021} +2I $. 

  \hfill Απ: $ A = 
  \begin{pmatrix*}[r]
    6 & -10 \\
    2 & -2
  \end{pmatrix*} $ 
\end{enumerate}

\section*{Συμμετρικοί - Αντισυμμετρικοί - Ορθογώνιοι Πίνακες}

\begin{enumerate}
  \item Να υπολογιστεί η τιμή του $x$ ώστε ο πίνακας $ B = 
    \begin{pmatrix*}[r]
      4 & x+2 \\
      2x-3 & x+1 
    \end{pmatrix*} $, να είναι συμμετρικός. 

    \hfill Απ: $x=5$  
  \item Να δείξετε ότι για κάθε πίνακα $ A \in \textbf{M}_{m \times n}(\mathbb{R}) $, 
    οι πίνακες $ AA^{T} $ και $ A^{T}A $ είναι συμμετρικοί.

  \item Έστω $ A_{3\times 3} $ ορθογώνιος πίνακας. Να δείξετε ότι:
    \begin{enumerate}[i)]
      \item $ A^{T}(A-I) = -(A-I)^{T} $
      \item $ |A|=\pm 1 $
      \item $ |A|=1 \Rightarrow |A-I|=0 $
    \end{enumerate}

    %spand. ex.28 p.168
  \item Έστω $ A,B \in \textbf{M}_{n}(\mathbb{R}) $ ορθογώνιοι πίνακες. Να δείξετε ότι: 
    \begin{enumerate}[i)]
      \item $ AB, A^{T}B, AB^{T} $ και $ A^{T}B^{T} $ είναι επίσης ορθογώνιοι.
      \item $ A^{-1} $ είναι επίσης ορθογώνιος.
    \end{enumerate}

    %spand. ex.27 p.167
  \item Αν $A_{n\times n}$ συμμετρικός, $B_{n\times n}$ αντισυμμετρικός και 
    $C =(A+B)^{-1}(A-B) $, τότε να δείξετε ότι 
    \begin{enumerate}[i)]
      \item $C^{T}(A+B)C=A+B $
      \item $ C^{T}(A-B)C=A-B $
    \end{enumerate}

  \item Αν $A_{n\times n}$ ορθογώνιος και $ A+I $ αντιστρέψιμος, τότε να δείξετε ότι 
    $ (A+I)^{-1} + [(A+I)^{-1}]^{T} = I $ .
\end{enumerate}


\section*{Αντίστροφος Πίνακας}



\begin{enumerate}
  
  \item Έστω $ A = 
    \begin{pmatrix*}[r]
      4 & 1 \\
      -3 & -1
    \end{pmatrix*} $ και $ B = 
    \begin{pmatrix*}[r]
      1 & 1 \\
      -3 & -4
    \end{pmatrix*}$. 
    \begin{enumerate}[i)]
      \item Να αποδείξετε ότι ο $A$ αντιστρέφεται και ότι $ A^{-1} = B $.
      \item Να αποδείξετε ότι ο $ B $ αντιστρέφεται. Ποιος είναι ο $ B^{-1} $;
      \item Να αποδείξετε ότι ο $ A^{2} $ αντιστρέφεται. Ποιος είναι ο $ (A^{2})^{-1} $;
      \item Να λύσετε την εξίσωση $ AX= 
        \begin{pmatrix*}[r]
          1 & -1 & 0 \\
          0 & 1 & 2
        \end{pmatrix*} $.
    \end{enumerate}

    \hfill Απ: $ X= 
    \begin{pmatrix*}[r]
      1 & 0 & 2 \\
      0 & 1 & 2
    \end{pmatrix*} $ 

  \item Αν $B$ είναι ο αντίστροφος του $ A^{2} $, να δείξετε ότι $ A^{-1}=AB $.

  \item Αν $ A,B,A-B $ και $ B^{-1}+A^{-1} $ αντιστρέψιμοι 
    $ n \times n $ πίνακες, τότε να δείξετε ότι:
    \begin{enumerate}[i)]
      \item $ (A-B)^{-1} = A^{-1}+A^{-1}(B^{-1}-A^{-1})^{-1}A^{-1} $
      \item $ (I+A)^{-1} = I-(A^{-1}+I)^{-1} $
      \item $\tr(I+A)^{-1} + \tr(A^{-1}+I)^{-1} = n $
    \end{enumerate}
    
    %tzouv ex4.41 p.130
    %spand. ex.26 p.167
  \item Έστω $ A, B \in \textbf{M}_{n}(\mathbb{R}) $ αντιστρέψιμοι. Να δείξετε ότι:
    \begin{enumerate}[i)]
      \item $ (B^{-1}AB)^{n} = B^{-1}A^{n}B, \; \forall n \in \mathbb{N} $
      \item $ (B^{-1}AB)^{-1} = B^{-1}A^{-1}B $
    \end{enumerate}

    %tzouv ex4.43 p.131
  \item Έστω $ A \in \textbf{M}_{n}(\mathbb{R}) $ τέτοιος ώστε $ A^{2}+A=3I $. 
    \begin{enumerate}[i)]
      \item Να δείξετε ότι ο πίνακας $ A+2I $ αντιστρέφεται και να βρείτε τον αντίστροφό
        του.
      \item Να λύσετε την εξίσωση $ AX+2A=X+3I $, όπου 
        $ X \in \textbf{M}_{n}(\mathbb{R}) $.
    \end{enumerate}

    %tzouv ex4.49 p.135
  \item Έστω $ A \in \textbf{M}_{n}(\mathbb{R}) $ τέτοιος ώστε $ A^{4}+3A^{2}+I=O $. 
    \begin{enumerate}[i)]
      \item Να δείξετε ότι ο πίνακας $ A^{2}+2I $ αντιστρέφεται και να βρείτε τον 
        αντίστροφό του.
    \end{enumerate}


\end{enumerate}








\end{document}

