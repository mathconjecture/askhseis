\documentclass[a4paper,table]{report}
\input{preamble.tex}
\input{definitions_ask.tex}


\renewcommand{\vec}{\mathbf}
\pagestyle{askhseis}
\begin{document}

\begin{center}
  \minibox{\large \bfseries \textcolor{Col1}{Ασκήσεις στα Διανύσματα}}
\end{center}

\vspace{\baselineskip}

\begin{enumerate}
  \item Να υπολογιστεί η γωνία $ \theta $, $ 0\leq \theta \leq \pi $ μεταξύ των διανυσμάτων $
    \vec{a} = (3,0,3) $ και $ \vec{b} = (0,6,6) $.

    \hfill Απ: $ \frac{\pi}{3} $

  \item Να δειχθεί με τη βοήθεια του εσωτερικού γινομένου ότι
    \begin{enumerate}[i)]
      \item τα διανύσματα $ \vec{a}_1 = (5,-2,1) $ και $ \vec{b}_1 = (-15,6,-3) $ είναι παράλληλα.
      \item τα διανύσματα $ \vec{a}_2 = (4,-3,2) $ και $ \vec{b}_2 = (-1,-2,1) $ είναι κάθετα.
    \end{enumerate}

  \item Να υπολογιστεί το εξωτερικό γινόμενο των διανυσμάτων $ \vec{a} = (2,-1,1) $ και $ \vec{b} = (-1,2,-3) $.

    \hfill Απ: $ (1,5,3) $

  \item Να υπολογιστεί το εμβαδό του παραλληλογράμμου που σχηματίζεται από τα διανύσματα $ \vec{a} = (3,-2,-2) $ και $ \vec{b} = (-1,0,-1) $.

    \hfill Απ: $3 $ 
\end{enumerate}

\end{document}
