\input{$HOME/Desktop/preamble/preamble.tex}
\input{$HOME/Desktop/preamble/definitions.tex}

\everymath{\displaystyle}
\thispagestyle{empty}

\begin{document}

\begin{center}
	\fbox{\Large\bfseries Μήκος Καμπύλης}
\end{center}

\vspace{\baselineskip}

\begin{enumerate}

	\item Να υπολογιστεί το μήκος του τόξου της καμπύλης με εξίσωση $ y =
		\ln{\frac{e^{x} + 1}{e^{x} - 1}}$, από $ x_{1} = a $ εως $ x_{2} = b $.

		\hfill Απ: $ \ln{\frac{e^{b} - e^{-b}}{e^{a} - e^{-a}}} $

	\item Να υπολογιστεί το μήκος της καμπύλης $y=\ln(\cos x)$ όταν $\frac{\pi}{6}\leq x
		\leq \frac{\pi}{3}$.

		\hfill Απ: $\ln\frac{2+\sqrt{3}}{\sqrt{3}}$

	\item Να υπολογιστεί το μήκος της καμπύλης $y=\frac{e^{x}+e^{-x}}{2}$, από $x=0$ έως $x=1$.

		\hfill Απ: $\frac{e^{2}-1}{2e}$

	\item Να υπολογιστεί το μήκος της καμπύλης που βρίσκεται στο 1ο και 4ο
		τεταρτημόριο και ορίζεται από τις καμπύλες $ y^{2} = 2x^{3} $ και $ x^{2} +
		y^{2} = 20 $.

		\hfill Απ: $ \frac{8}{27} \left(10 \sqrt{10} -1\right) + 2 \sqrt{20}
		\left(\frac{\pi}{2} - \arcsin\Bigl(\frac{\sqrt{5}}{5}\Bigr)\right) $ 

	\item Να υπολογιστεί το μήκος της αστροειδούς καμπύλης $ x^{\frac{2}{3}}
		+ y^{\frac{2}{3}} = a^{\frac{2}{3}} $.

		\hfill Απ: 6a

	\item Να υπολογιστεί το μήκος του τόξου της καμπύλης $ \frac{1}{4} y^{2} = x +
		\frac{1}{2} \ln{y} $ που περιορίζεται από τις ευθείες $ y = 1 $ και $ y =
		2$.

		\hfill Απ: $ \frac{3}{4} + \frac{\ln{2}}{2} $

	\item Να υπολογιστεί το μήκος της καμπύλης $ x = e^{t} \sin{t} $, $ y = e^{t}
		\cos{t} $, για $ t \in [0, \pi] $.

		\hfill Απ: $ \sqrt{2} (e^{\pi} - 1)  $

		\hfill Απ: $\frac{8}{27}(10\sqrt{10}-1)$

	\item Να υπολογιστεί το μήκος της καμπύλης $x=1+\sin t$, $y=2+\cos t$, όταν $0\leq t\leq \pi$.

		\hfill Απ: $4$

	\item Να υπολογιστεί το μήκος της καμπύλης $x=2-\ln(1+t^{2})$, $y=1+\arccos t$ για $0\leq t\leq 1$.
		
  \hfill Απ: $\ln(\sqrt{2}+1)$

	\item Να υπολογιστεί το μήκος της καρδιοειδούς καμπύλης $ r = a + a \cos{\theta}
		$, $ a > 0 $.

		\hfill Απ: $ 8 a $

	\item Να υπολογιστεί το μήκος της καμπύλης $ r = \cos^{3}{\left(\frac{\theta}{3}\right)} $, όταν $ 0\leq
		\theta \leq \pi $.
\end{enumerate}



\end{document}
