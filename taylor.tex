\documentclass[a4paper,table]{report}
\input{preamble_ask.tex}
\input{definitions_ask.tex}

\everymath{\displaystyle}
\pagestyle{askhseis}

\begin{document}

\begin{center}
  \minibox{\bfseries\large \textcolor{Col1}{Ασκήσεις στις Παραγώγους}}
\end{center}

\vspace{\baselineskip}


\begin{enumerate}
  \item {\bfseries Να υπολογιστεί το προσεγγιστικό πολυώνυμο Maclaurin 3ης τάξης, των 
    παρακάτω συναρτήσεων.}
    \begin{enumerate}[i)]
      \item $ y= \mathrm{e}^{-x} $ 
        \hfill Απ: $ \mathrm{e}^{-x} \approx 1-x+ \frac{1}{2} x^{2} - \frac{1}{6} x^{3} $ 
      \item $ y= \ln{(x+2)} $ 
        \hfill Απ: $ \ln{(x+2)} \approx \ln{2} + \frac{1}{2} x - \frac{1}{8} x^{2} +
        \frac{1}{24} x^{3} $ 
      \item $ y= \frac{1}{x-1} $ \hfill Απ: $ \frac{1}{x-1} \approx -1 -x -x^{2} - x^{3} $ 
      \item $ y= \sqrt{x+1} $ \hfill Απ: $ \sqrt{x+1} \approx 1 + \frac{1}{2} x -
        \frac{1}{8} x^{2} + \frac{1}{16} x^{3} $ 
    \end{enumerate}

  \item {\bfseries Να υπολογιστεί το προσεγγιστικό πολυώνυμο Taylor 3ης τάξης, γύρω από 
      το σημείο $ x_{0}=1 $, των παρακάτω συναρτήσεων.}
    \begin{enumerate}[i)]
      \item $ y= \ln{(x+1)} $ 
        \hfill Απ: $ \ln{(x+1)} \approx \ln{2} + \frac{1}{2} (x-1) - \frac{1}{8}
        (x-1)^{2} + \frac{1}{24} (x-1)^{3} $ 
      \item $ y= \frac{1}{x+1} $ \hfill Απ: $ \frac{1}{x+1} \approx \frac{1}{2} -
        \frac{1}{4} (x-1) + \frac{1}{3} (x-1)^{2} - \frac{1}{16} (x-1)^{3} $ 
    \end{enumerate}
\end{enumerate}




\end{document}






