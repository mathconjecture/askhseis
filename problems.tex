\input{preamble_ask.tex}
\input{definitions_ask.tex}
\input{tikz.tex}

\pagestyle{askhseis}

\begin{document}

\begin{center}
  \minibox{\large \bfseries \textcolor{Col1}{Προβλήματα Ακροτάτων και Ρυθμού Μεταβολής}}
\end{center}

\vspace{\baselineskip}

\begin{enumerate}

	\item {\bfseries (Ιαν 2016)} Να προσδιοριστούν οι διαστάσεις μιας πισίνας $ \SI{32}{m^{3}} $ με
		τετραγωνική βάση έτσι ώστε η επιφάνεια των εσωτερικών τοίχων και του
		πυθμένα να είναι ελάχιστη. 

		\hfill Απ: $x=4, y=2$

	\item {\bfseries (Ιαν 2016)} Να βρεθεί η εξίσωση της ευθείας $ (\varepsilon)
		$ που περνάει από γνωστό σημείο $ P(a,b) $ και σχηματίζει με τους άξονες
		συντεταγμένων τρίγωνο ελάχιστου εμβαδού $ (a>0,\, b>0) $.

	\item Θεωρούμε τρίγωνο ΑΒΓ με  ΒΓ$=a $ και ύψος ΑΔ$=h$. Επίσης θεωρούμε τα
		εγγεγραμένα σε αυτό ορθογώνια των οποίων η μία πλευρά βρίσκεται πάνω στη
		ΒΓ. Να βρεθεί εκείνο το παραλληλόγραμμο που έχει μέγιστο εμβαδό.
		
		\hfill Απ: $ x = \frac{h}{2}, y= \frac{a}{2} $

	\item Να εγγραφεί ορθογώνιο παραλληλόγραμμο με μέγιστο εμβαδό στην έλλειψη $
		\frac{x^{2}}{a^{2}} + \frac{y^{2}}{b^{2}} = 1 $. 

		\hfill Απ: $ x = \frac{a\sqrt{2}}{2}, y = \frac{b \sqrt{2}}{2} $

	\item Ένα φύλλο ειδικού χαρτιού για μεγάλο μηχανολογικό σχέδιο, έχει
		επιφάνεια \SI{2}{m^{2}}. Στο σχέδιο που θα γίνει πρέπει να αφεθούν
		περιθώρια στην πάνω και στην κάτω πλευρά του σχεδίου, \SI{21}{cm}, ενώ
		στις πλαινές πλευρές \SI{14}{cm}. Ποιες πρέπει να είναι οι διαστάσεις
		τουτου φύλλου ώστε το καθαρό εμβαδό για σχεδίαση να είναι μέγιστο?

		\hfill $ x = \sqrt{3}, y = \frac{2 \sqrt{3}}{3} $

	\item Δυο πόλεις Α και Β βρίσκονται προς το ίδιο μέρος της όχθης ενός
		ποταμού και απέχουν απ᾽ αυτήν 10 και 15 χιλιόμετρα αντίστοιχα. Οι
		κάθετες προβολές των δύο πόλεων στην όχθη απέχουν 20 χιλιόμετρα. Οι δύο
		πόλεις πρέπει να εφοδιαστούν με νερό από ένα εργοστάσιο που θα
		κατασκευαστεί στην όχθη του ποταμού. Σε ποιο σημείο της όχθης πρέπει να
		κατασκευαστεί το εργοστάσιο ώστε για τους αγωγούς που θα συνδέσουν αυτό
		με τις πόλεις να έχουμε το ελάχιστο κόστος?

		\hfill Απ: $ x = \SI{8}{km} $

	\item Πλοίο Α ξεκινάει από ένα λιμάνι στις 12 μ.μ. και κατευθύνεται δυτικά
		με ταχύτητα  \SI{9}{km/h}. Πλοίο Β ξεκινάει από το ίδιο λιμάνι στη 1
		μ.μ. και κατευθύνεται νότια με ταχύτητα \SI{12}{km/h}. Με τι ρυθμό
		απομακρύνονται μεταξύ τους τα δύο πλοία στις 3 μ.μ.?

		\hfill $ \dv{s}{t} = \SI{14,7}{km/h} $

\end{enumerate}

\end{document}
