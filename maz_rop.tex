\input{$HOME/Desktop/preamble/preamble.tex}
\input{$HOME/Desktop/preamble/definitions.tex}


\thispagestyle{empty}

\begin{document}


\begin{center}
  \fbox{\bfseries\Large Ασκήσεις στα Υλικά Χωρία}
\end{center}

\vspace{\baselineskip}

\begin{enumerate}
  \item Να βρεθεί το κέντρο μάζας του υλικού χωρίου $D$ που ορίζεται από τις καμπύλες $y^{2}=x$ και $y=x^{3}$, όταν η συνάρτηση πυκνότητας μάζας, είναι $\delta(x,y)=y$

  \hfill Απ: $\overline{x}=\frac{7}{12}$, $\overline{y}=\frac{14}{25}$

  \item Να βρεθούν οι συντεταγμένες του κέντρου βάρους του ομογενούς $(\delta(x,y)=1)$ χωρίου που περικλείεται από τις καμπύλες $y^{2}=2x$, $x=2$ και $y=0$.

  \hfill Απ: $\overline{x}=\frac{6}{5}$, $\overline{y}=\frac{3}{4}$

  \item  Να υπολογιστούν το κέντρο μάζας και η ροπή αδράνειας ως προς την αρχή των αξόνων της επιφάνειας που περικλείεται από τις καμπύλες $y=x^{3}$ και  $y=4x$ με $x,y\geq 0$, όταν $\delta(x,y)=1$.

  \hfill Απ: $\overline{x}=\frac{16}{15}$, $\overline{y}=\frac{64}{21}$, $I_{o}=\frac{848}{15}$

  \item Να βρεθούν οι συντεταγμένες του κέντρου μάζας του στερεού χωρίου $V$ που είναι το πρώτο όγδοο σφαίρας ακτίνας $R=1$ όταν η κατανομή μάζας στο χωρίο, δίνεται από τη συνάρτηση πυκνότητας $\delta(x,y)=1$.

  \hfill Απ: $\overline{x}=\frac{3}{8}$, $\overline{y}=\frac{3}{8}$, $\overline{z}=\frac{3}{8}$

  \item Να υπολογιστεί η συνολική μάζα του στερεού χωρίου $V$ που περικλείεται από τις επιφάνειες $z^{2}=x^{2}+y^{2}$, $z=0$ και $z=1$, όταν η συνάρτηση πυκνότητας μάζας είναι $\delta(x,y)=kz$, $k\in\mathbb{R}$.

  \hfill Απ: $M=\frac{k\pi}{4}$

  \item Να βρεθούν οι συντεταγμένες του κέντρου βάρους του στερεού $V$ που περικλείεται από τις επιφάνειες $z=x^{2}+y^{2}$, $x^{2}+y^{2}=2x$ και το επίπεδο $z=0$ όταν η συνάρτηση πυκνότητας μάζας είναι $\delta(x,y,z)=1$.

  \hfill Απ: $\overline{x}=\frac{4}{3}$, $\overline{y}=0$, $\overline{z}=\frac{10}{9}$

\end{enumerate}


\end{document}
