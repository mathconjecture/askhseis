\documentclass[a4paper,table]{report}
\input{preamble_ask.tex}
\input{definitions_ask.tex}


% \DeclareMathOperator{\r}{r}

\pagestyle{askhseis}

\begin{document}

\begin{center}
\minibox{\large\bfseries\textcolor{Col1}{Ορίζουσες}}
\end{center}

\vspace{\baselineskip}

\begin{enumerate}
  \item Να λυθεί στο $\mathbb{R}$ η εξίσωση $ D=0 $, όταν:
    \begin{enumerate}[i)]
      \item 
        $
        D = 
        \begin{vmatrix*}[r]
          x+1 & x+2 & x+3 & x+4 \\
          2x-3 & 2x-3 & 0 & 2x-3 \\
          2 & 1 & 1 & 3 \\
          x & x & 1 & 1
        \end{vmatrix*} 
        $
        \hfill Απ: $ x=3/2 $, $x=0$, $ x=-3 $   
      \item 
        $
        D=
        \begin{vmatrix*}[r]
          x & x & x & x + \lambda \\
          x & x & x+ \lambda & x \\
          x & x+ \lambda & x & x \\
          x+ \lambda & x & x & x 
        \end{vmatrix*}
        $
        \hfill Απ:
        \begin{tabular}{l}
          αν $\lambda =0$ τότε αληθεύει $ \forall x \in \mathbb{R} $ \\
          αν $\lambda \neq 0$ τότε αληθεύει για $ x=- \lambda /4 $ 
        \end{tabular} 
    \end{enumerate}

  \item Να υπολογιστεί η τιμή της ορίζουσας
    \[
      D = 
      \begin{vmatrix*}[r]
        \sin{a} & \cos{a} & \sin{(a+x)} \\
        \sin{b} & \cos{b} & \sin{(b+x)} \\
        \sin{c} & \cos{c} & \sin{(c+x)} \\
      \end{vmatrix*} 
    \]
    \hfill \textcolor{Col2}{Υπόδειξη:} 
    $ \sin{(a+b) = \sin{a} \cos{b} + \cos{a} \sin{b}} $ \quad Απ: $ 0 $  

  \item Να αποδείξετε ότι
    \[
      \begin{vmatrix*}[r]
        b_{1}+ c_{1} & c_{1}+ a_{1} & a_{1}+ b_{1} \\
        b_{2}+ c_{2} & c_{2}+ a_{2} & a_{2}+ b_{2} \\
        b_{3}+ c_{3} & c_{3}+ a_{3} & a_{3}+ b_{3} 
      \end{vmatrix*} = 2 
      \begin{vmatrix*}[r]
        a_{1}& b_{1}& c_{1} \\
        a_{2}& b_{2}& c_{2} \\
        a_{3}& b_{3}& c_{3} 
      \end{vmatrix*}
    \]

  \item Να υπολογίσετε την $ 5\times 5$ ορίζουσα $ D = 
    \begin{pmatrix*}[r]
    \mathbb{O} & \mathbb{I}_{3} \\
  \mathbb{I}_{2} & \mathbb{O} \\
\end{pmatrix*} $ όπου το επάνω αριστερά $ \mathbb{O} $ είναι ο μηδενικός $ 3\times 2 $ 
πίνακας και το κάτω δεξιά $ \mathbb{O} $ είναι ο μηδενικός $ 2\times 3 $ πίνακα.
\hfill Απ: $ 1 $ 

\item Αν ο πίνακας $ A \in \textbf{M}_{n}(\mathbb{C}) $ είναι αντισυμμετρικός 
  και $n$ περιττός, να αποδείξετε ότι $ |A|=0 $.

\item Να βρεθούν οι τιμές του $ x $ για τις οποιές ισχύει
  \[
    \begin{vmatrix*}[r]
      x & x & x & x & x \\
      a & x & x & x & x \\
      a & a & x & x & x \\
      a & a & a & x & x \\
      a & a & a & a & x
    \end{vmatrix*} = 0
   \]
   \hfill Απ: $ x=0 $, $ x=a $  
\end{enumerate}



\end{document}
