\documentclass[a4paper,table]{report}
\input{$HOME/Desktop/preamble/preamble.tex}
\input{$HOME/Desktop/preamble/definitions.tex}


\everymath{\displaystyle}

\thispagestyle{empty}


\begin{document}

\begin{center}
    \fbox{\large\bfseries Ασκήσεις στη σύνθεση συναρτήσεων }
\end{center}

\vspace{\baselineskip}


\begin{enumerate}
    \item Έστω $ f(x) = 2x-3 $, $ g(x) = x^{2}+4 $ και $ h(x) = x-1 $. 
        Να υπολογίσετε τις 
        \begin{enumerate}[i)]
            \item $ f \circ g $ και $ g \circ f $
            \item $ f \circ f $ και $ g \circ g $
            \item $ h \circ h \circ h $
        \end{enumerate}

        \hfill Απ:  \begin{tabular}{l}
            $ f \circ g = 2x^{2}+5 $, $ g \circ f = (2x-3)^{2}+4 $ \\
            $ f \circ f = 4x-9 $, $ g \circ g = x^{2}+8 $ \\
            $ h \circ h \circ h = x-3 $
        \end{tabular}

    \item Έστω $ f(x) = x^{2}+1 $, με $ x>0 $ και $ g(x) = -x+3 $, με $ x>3 $. 
        Να υπολογίσετε τις συναρτήσεις $ f \circ g $ και $ g \circ f $.

        \hfill Απ: \begin{tabular}{l}
            $ f \circ g : $ δεν ορίζεται \\
            $ D_{g \circ f}=(\sqrt{2}, +\infty) $, $ g \circ f = -x^{2}+2 $
        \end{tabular}

    \item Έστω $ f(x) = \sqrt{x+1} $ και $ g(x) = \ln{(x-1)} $. Να υπολογίσετε 
        τις συναρτήσεις $ f \circ g $ και $ g \circ f $.

        \hfill Απ:  \begin{tabular}{l}
            $ D_{f \circ g}=[1,+\infty) $, $ f \circ g = \sqrt{\ln{x}} $ \\
            $ D_{g \circ f}=(-1,+\infty) $, $ g \circ f = \ln{\sqrt{x+1}} -1$
        \end{tabular}

    \item Έστω $ f(x) = \sqrt{x-2} $ και $ g(x) = \frac{1}{x-1} $. Να υπολογίσετε 
        τηυ συνάρτηση $ g \circ f $.

        \hfill Απ:  \begin{tabular}{l}
            $ D_{g \circ f}= [2,3) \cup (3,+\infty) $, 
            $ g \circ f = \frac{1}{\sqrt{x-2} -1} $
        \end{tabular}
\end{enumerate}








\end{document}



