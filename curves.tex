\input{preamble_ask.tex}
\input{definitions_ask.tex}
\input{tikz.tex}
\input{myboxes.tex}

\geometry{left=1.5cm,right=1.5cm}

\begin{document}

\begin{center}
  \minibox{\bfseries\large \textcolor{Col1}{Ασκήσεις στις Καμπύλες}} 
\end{center} 

\vspace{\baselineskip} 


\section*{Καμπύλες}


\begin{enumerate}
  \item Αν οι παρακάτω καμπύλες αποτελούν την τροχιά ενός σωματιδίου που κινείται 
    στο χώρο, να βρεθούν:
    \begin{myitemize}
      \item Η ταχύτητα και η επιτάχυνση του σωματιδίου κάθε χρονική στιγμή.
      \item Το μέτρο της ταχύτητας, καθώς και η κατεύθυνση της κίνησης του σωματιδίου 
        την δοσμένη χρονική στιγμή.
      \item Να γραφεί η ταχύτητα τη δοσμένη χρονική στιγμή, ως το γινόμενο του μέτρου 
        της ταχύτητας και της κατεύθυνσης της κίνησης.
    \end{myitemize}

    \begin{enumerate}[i)]
      %thomas 13.1 ex. 9 p. 714
      \item $ \mathbf{r}(t)= (t+1)\, \mathbf{i} + (t^{2}-1)\, \mathbf{j} + 2t \,
        \mathbf{k} $ στο σημείο $ t=1 $.
        \hfill Απ: $ \mathbf{v}(1) = 3(1/3 \mathbf{i}+2/3 \mathbf{j}+2/3 \mathbf{k}) $ 

      %thomas 13.1 ex. 11 p. 714
      \item $ \mathbf{r}(t)=2 \cos{t}\, \mathbf{i} + 3 \sin{t}\, \mathbf{j} + +4t \,
        \mathbf{k} $ στο σημείο $ t= \pi /2 $.
        \hfill Απ: $ \mathbf{v}(\pi /2) = 2 \sqrt{5} (-1/ \sqrt{5} \mathbf{i}+2/
        \sqrt{5} \mathbf{k}) $ 

      %thomas 13.1 ex. 14 p. 714
      \item $ \mathbf{r}(t)= \mathrm{e}^{-t}\, \mathbf{i} + 2 \cos{3t}\, \mathbf{j} + 2
        \sin{3t} \, \mathbf{k} $ στο σημείο $ t=0 $.
        \hfill Απ: $ \mathbf{v}(0) = \sqrt{37} (-1/ \sqrt{37} \mathbf{i}+ 6/
        \sqrt{37} \mathbf{k}) $ 
    \end{enumerate}

  \item Αν οι παρακάτω καμπύλες αποτελούν την τροχιά ενός σωματιδίου που κινείται 
    στο χώρο, να βρεθεί η γωνία που σχηματίζει το διάνυσμα της ταχύτητας και της 
    επιτάχυνσης του σωματιδίου, τη χρονική στιγμή $ t=0 $.
    \begin{enumerate}[i)]
      %thomas 13.1 ex. 15 p. 714
      \item $ \mathbf{r}(t)=(3t+1)\, \mathbf{i} + \sqrt{3} t\, \mathbf{j} + t^{2} \, 
        \mathbf{k} $
        \hfill Απ: $ \theta = \pi /2 $
      %thomas 13.1 ex. 16 p. 714
      \item $ \mathbf{r}(t)=(\frac{\sqrt{2}}{2} t)\, \mathbf{i} + 
        (\frac{\sqrt{2}}{2} t-16t^{2})\, \mathbf{j} $
        \hfill Απ: $ \theta = 3\pi /4 $ 
      %thomas 13.1 ex. 18 p. 714
      \item $ \mathbf{r}(t)=4/9(1+t)^{3/2}\, \mathbf{i} + 4/9(1-t)^{3/2}\, \mathbf{j} + 
        \frac{1}{3} t\, \mathbf{k} $
        \hfill Απ: $ \theta = \pi /2 $ 
    \end{enumerate}

  \item Αν οι παρακάτω καμπύλες αποτελούν την τροχιά ενός σωματιδίου που κινείται 
    στο χώρο, να βρεθούν οι παραμετρικές εξισώσεις της εφαπτομένης της καμπύλης, τη 
    δοσμένη χρονική στιγμή.
    \begin{enumerate}[i)]
      %thomas 13.1 ex. 19 p. 714
      \item $ \mathbf{r}(t)= \sin{t}\, \mathbf{i} + (t^{2}- \cos{t})\, \mathbf{j} +
        \mathrm{e}^{t} \, \mathbf{k} $ στο σημείο $ t=0 $.
        \hfill Απ: $ x=t,\;y=-1,\;z=1+t $ 
      %thomas 13.1 ex. 20 p. 714
      \item $ \mathbf{r}(t)=t^{2}\, \mathbf{i} + +(2t-1)\, \mathbf{j} + t^3 \, \mathbf{k}
        $ στο σημείο $ t=2 $.
        \hfill Απ: $ x=4+4t,\;y=3+2t,\;z=8+12t $
      %thomas 13.1 ex. 22 p. 714
      \item $ \mathbf{r}(t)= \cos{t}\, \mathbf{i} + \sin{t}\, \mathbf{j} + \sin{2t} \,
        \mathbf{k} $ στο σημείο $ t= \pi /2 $. 
        \hfill Απ: $ x=-t,\;y= 1,\; z=-2t $
    \end{enumerate}

  \item Να βρεθεί το διάνυσμα θέσης $ \mathbf{r}(t) $ του σωματιδίου, αν είναι γνωστή η
    ταχύτητα ή η επιτάχυνσή του κάθε χρονική στιγμή.
    \begin{enumerate}[i)]
      %thomas 13.2 ex. 11 p. 714
      \item $ \mathbf{v}(t) = -t \mathbf{i}- t \mathbf{j}- t \mathbf{k} $ με αρχ.
        θέση $ \mathbf{r}(0)= \mathbf{i}+2 \mathbf{j}+3 \mathbf{k} $
        \hfill Απ: $ \mathbf{r}(t) = (-\frac{t^{2}}{2} +1) \mathbf{i}+(-
        \frac{t^{2}}{2} +2) \mathbf{j}+ (- \frac{t^{2}}{2} +3) \mathbf{k} $ 

      %thomas 13.2 ex. 12 p. 714
      \item $ \mathbf{v}(t) =(180t) \mathbf{i}+(180t-16t^{2}) \mathbf{j} $ με αρχ. 
        θέση $ \mathbf{r}(0)=100 \mathbf{j} $
        \hfill Απ: $ \mathbf{r}(t) = 90t^{2} \mathbf{i}+(90t^{2}- \frac{16}{3} t^{3}+100)
        \mathbf{j} $ 

      %thomas 13.2 ex. 14 p. 714
      \item $ \mathbf{v}(t) = (t^{3}+4t) \mathbf{i} +t \mathbf{j}+2t^{2} \mathbf{k} $ 
        με αρχ. θέση $ \mathbf{r}(0) = \mathbf{i}+ \mathbf{j} $
        \hfill Απ: $ \mathbf{r}(t) = (\frac{t^{4}}{4} + 2t^{2}+1) \mathbf{i}+
        (\frac{t^{2}}{2}+1) \mathbf{j}+ \frac{2t^{3}}{3} \mathbf{k} $  

      %thomas 13.2 ex. 15 p. 714
      \item $ \mathbf{a}(t) = -32 \mathbf{k} $ με αρχ. θέση $ \mathbf{r}(0) = 100
        \mathbf{k} $ και αρχ. ταχύτ. $ \mathbf{v}(0)=8 \mathbf{i}+8 \mathbf{j} $
        \hfill Απ: $ \mathbf{r}(t) = 8t \mathbf{i} +8t \mathbf{j}+ (100-16t^{2}
        \mathbf{k}) $ 

      %thomas 13.2 ex. 16 p. 714
      \item $ \mathbf{a}(t) = -(\mathbf{i}+ \mathbf{j}+ \mathbf{k}) $ με 
        αρχ. θέση $ \mathbf{r}(0) = 10 (\mathbf{i}+\mathbf{j}+\mathbf{k}) $ 
        και αρχ. ταχ. $ \mathbf{v}(0)= \mathbf{0}$ 
        \hfill Απ: $ \mathbf{r}(t) = (- \frac{t^{2}}{2} +10) (\mathbf{i}+  \mathbf{j}+  
        \mathbf{k}) $ 
    \end{enumerate}

    \section*{Ευθύγραμμη Κίνηση}
    
      %thomas 13.2 ex. 17 p. 714
  \item Τη χρονική στιγμή $t=0$ ένα σωματίδιο βρίσκεται στη θέση $ (1,2,3) $ και 
    κινείται ευθύγραμμα προς στο σημείο $ (4,1,4) $ με αρχική ταχύτητα (μέτρο) $ 2 $ 
    και σταθερή επιτάχυνση $ \mathbf{a}= 3 \mathbf{i}- \mathbf{j}+ \mathbf{k} $. 
    Να βρείτε το διάνυσμα θέσης $ \mathbf{r}(t) $ του σωματιδίου κάθε χρονική στιγμή.

    \hfill Απ: $ \mathbf{r}(t)=(\frac{3}{2} t^{2}+ \frac{6}{\sqrt{11}} t+1)
    \mathbf{i}-(\frac{1}{2} t^{2}+ \frac{2}{\sqrt{11}} t-2) \mathbf{j}+
    (\frac{1}{2} t^{2}+ \frac{2}{\sqrt{11}}t+3) \mathbf{k} $ 

      %thomas 13.2 ex. 18 p. 714
  \item Τη χρονική στιγμή $t=0$ ένα σωματίδιο βρίσκεται στη θέση $ (1,-1,2) $ και 
    κινείται ευθύγραμμα προς στο σημείο $ (3,0,3) $ με αρχική ταχύτητα (μέτρο) $ 2 $ 
    και σταθερή επιτάχυνση $ \mathbf{a}= 2 \mathbf{i}+ \mathbf{j}+ \mathbf{k} $. 
    Να βρείτε το διάνυσμα θέσης $ \mathbf{r}(t) $ του σωματιδίου κάθε χρονική στιγμή.

    \hfill Απ: $ \mathbf{r}(t) = (t^{2}+ \frac{4}{\sqrt{6}} t+1)
    \mathbf{i}+(\frac{1}{2} t^{2}+ \frac{2}{\sqrt{6}} t-1)
    \mathbf{j}+(\frac{1}{2} t^{2}+ \frac{2}{\sqrt{6}} t+2) \mathbf{k} $ 
\end{enumerate}

\end{document}
