\input{$HOME/Desktop/preamble/preamble.tex}
\input{$HOME/Desktop/preamble/definitions.tex}

\usepackage{array}

\pagestyle{empty}

\begin{document}

\begin{center}
  \fbox{\large \bfseries Ασκήσεις Καμπύλες}
\end{center}

\vspace{\baselineskip}

\begin{enumerate}


\item Έστω ένα σώμα διαγράφει σπειροειδή τροχιά προς τα πάνω έχοντας διάνυσμα θέσεως $\rt{3\cos t}{3\sin t}{t^2}$. Να βρεθούν:
\begin{enumerate}[i)]
  \item Τα διανύσματα $\vb{v}, \vb{a}$.
  \item Το μέτρο $\abs{\vb{v}}$ την τυχαία χρονική στιγμή $t$.
  \item Τη χρονική στιγμή κατά την οποία $\vb{v}\perp\vb{a}$
\end{enumerate}


\hfill Απ: \begin{tabular}{>{$}l<{$}>{$}l<{$}}
    \rm{i)} & \vect{v(t)}{-3\sin t}{3\cos t}{2t} \\
      & \vect{a(t)}{-3\cos t}{(-3)\sin t}{2} \\
     \rm{ii)} &  \|\vb{v(t)}\|=\sqrt{9+4t^{2}} \\
     \rm{iii)} &  t=0
\end{tabular}

\item Ένα σώμα κινείται επί της ελικοειδούς καμπύλης $\rt{\cos t}{\sin t}{t}$. Πόσο είναι το μήκος της καμπύλης από $t=0$ έως $t=2\pi$.

\hfill Απ: $s=2\sqrt{2}\pi$

\item Αν η επιτάχυνση ενός σώματος δίνεται από την εξίσωση $\vect{a}{}{2}{6t}$ τότε να βρείτε την εξίσωση της ταχύτητας και της θέσης του σώματος αν είναι γνωστή η αρχική ταχύτητα $\vb{v}(0)=\vb{j}-\vb{k}$ και η αρχική θέση $\vb{r}(0)=\vb{i}-2\,\vb{j}+3\,\vb{k}$.

\hfill Απ: \begin{tabular}{>{$}l<{$}}
  \vb{v}(t)=t\,\vb{i}+(2t+1)\,\vb{j}+(3t^{2}-1)\,\vb{k} \\
  \vb{r}(t)=(\frac{1}{2}t^{2}+1)\,\vb{i}+(t^{2}+t-2)\,\vb{j}+(t^{3}-t+3)\,\vb{k}
\end{tabular}


\item Ένα σωμάτιο κινείται σε τροχιά με παραμετρικές εξισώσεις $x(t)=t\sin t + \cos t$ και $y(t)=-t\cos t+\sin t$, όπου $t$ είναι ο χρόνος.
\begin{enumerate}[i)]
  \item Να βρεθεί το μήκος $s$ της τροχιάς στο διάστημα $t=0$ έως $t=2$.
  \item Να βρεθούν η εφαπτόμενη (επιτρόχιος) και η κάθετη (κεντρομόλος) συνιστώσα της επιτάχυνσης όταν $t=2$.
  \item Να βρεθεί η θέση του κέντρου καμπυλότητας όταν $t=2$.
\end{enumerate}

\hfill Απ: $\begin{tabular}{>{$}l<{$}>{$}l<{$}}
  \rm{i)} & s=2 \\
  \rm{ii)} & \vb{\alpha}=\vb{T}+t\vb{N} \\
  \rm{iii)} & K(\cos t, \sin t), t=2
\end{tabular}$


\end{enumerate}



\end{document}
