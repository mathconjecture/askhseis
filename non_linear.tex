\input{preamble/preamble_ask.tex}
\input{preamble/definitions.tex}


\thispagestyle{empty}


\begin{document}

\begin{center}
    \minibox[frame]{\bfseries \large Μη Γραμμικά Συστήματα}
\end{center}

\vspace{\baselineskip}

\begin{enumerate}
    \item Να υπολογιστούν και να χαρακτηριστούν τα κρίσιμα σημεία των παρακάτων 
        μη-γραμμικών συστημάτων.

        \begin{enumerate}[i)]
            \setlength\itemsep{2\baselineskip}
        \item $ 
            \left.
                \begin{matrix*}[l]
                    x' = (2+x)(y-x) \\
                    y' = (4-x)(y+x) 
                \end{matrix*}
            \right\}$ \hfill Απ:  
            \begin{tabular}{ll}
                $ (0,0) $ & σάγμα, ασταθές \\
                $ (4,4) $ & εστία ,ασυμπτ. ευσταθής \\
                $ (-2,2) $ & κόμβος, ασταθής
            \end{tabular} 

        \item $ 
            \left.
                \begin{matrix*}[l]
                    x' = 1-y \\
                    y' = x^{2} - y^{2}
                \end{matrix*}
            \right\}$ \hfill Απ:  
            \begin{tabular}{ll}
                $ (-1,1) $ & σάγμα, ασταθές  \\
                $ (1,1) $ & εστία, ασυμπτ. ευσταθής
            \end{tabular}  

        \item $ 
            \left.
                \begin{matrix*}[l]
                    x' = - (x-y)(1-x-y) \\
                    y' = x(2+y)
                \end{matrix*}
            \right\} $ \hfill Απ: \begin{tabular}{ll}
                $ (0,0) $ & σάγμα, ασταθές \\
                $ (-2,2) $ & κόμβος, ασυμπτ. ευσταθής \\
                $ (0,1) $ & εστία, ασυμπτ. ευσταθής \\
                $ (3 -2) $ & κόμβος, ασταθής
            \end{tabular} 

        \item $ 
            \left.
                \begin{matrix}
                    x' = x-y^{2} \\
                    y' = y-x^{2}
                \end{matrix}
            \right\}$ \hfill Απ: \begin{tabular}{ll}
                $ (0,0) $ & άστρο, ασταθές \\
                $ (1,1) $ & σάγμα, ασταθές
            \end{tabular}

        \item $ 
            \left.
                \begin{matrix}
                    x' = 1-xy \\
                    y' = x-y^{3}
                \end{matrix}
            \right\}$ \hfill Απ: \begin{tabular}{ll}
                $ (1,1) $ & εκφ. κόμβος, ασταθής \\
                $ (-1,-1) $ & σάγμα, ασταθές
            \end{tabular} 

        \item $ 
            \left.
                \begin{matrix*}[l]
                    x' = (1+x) \sin{y} \\
                    y' = 1-x - \cos{y}
                \end{matrix*}
            \right\}$ \hfill Απ: \begin{tabular}{ll}
                $ (0,2k \pi) $ & κέντρο, ευσταθές \\
                $ (2, (2k+1) \pi) $ & σάγμα, ασταθές
            \end{tabular} 

        \end{enumerate}

    \item Έστω τα μη-γραμμικά συστήματα 
        \begin{align}
            \left.
                \begin{matrix*}[l]\label{sys1}
                    x' = y+x(x^{2}+y^{2}) \\
                    y' = -x+y(x^{2}+y^{2})
                \end{matrix*}
            \right\} 
            \intertext{και} 
            \left.
                \begin{matrix*}[l]\label{sys2}
                    x' = y-x(x^{2}+y^{2}) \\
                    y' = -x - y(x^{2}+y^{2})
                \end{matrix*}
            \right\} 
        \end{align} 
        \begin{enumerate}[i)]
            \item Να δείξετε ότι το $ (0,0) $ είναι κρίσιμο σημείο και για τα δύο 
                συστήματα, καθώς και ότι είναι κέντρο αντίστοιχα.
            \item Για το σύστημα~\eqref{sys2}, να δείξετε ότι $ r' < 0 $ και ότι 
                $ r \to 0 $ καθώς $ t \to \infty $, άρα το $ (0,0) $ είναι 
                ασυμπτωτικά ευσταθές.
            \item Για το σύστημα~\eqref{sys1}, να δείξετε ότι η λύση του αντίστοιχου 
                ΠΑΤ, για την $r$, με $ r(0)=r_{0} $, είναι μη-φραγμένο, καθώς 
                $ t \to 1/2 {r_{0}}^{2} $, άρα το $ (0,0) $ είναι ασταθές.
        \end{enumerate}  
\end{enumerate}

\end{document}
