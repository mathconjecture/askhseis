\input{preamble.tex}
\input{definitions_ask.tex}

\geometry{top=2.5cm}
\pagestyle{askhseis}

\renewcommand{\vec}{\mathbf}

\newcommand{\twocolumnsidesll}[2]{\begin{minipage}[c]{0.65\linewidth}
        #1
        \end{minipage}\hfill\begin{minipage}[c]{0.30\linewidth}
        #2
    \end{minipage}
}




\begin{document}

\begin{center}
  \minibox{\large \bfseries \textcolor{Col1}{Λύσεις Ασκήσεων στα Διανύσματα}}
\end{center}

\vspace{\baselineskip}

\begin{enumerate}[start=9]
  \item Να δείξετε ότι 
    \begin{enumerate}[(i)]
      \item $ \norm{\mathbf{u}+ \mathbf{v}}^{2} - \norm{\mathbf{u}- \mathbf{v}}^{2} 
        = 4 \mathbf{u}\cdot \mathbf{v} $
      \item $ \norm{\mathbf{u}+ \mathbf{v}}^{2} + \norm{\mathbf{u}- \mathbf{v}}^{2} 
        = 2 \norm{\mathbf{u}} ^{2} + 2 \norm{\mathbf{v}} ^{2} $
    \end{enumerate}
    \begin{solution}
    \item {}
      \begin{enumerate}[i)]
        \item 
          \begin{align*}
            \norm{\mathbf{u}+ \mathbf{v}}^{2} - \norm{\mathbf{u}- \mathbf{v}}^{2} 
            &= (\mathbf{u}+ \mathbf{v})\cdot (\mathbf{u}+ \mathbf{v}) - (\mathbf{u}-
            \mathbf{v}) \cdot (\mathbf{u}- \mathbf{v}) \\ 
            &=
            \norm{\mathbf{u}}^{2} + \mathbf{u} \cdot \mathbf{v} + \mathbf{v}
            \cdot \mathbf{u} +
            \norm{\mathbf{v}}^{2} - (\norm{\mathbf{u}}^{2} - \mathbf{u} \cdot  
            \mathbf{v} - \mathbf{v} \cdot \mathbf{u} +  \norm{\mathbf{v}}^{2}) 
            & \textcolor{Col1}{(\mathbf{u}^{2} = \norm{\mathbf{u}}^{2})} \\
            &=
            \norm{\mathbf{u}}^{2} + 2 \mathbf{u} \cdot \mathbf{v} +
            \norm{\mathbf{v}}^{2} - \norm{\mathbf{u}}^{2} + 2 \mathbf{u}
            \cdot \mathbf{v} - \norm{\mathbf{v}}^{2} 
            & \textcolor{Col1}{(\mathbf{u} \cdot \mathbf{v} = \mathbf{v} \cdot
            \mathbf{u})} \\ 
            &= 4 \mathbf{u} \cdot \mathbf{v} 
          \end{align*} 
        \item 
          \begin{align*}
            \norm{\mathbf{u}+ \mathbf{v}}^{2} + \norm{\mathbf{u}- \mathbf{v}}^{2} 
            &= (\mathbf{u}+ \mathbf{v})\cdot (\mathbf{u}+ \mathbf{v}) + (\mathbf{u}-
            \mathbf{v}) \cdot (\mathbf{u}- \mathbf{v}) \\ 
            &=
            \norm{\mathbf{u}}^{2} + \mathbf{u} \cdot \mathbf{v} + \mathbf{v}
            \cdot \mathbf{u} +
            \norm{\mathbf{v}}^{2} + \norm{\mathbf{u}}^{2} - \mathbf{u} \cdot  
            \mathbf{v} - \mathbf{v} \cdot \mathbf{u} +  \norm{\mathbf{v}}^{2} \\
            &=
            \norm{\mathbf{u}}^{2} + 2 \mathbf{u} \cdot \mathbf{v} +
            \norm{\mathbf{v}}^{2} + \norm{\mathbf{u}}^{2} - 2 \mathbf{u}
            \cdot \mathbf{v} + \norm{\mathbf{v}}^{2} \\ 
            &= 2 \norm{\mathbf{u}}^{2} + 2 \norm{\mathbf{v}}^{2}
          \end{align*}
      \end{enumerate}
    \end{solution}

  \item Να δείξετε ότι οι παρακάτω ισότητες αληθεύουν αν και μόνον αν 
    τα διανύσματα $ \mathbf{a}, \mathbf{b} $ είναι γραμμικώς εξαρτημένα.
    \begin{enumerate}[(i)]
      \item $(\vec{a}\cdot \vec{b})^{2} = \vec{a}^{2}\cdot \vec{b}^{2}$
      \item $|\vec{a}\cdot \vec{b}| = ||\vec{a}|| \cdot ||\vec{b}||$
    \end{enumerate}
    \begin{solution}
    \item {}
      \begin{enumerate}[i)]
        \item Αν $ \mathbf{a}= \mathbf{0} $ ή $ \mathbf{b}= \mathbf{0} $ (οπότε 
          τα $ \mathbf{a} $ και $ \mathbf{b} $ είναι γραμμικώς εξαρτημένα), τότε 
          προφανώς ισχύουν αυτές οι ισότητες. 
          Αν $ \mathbf{a} \neq \mathbf{0} $ και $ \mathbf{b} \neq 0 $ τότε έχουμε:
          \begin{align*}
            (\mathbf{a}\cdot \mathbf{b})^{2} 
            &= \mathbf{a}^{2} \cdot \mathbf{b} ^{2} \Leftrightarrow \\ 
            (\norm{\mathbf{a}} \norm{\mathbf{b}} \cos{\theta})^{2} 
            &= \norm{\mathbf{a}} ^{2} \norm{\mathbf{b}} ^{2}
            \Leftrightarrow \\ 
            \norm{\mathbf{a}} ^{2} \norm{\mathbf{b}} ^{2} \cos^{2}{\theta} 
            &= \norm{\mathbf{a}} ^{2} \norm{\mathbf{b}} ^{2} 
            \overset{\mathbf{a}, \mathbf{b} \neq \mathbf{0}}{\Leftrightarrow} \\
            \cos^{2}{\theta} &= 1 \Leftrightarrow \\ 
            \cos{\theta} &= \pm 1 \Leftrightarrow \\
            \theta = 0 \; &\text{ή} \; \theta = \pi \Leftrightarrow \\
            \mathbf{a}, \mathbf{b} &\; \text{παράλληλα} \\
            \mathbf{a}, \mathbf{b} &\; \text{γραμμικώς εξαρτημένα} 
          \end{align*}
        \item
          \begin{align*}
            \abs{\mathbf{a} \cdot \mathbf{b}} 
            &= \norm{\mathbf{a}} \norm{\mathbf{b}} \Leftrightarrow \\
            \abs{\norm{\mathbf{a}} \norm{\mathbf{b}} \cos{\theta}} 
            &= \norm{\mathbf{a}} \norm{\mathbf{b}} 
            \overset{\mathbf{a}, \mathbf{b} \neq \mathbf{0}}{\Leftrightarrow} \\
            \abs{\cos{\theta}} &= 1 \Leftrightarrow \\
            \cos{\theta} &= \pm 1 \\
            \theta = 0 \; &\text{ή} \; \theta = \pi \Leftrightarrow \\
            \mathbf{a}, \mathbf{b} &\; \text{παράλληλα} \\
            \mathbf{a}, \mathbf{b} &\; \text{γραμμικώς εξαρτημένα} 
          \end{align*}
      \end{enumerate}
    \end{solution}

  \item Να δείξετε ότι $ (\vec{a}\times \vec{b})^{2} + (\vec{a}\cdot \vec{b})^{2} =
    \vec{a}^{2}\cdot \vec{b}^{2} $  
    \begin{solution}
      \begin{align*}
        (\vec{a}\times \vec{b})^{2} + (\vec{a}\cdot \vec{b})^{2} 
        &= \norm{\mathbf{a} \times \mathbf{b}}^{2} + (\mathbf{a}\cdot \mathbf{b})^{2} \\
        &= (\norm{\mathbf{a}} \norm{\mathbf{b}} \sin{\theta})^{2} + (\norm{\mathbf{a}}
        \norm{\mathbf{b}} \cos{\theta})^{2} \\
        &= \norm{\mathbf{a}}^{2} \norm{\mathbf{b}
        }^{2} (\sin^{2}{\theta } + \cos^{2}{\theta}) \\
        &= \mathbf{a}^{2} \mathbf{b}^{2}
    \end{align*}
  \end{solution}

\item Να δείξετε ότι
  \begin{enumerate}[i)]
    \item $ ( \vec{a} - \vec{b} ) \times ( \vec{a} + \vec{b} ) = 2 (\vec{a} \times 
      \vec{b}) $
    \item Πώς ερμηνεύεται γεωμετρικά το παραπάνω αποτέλεσμα αν τα διανύσματα 
      $ \vec{a} $ και $ \vec{b} $ είναι γραμμικώς ανεξάρτητα;
  \end{enumerate}
  \begin{solution}
    \begin{align*}
      (\mathbf{a}- \mathbf{b}) \times (\mathbf{a}+ \mathbf{b}) = (\mathbf{a} \times
      \mathbf{a}) + (\mathbf{a} \times \mathbf{b}) - (\mathbf{b} \times  \mathbf{a}) - 
      (\mathbf{b} \times \mathbf{b}) = \mathbf{0} + (\mathbf{a} \times \mathbf{b}) +
      (\mathbf{a} \times \mathbf{b}) - \mathbf{0} = 2 (\mathbf{a} \times \mathbf{b})
    \end{align*}
  \end{solution}

  \subsection*{Γεωμετρική Ερμηνεία}
  
\twocolumnsidesll{
  Αν $ \mathbf{a} $ και $ \mathbf{b} $ όπως στο σχήμα, τότε το παραλληλόγραμμο έχει 
  εμβαδό $ E= \norm{\mathbf{a} \times \mathbf{b}} $. Επομένως η παραπάνω ισότητα, δείχνει
  ότι το παραλληλόγραμμο που ορίζουν τα διανύσματα $ \mathbf{a} - \mathbf{b} $ και 
  $ \mathbf{a}+ \mathbf{b} $ έχει διπλάσιο εμβαδό από αυτό που ορίζουν τα 
  $ \mathbf{a} $ και $ \mathbf{b} $.
}{
\begin{tikzpicture}[>=latex]
		\node  (0) at (0, 0) {};
		\node  (1) at (3, 0) {};
		\node  (2) at (1, 1.6) {};
		\node  (3) at (4, 1.6) {};
		\draw[very thick,->,Col1] (0.center) to node[midway,below] 
      {$ \mathbf{a}$} (1.center);
		\draw (1.center) to (3.center);
		\draw (3.center) to (2.center);
		\draw[very thick,<-,Col1] (2.center) to node[midway,left] 
      {$ \mathbf{b} $} (0.center);
		\draw[very thick,->] (0.center) to node[pos=0.4,above left,sloped] 
      {$ \mathbf{a}+ \mathbf{b} $} (3.center);
		\draw[very thick,->] (2.center) to node[pos=0.35,above,sloped] 
      {$ \mathbf{a}- \mathbf{b} $} (1.center);
\end{tikzpicture}
}

\end{enumerate}

\begin{enumerate}[start=15]
  \item Να δείξετε ότι τα σημεία $ A(1,2,3) $, $ B(4,2,4) $, $ C(2,4,0) $ και 
    $ D(-1,1,5) $ είναι κορυφές \textbf{τετραέδρου} και στη συνέχεια να υπολογίσετε τον 
    όγκο του. 
    \begin{solution}
    \item {}
      \twocolumnsidesll{
        Σχηματίζουμε τα διανύσματα:
        \begin{myitemize}
          \item $ \mathbf{AB} = (4-1,2-2,4-3) = (3,0,1) $
          \item $ \mathbf{AC} = (2-1,4-2,0-3) = (1,2,-3) $
          \item $ \mathbf{AD} = (-1-1,1-2,5-3) = (-2,-1,2) $
        \end{myitemize}
      }{
        \begin{tikzpicture}[>=stealth,scale=0.8]
          \node (a) at (0, 0) {};
          \node at (a) [left] {$A$} ;
          \node (b) at (3, 0) {};
          \node (c) at (0.5, 2) {};
          \node (d) at (3.5, 2) {};
          \node (e) at (1.5, 0.75) {};
          \node at (e) [left=5pt] {$C$} ;
          \node at (c) [left=5pt] {$D$} ;
          \node at (b) [right=5pt] {$B$} ;
          \node (f) at (4.5, 0.75) {};
          \node (g) at (2, 2.75) {};
          \node (h) at (5, 2.75) {};
          \draw[dashed] (c.center) to (d.center);
          \draw[dashed] (d.center) to (b.center);
          \draw[dashed] (e.center) to (f.center);
          \draw[dashed] (e.center) to (g.center);
          \draw[dashed] (g.center) to (h.center);
          \draw[dashed] (h.center) to (f.center);
          \draw[dashed] (d.center) to (h.center);
          \draw[dashed] (b.center) to (f.center);
          \draw[dashed] (c.center) to (g.center);
          \draw[very thick,->,Col1] (a.center) to (b.center);
          \draw[very thick,->,Col1] (a.center) to (c.center);
          \draw[very thick,->,Col1] (a.center) to (e.center);
          \draw[fill=blue!15] (c.center) -- (b.center) -- (e.center) -- cycle ;
        \end{tikzpicture}
      }
      Υπολογίζουμε το μικτό γινόμενο:
      \[
        [\mathbf{AB}, \mathbf{AC}, \mathbf{AD}] = 
        \begin{vmatrix*}[r]
          3 & 0 & 1 \\
          1 & 2 & -3 \\
          -2 & -1 & 2
        \end{vmatrix*} = 6 \neq 0
      \] 
      Επομένως, τα διανύσματα $ \mathbf{AB}, \mathbf{AC}, \mathbf{AD} $, δεν 
      είναι συνεπίπεδα, άρα τα σημεία $ A,B,C,D $ δεν ανήκουν στο ίδιο επίπδο και 
      γι᾽ αυτό σχηματίζουν τετραέδρο (με τη γαλάζια «βάση» στο σχήμα).  
      Για τον όγκο του τετραέδρου, ισχύει ο τύπος
      \[
        V_{\text{τετρ/δου}} = \frac{1}{6} V_{\text{παρ/δου}} = \frac{1}{6}
        \abs{[\mathbf{AB}, \mathbf{AC}, \mathbf{AD}]} = \frac{1}{6} \cdot \abs{6} = 1
      \] 
    \end{solution}

\end{enumerate}


\end{document}

