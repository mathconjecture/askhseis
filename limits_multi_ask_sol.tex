\documentclass[a4paper,table]{report}
\input{preamble_ask.tex}
\input{definitions_ask.tex}
\input{tikz.tex}

\pagestyle{vangelis}
% \everymath{\displaystyle}

\geometry{ left=1cm,right=1cm }


\begin{document}

\begin{center}
  \minibox{\bfseries \textcolor{Col1}{Όρια - Συνέχεια (Λύσεις)}}
\end{center}

\vspace{\baselineskip}

\begin{enumerate}

  \item Να υπολογιστούν τα παρακάτω όρια.

    \begin{myitemize}

      \item $ \lim\limits_{\substack{x\to 0 \\y \to 0}} xy 
        \sin{\Bigl(\frac{1}{y}\Bigr)} $
        \begin{solution}
          \[
            \abs{xy \sin{\Bigl(\frac{1}{y}\Bigr)}} = \abs{xy} 
            \cdot \abs{\sin{\Bigl(\frac{1}{y}\Bigr)}} \leq \abs{xy} \cdot 1 = \abs{xy} 
          \]
          και επειδή $ \lim\limits_{\substack{x\to 0 \\y \to 0}} xy = 0 $, έπεται από 
          γνωστό θεώρημα ότι και 
          \[
            \lim\limits_{\substack{x\to 0 \\y \to 0}} xy 
            \sin{\Bigl(\frac{1}{y}\Bigr)} = 0
          \] 
        \end{solution}

      \item $ \lim\limits_{\substack{x\to 0 \\y \to 0}} 
        \frac{(1+y^{2}) \sin{x} }{x} $
        \begin{solution} 
          $\lim\limits_{\substack{x\to 0 \\y \to 0}} \frac{(1+y^{2}) \sin{x}}{x} = 
          \lim\limits_{y \to 0} (1+y^{2}) \cdot
          \lim\limits_{x \to 0} \frac{\sin{x}}{x} = (1+0^{2}) \cdot 1 =
          1$ 
        \end{solution}

      \item $ \lim\limits_{\substack{x\to 0 \\y \to 0}}
        \frac{\sin{(x^{3}+y^{3})}}{x^{2}+y^{2}} $
        \begin{solution}
        \item {}
          \textbf{Ά Τρόπος:} Με Πολικές

          \textbf{B Τρόπος:} 

          $ \abs{\frac{\sin{(x^{3}+y^{3})}}{x^{2}+y^{2}}} \leq
          \frac{\abs{x^{3}+y^{3}}}{x^{2}+y^{2}} = \abs{\frac{x^{3}}{x^{2}+y^{2}} +
          \frac{y^{3}}{x^{2}+y^{2}}} \leq \abs{\frac{x^{3}}{x^{2}+y^{2}}} +
          \abs{\frac{y^{3}}{x^{2}+y^{2}}} = \abs{x} \frac{x^{2}}{x^{2}+y^{2}} +
          \abs{y} \frac{y^{2}}{x^{2}+y^{2}} \leq \abs{x} \cdot 1 + \abs{y} \cdot 1 = 
          \abs{x} + \abs{y} $

          Όμως $ \lim_{(x,y) \to (0,0)} (\abs{x} + \abs{y}) = 0 $, 
          άρα από γνωστό θεώρημα έχουμε ότι και $ \lim_{(x,y) \to (0,0)}
          \frac{\sin{(x^{3}+y^{3})}}{x^{2}+y^{2}} = 0 $.
        \end{solution}
    \end{myitemize}

  \item Να δείξετε ότι τα παρακάτω όρια δεν υπάρχουν.

    \begin{myitemize}
      \item $ \lim_{(x,y) \to (0,0)} \Bigl(\frac{\abs{y}}{x^{2}} \cdot 2^{
        -\frac{\abs{y}}{x^{2}}}\Bigr) $
        \begin{solution}
          Προσεγγίζουμε το σημείο $ (0,0) $ μέσω των καμπυλών $ y= \lambda x^{2} $.
          \[
            \lim\limits_{\substack{x\to 0 \\y \to 0}} 
            \Biggl(\frac{\abs{y}}{x^{2}} \cdot 2^{-\frac{\abs{y}}{x^{2}}}\Biggr)
            \overset{y= \lambda x^{2}}{=} \lim_{x \to 0} 
            \Biggl(\frac{\abs{\lambda x^{2}}}{x^{2}} \cdot 2^{-\frac{\abs{\lambda
              x^{2}}}{x^{2}}}\Biggr) = \lim_{x \to 0} \Bigl(\abs{\lambda}\cdot 2^{-
            \abs{\lambda}}\Bigr) = \Bigl(\abs{\lambda}\cdot 2^{-
        \abs{\lambda}}\Bigr)
      \] 
      Επομένως δεν υπάρχει το όριο, γιατί εξαρτάται από το $ \lambda $.
    \end{solution}
\end{myitemize}

\item Να εξετάσετε αν οι παρακάτω συναρτήσεις μπορούν να οριστούν κατάλληλα ώστε να 
  είναι συνεχείς.

  \begin{myitemize}
    \item $ f(x,y) =  \Bigl[\sqrt{x^{2}+y^{2}} \ln{(\sqrt{x^{2}+y^{2}})}\Bigr] $
      \begin{solution}
        Παρατηρούμε, ότι στο σημείο $ (0,0) $ η τιμή της $f$ είναι απροσδιόριστη 
        $(  0\cdot (- \infty) ) $. Γνωρίζουμε, επίσης, ότι για να είναι η $f$
        συνεχής στο σημείο $(0,0)$, πρέπει 
        $ \lim_{(x,y)\to (0,0)} f(x,y) = f(0,0) $. 
        Επομένως θα υπολογίσουμε το όριο της $f$ στο $ (0,0) $.
        Προκειμένου να μπορούμε να εφαρμόσουμε τον κανόνα L' H\"opital, για να διώξουμε
        την απροσδιοριστία, αλλά και επειδή, η ποσότητα, $ x^{2}+y^{2} $
        εμφανίζεται δύο φορές, εφαρμόζουμε \textbf{πολικές} συντεταγμένες.
        \begin{align*}
          \lim\limits_{\substack{x\to 0 \\y \to 0}} 
          \Bigl[\sqrt{x^{2}+y^{2}} \ln{(\sqrt{x^{2}+y^{2}})}\Bigr] = \lim_{r \to 0} 
          ( r \cdot \ln{r})
          \overset{0(- \infty)}{=} \lim_{r \to 0} \frac{\ln{r}}{\frac{1}{r}}
          \overset{\left(\frac{0}{0}\right)}{\underset{\mathrm{LH}}{=}} \lim_{r \to
          \infty} = \frac{\frac{1}{r}}{(- \frac{1}{r^{2}})} = - \lim_{r \to 0} r = 0
        \end{align*}
        Επομένως, αρκεί να ορίσουμε $ f(0,0) = 0 $, ώστε να είναι η $f$ συνεχής.
      \end{solution}
    \item $ f(x,y,z) =  
      \dfrac{1 - \cos{\sqrt{(x^{2}+y^{2}+z^{2}})}}{x^{2}+y^{2}+z^{2}} $
      \begin{solution}
        Παρατηρούμε, ότι στο σημείο $ (0,0,0) $ η τιμή της $f$ είναι απροσδιόριστη 
        $ (0/0) $. Γνωρίζουμε, επίσης, ότι για να είναι η $f$ συνεχής στο σημείο 
        $ (0,0,0) $, πρέπει 
        $\lim_{(x,y,z) \to (0,0,0)} f(x,y,z) = f(0,0,0)$. 
        Οπότε, έχουμε
        \begin{align*}
          \lim\limits 
          \frac{1 - \cos{\sqrt{x^{2}+y^{2}+z^{2}}}}{x^{2}+y^{2}+z^{2}} 
            &= \lim\limits \frac{2
            \sin^{2}{\Bigl(\frac{\sqrt{(x^{2}+y^{2}+z^{2})}}{2}\Bigr)}}{x^{2}+y^{2}+z^{2}}
            = \lim\limits \frac{{
              \sin^{2}{\Bigl(\frac{\sqrt{(x^{2}+y^{2}+z^{2})}}{2}\Bigr)}}}{\frac{x^{2}+y^{2}+z^{2}}{2}
            \cdot \frac{1}{2}} \cdot \frac{1}{2} = \frac{1}{2} \cdot
            \lim\limits \frac{{
            \sin^{2}{\Bigl(\frac{\sqrt{(x^{2}+y^{2}+z^{2})}}{2}\Bigr)}}}{\Bigl(\frac{\sqrt{x^{2}+y^{2}+z^{2}}}{2}\Bigr)^{2}}
            \\
            &= \frac{1}{2} \cdot 
            \lim\limits \Biggl(\frac{{
            \sin{\Bigl(\frac{\sqrt{(x^{2}+y^{2}+z^{2})}}{2}\Bigr)}}}{\frac{\sqrt{x^{2}+y^{2}+z^{2}}}{2}}\Biggr)^{2}
            = \frac{1}{2} \cdot 1^{2} = \frac{1}{2}
        \end{align*} 
        Επομένως, αρκεί να ορίσουμε $ f(0,0,0) = \frac{1}{2} $, ώστε να είναι η $f$
        συνεχής.
      \end{solution}
  \end{myitemize}

\item Να δείξετε ότι η συνάρτηση $ f(x,y) = 
  \begin{cases}
    \frac{1 - \cos{\sqrt{xy}}}{y}, & y \neq 0 \\
    \frac{x}{2}, & y = 0
  \end{cases}$
  είναι συνεχής σε κάθε σημείο του άξονα $x$.
  \begin{solution}
  \item {}
    \begin{myitemize}
      \item Για $ y= 0 $, εξετάζουμε τη συνέχεια της $f$ με τον ορισμό.
        Έστω $ (x_{0},0) $ ένα τυχαίο σημείο του άξονα $x$. 
        Έχουμε, $ f(x_{0}, 0) =
        \frac{x_{0}}{2} $
        \begin{align*}
          \lim\limits_{\substack{x\to x_{0} \\y \to
          0}} f(x,y) = \lim\limits_{\substack{x\to x_{0} \\y \to 0}} 
          \frac{1 - \cos{\sqrt{xy}}}{y} 
                &= \lim\limits_{\substack{x\to x_{0} \\y \to 0}} 
                \frac{2 \sin^{2}{\left(\frac{\sqrt{xy}}{2}\right)}}{y} 
                = \lim\limits_{\substack{x\to x_{0} \\y \to 0}}
                \biggl(\frac{\sin^{2}{\frac{\sqrt{xy}}{2}}}{\frac{y}{2} \frac{x}{2}} 
                \cdot \frac{x}{2}\biggr) 
                = \lim\limits_{\substack{x\to x_{0} \\y \to 0}} 
                \frac{\sin^{2}{\frac{\sqrt{xy}}{2}}}{\Bigl(\frac{\sqrt{xy}}{2}\Bigr)^{2}} 
                \cdot \lim_{x \to x_{0}} \frac{x}{2} \\ 
                &= \lim\limits_{\substack{x\to x_{0} \\y \to 0}} \biggl(
                \frac{\sin{\frac{\sqrt{xy}}{2}}}{\frac{\sqrt{xy}}{2}}\biggr)^{2}
                \cdot \lim_{x \to x_{0}} \frac{x}{2} = 1^{2} \cdot \frac{x_{0}}{2}
                = \frac{x_{0}}{2}
        \end{align*} 
        Επομένως η $ f $ είναι συνεχής στο τυχαίο σημείο 
        $ (x_{0},0) $ του άξονα $x$, και άρα είναι συνεχής σε κάθε σημείο του 
        άξονα $x$.
    \end{myitemize}
  \end{solution}


\item Να εξεταστούν πλήρως ως προς τη συνέχεια οι συναρτήσεις:

  \begin{enumerate}[(i)]
    \item $ f(x,y) = 
      \begin{cases}
        \frac{ xy }{ x^{2} + y^{2} } ,& (x,y)\neq (0,0) \\
        0 ,& (x,y)= (0,0)
      \end{cases} $
      \begin{solution}
      \item {}
        \begin{myitemize}
          \item Αν $ (x_{0}, y_{0}) \neq (0,0) $ τότε $ f(x,y) = 
            \frac{ xy }{ x^{2} + y^{2} } $, η οποία είναι συνεχής ως ρητή.
          \item Αν $ (x_{0}, y_{0}) = (0,0) $ τότε θα εξετάσουμε τη συνέχεια με τον
            ορισμό. 
            Γι᾽ αυτό θα βρούμε το όριο της $ f(x,y) $ όταν $ (x,y) \to (0,0) $. Έχουμε
            \begin{align*}
              \lim\limits_{\substack{x\to 0 \\y \to 0}} \frac{xy}{x^{2}+y^{2}}
              \overset{y= \lambda x}{=} \lim_{x \to 0} \frac{x \lambda x}{x^{2}+ \lambda
              ^{2}x^{2}} = \lim_{x \to 0} 
              \frac{\lambda \cancel{x^{2}}}{\cancel{x^{2}}(1+ \lambda ^{2})} =
              \lim_{x \to 0} \frac{\lambda}{1 + \lambda ^{2}} = \frac{\lambda}{1 +
              \lambda^{2} }
            \end{align*}
        \end{myitemize}
        Άρα δεν υπάρχει το όριο της $f$ στο σημείο $ (0,0) $, γιατί εξαρτάται από το 
        $\lambda$, και άρα εξαρτάται από τη διαδρομή προσέγγισης. Επομένως η $f$ δεν 
        είναι συνεχής στο $ (0,0) $.
        
        Άρα η $f$ είναι συνεχής $ \forall (x,y) \in \mathbb{R}^{2} - \{ (0,0) \} $.
      \end{solution}

      \item $ f(x,y) = 
        \begin{cases} 
          \frac{y^{2}}{x^{2}+y^{2}}, & (x,y) 
          \neq (0,0) \\ 0, & (x,y) = (0,0)
        \end{cases} $
        \begin{solution}
        \item {}
          \begin{myitemize}
            \item Αν $ (x_{0}, y_{0}) \neq (0,0) $ τότε $ f(x,y) = 
              \frac{ y^{2} }{ x^{2} + y^{2} } $, η οποία είναι συνεχής ως ρητή.
            \item Αν $ (x_{0}, y_{0}) = (0,0) $ τότε θα εξετάσουμε τη συνέχεια με τον
              ορισμό. 
            Γι᾽ αυτό θα βρούμε το όριο της $ f(x,y) $ όταν $ (x,y) \to (0,0) $. Έχουμε
              \begin{align*}
                \lim\limits_{\substack{x\to 0 \\y \to 0}} \frac{y^{2}}{x^{2}+y^{2}}
                \overset{y= \lambda x}{=} \lim_{x \to 0} 
                \frac{\lambda^{2}x^{2}}{x^{2}+ \lambda
                ^{2}x^{2}} = \lim_{x \to 0} \frac{\lambda^{2}
              \cancel{x^{2}}}{\cancel{x^{2}}(1+ \lambda ^{2})} =
                \lim_{x \to 0} \frac{\lambda^{2}}{1 + \lambda ^{2}} =
                \frac{\lambda^{2}}{1 +
                \lambda ^{2}}
              \end{align*}
        Άρα δεν υπάρχει το όριο της $f$ στο σημείο $ (0,0) $, γιατί εξαρτάται από το 
        $\lambda$, και άρα εξαρτάται από τη διαδρομή προσέγγισης. Επομένως η $f$ δεν 
        είναι συνεχής στο $ (0,0) $.
        Άρα η $f$ είναι συνεχής $ \forall (x,y) \in \mathbb{R}^{2} - \{ (0,0) \} $.
          \end{myitemize}
        \end{solution}

      \item $ f(x,y)= 
        \begin{cases}
          xy^{2}\cos \frac{1}{x^{2}+y^{2}} &, (x,y)\neq (0,0)\\
          0 &, (x,y)=(0,0)
        \end{cases} $
        \begin{solution}
        \item {}
          \begin{myitemize}
            \item Αν $ (x_{0}, y_{0}) \neq (0,0) $ τότε $ f(x,y) = 
              xy^{2}\cos \frac{1}{x^{2}+y^{2}} $, 
              η οποία είναι συνεχής ως πράξεις συνεχών.
            \item Αν $ (x_{0}, y_{0}) = (0,0) $ τότε θα εξετάσουμε τη συνέχεια με τον
              ορισμό. 
              Γι᾽ αυτό θα βρούμε το όριο της $ f(x,y) $ όταν $ (x,y) \to (0,0) $. Έχουμε
              \begin{align*}
                \abs{xy^{2}\cos \frac{1}{x^{2}+y^{2}}}  = \abs{xy^{2}} \cdot 
                \abs{\cos{\frac{1}{x^{2}+y^{2}}} } \leq \abs{xy^{2}} \cdot 1 = 
                \abs{xy^{2}}
              \end{align*}
              όπου $ \lim\limits_{\substack{x\to 0 \\y \to 0}} xy^{2} = 
              0 \cdot 0^{2} = 0 $, 
              και άρα από γνωστό θεώρημα, έχουμε 
              \[
                \lim\limits_{\substack{x\to 0 \\y \to 0}} xy^{2} 
                \cos{\frac{1}{x^{2}+y^{2}}} = 0 
              \] 
              Επίσης, έχουμε $ f(0,0) = 0 $. Άρα 
              \[
                \lim\limits_{\substack{x\to 0 \\y \to 0}} f(x,y) = 
                \lim\limits_{\substack{x\to 0 \\y \to 0}} xy^{2} 
                \cos{\frac{1}{x^{2}+y^{2}}} = 0 = f(0,0)
              \] 
              Δηλαδή η $f$ είναι συνεχής στο σημείο $ (0,0) $.
              Επομένως η $f$ είναι συνεχής $ \forall (x,y) \in \mathbb{R}^{2} $.
          \end{myitemize}
        \end{solution}




  \end{enumerate}

  \end{enumerate}







  \end{document}
