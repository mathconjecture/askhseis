\documentclass[a4paper,table]{report}
\input{preamble.tex}
\input{definitions_ask.tex}


\pagestyle{askhseis}


\begin{document}

\begin{center}
  \minibox{\large\bfseries \textcolor{Col1}{Ασκήσεις στις Ιδιοτιμές-Ιδιοδιανύσματα
  Πίνακα}}
\end{center}

\vspace{\baselineskip}

\begin{enumerate}

\item Να βρεθούν οι ιδιοτιμές και τα ιδιοδιανύσματα των παρακάτω πινάκων:

\begin{enumerate}[i)]

\item $\begin{pmatrix}
3 & 2 \\
2 & 3
\end{pmatrix}$\hfill Απ: \begin{tabular}{l}
$\lambda_1=1$, $\lambda_2=5$ \\
$X_1=(-1,1)^T$ \\
$X_2=(1,1)^T$
\end{tabular}

\item $\begin{pmatrix}
2 & -3 \\
3 & \phantom{-}2
\end{pmatrix}$\hfill Απ: \begin{tabular}{l}
$\lambda_{1,2}=2\pm 3i$ \\
$X_1=(i,1)^T$ \\
$X_2=(-i,1)^T$
\end{tabular}

\item $\begin{pmatrix}
\phantom{-}13 & -3 & \phantom{-}5 \\
\phantom{-}0 & \phantom{-}4 & \phantom{-}0 \\
-15 & \phantom{-}9 & -7
\end{pmatrix}$ \hfill Απ: \begin{tabular}{l}
$\lambda_1=4$, $\lambda_2=8$, $\lambda_3=-2$ \\ 
$X_1=(1,-2,-3)^T$ \\
$X_2=(1,0,-1)^T$ \\
$X_3=(1,0,-3)^T$
\end{tabular}

\end{enumerate}

\item Να βρεθούν οι ιδιοτιμές και τα ιδιοδιανύσματα των παρακάτω πινάκων 
  και να εξετάσετε αν οι πίνακες διαγωνιοποιούνται:

\begin{enumerate}[i)]

\item $\begin{pmatrix}
2 & 1 \\
0 & 2
\end{pmatrix}$\hfill Απ: \begin{tabular}{l}
$\lambda_{1,2}=2 \; (\text{διπλή})$ \\
$X_1=(1,0)^T$ \\
\end{tabular}

\item $\begin{pmatrix}
-2 & 0 & \phantom{-}0 \\
-2 & 2 & \phantom{-}2 \\
\phantom{-}0 & 0 & -2
\end{pmatrix}$ \hfill Απ: \begin{tabular}{l}
$\lambda_1=2$, $\lambda_{2,3}=-2 \; (\text{διπλή})$ \\ 
$X_1=(0,1,0)^T$ \\
$X_2=(2,1,0)^T$ \\
$X_3=(1,0,1)^T$
\end{tabular}



\item $\begin{pmatrix}
\phantom{-}0 & \phantom{-}1 & -2 \\
-2 & \phantom{-}2 & -2 \\
\phantom{-}1 & -1 & \phantom{-}3
\end{pmatrix}$ \hfill Απ: \begin{tabular}{l}
$\lambda_1=1$, $\lambda_{2,3}=2 \; (\text{διπλή})$ \\ 
$X_1=(0,2,1)^T$ \\
$X_2=(-1,0,1)^T$ \\
\end{tabular}

\item $ \begin{pmatrix*}[r]
		-3 & -2 & -4 \\
		0 & -1 & 0 \\
		1 & 1 & 1 
\end{pmatrix*}$ \hfill Απ: \begin{tabular}{l}
$ \lambda_{1} = \lambda_{2} = \lambda_{3} = -1 $ \\
$ X_{1} = (-1,1,0)^T$ \\
$ X_{2} = (-2,0,1)^{T} $
\end{tabular}

\item $ \begin{pmatrix*}[r]
		-4 & 0  & 1 \\
		4 & -3 & -3 \\
		-1 & 0 & -2 
\end{pmatrix*}$ \hfill Απ: \begin{tabular}{l}
$ \lambda_{1} = \lambda_{2} = \lambda_{3} = -3 $ \\
$ X_{1} = (0,1,0)^{T} $
\end{tabular}

\end{enumerate}

\item Αν $A=\begin{pmatrix}
1 & 4 \\
2 & 3 
\end{pmatrix}$ τότε να υπολογιστεί ο $A^{10}$.
\hfill Απ: $A^{10}=\frac{1}{3}\begin{pmatrix}
5^{10}+2 & 2\cdot 5^{10}-2 \\
5^{10}-1 & 2\cdot 5^{10}+1
\end{pmatrix}$

\item Για ποιες τιμές του $\lambda$ ο πίνακας $A=\begin{pmatrix}
2 & -1 & 1 \\
0 & \phantom{-}1 & 1 \\
0 & \phantom{-}0 & \lambda
\end{pmatrix}$
διαγωνιοποιείται?

\item Να βρεθεί ο ορθογώνιος πίνακας $Q$ που διαγωνιοποιεί τον πίνακα
	$A = \begin{pmatrix*}
		1 & 2 & 2 \\
		2 & 1 & 2 \\
		2 & 2 & 5 
	\end{pmatrix*}.$
	
	\hfill Απ: $ Q = \begin{pmatrix*}
		1/ \sqrt{6} & -1/ \sqrt{2} & -1/ \sqrt{3}	\\
		1/ \sqrt{6} & 1/ \sqrt{2} & -1\ \sqrt{3} \\
		2/ \sqrt{6} & 0 & 1/ \sqrt{3} 
	\end{pmatrix*} $

\end{enumerate}



\end{document}
