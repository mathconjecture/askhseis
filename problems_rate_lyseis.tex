\input{$HOME/Desktop/preamble/preamble.tex}
\input{$HOME/Desktop/preamble/definitions.tex}



\begin{document}
	
\chapter{Ρυθμός Μεταβολής}

\section{Προβλήματα στο ρυθμό μεταβολής}

\vspace{\baselineskip}

\begin{enumerate}
	\item {\bfseries \boldmath    Φουσκώνουμε με αέρα ένα σφαιρικό μπαλόνι, έτσι
		ώστε ο όγκος του να αυξάνει με ρυθμό $\SI{100}{cm^{3}\per s}$. Πόσο γρήγορα
	αυξάνει η ακτίνα του μπαλονιού όταν η διάμετρος του είναι $\SI{50}{cm}$?}

		\fbox{\bfseries Λύση:}

		\vspace{\baselineskip}

		Έστω $V$ ο όγκος του μπαλονιού και $r$ η ακτίνα του.
		Γνωρίζουμε ότι
		\[
			V=\frac{4}{3}\pi r^{3}
		\]
		Παραγωγίζοντας την παραπάνω σχέση ως προς $t$ έχουμε:
		\[
			\dv{V}{t} = 4\pi r^{2}\dv{r}{t} \Leftrightarrow
			\dv{r}{t} = \frac{1}{4\pi r^{2}}\dv{V}{t}
		\]
		Με αντικατάσταση των τιμών $r=25$ και $\dv{V}{t}=100$ έχουμε
		\[
			\dv{r}{t} = \frac{1}{4\pi (25)^{2}}100 = \frac{1}{25\pi}.
		\]
		Επομένως, η ακτίνα του μπαλονιού αυξάνει με ρυθμό $\frac{1}{25\pi}\si{cm\per s}$.

	\item {\bfseries \boldmath Μια σκάλα μήκους $\SI{10}{m}$ είναι ακουμπησμένη
		σε ένα κατακόρυφο τοίχο. Εάν το κάτω μέρος (βάση) της σκάλας ολισθαίνει
	οριζόντια, απομακρυνόμενη από τον τοίχο με ρυθμό $\SI{1}{m\per s}$, πόσο
γρήγορα η κορυφή της σκάλας πέφτει (ολισθαίνοντας κατακόρυφα στον τοίχο), όταν η
βάση της σκάλας απέχει $\SI{6}{cm}$ από τον τοίχο?}

		\fbox{\bfseries Λύση:}

		\vspace{\baselineskip}

		Έστω $x=x(t)$ η απόσταση της βάσης της σκάλας από τον τοίχο και $y=y(t)$ η απόσταση της κορυφής της σκάλας από το έδαφος.

		Από το Πυθαγόρειο θεώρημα έχουμε ότι
		\begin{align}\label{eq:pyth}
			x^{2}+y^{2}&=10^{2}  \notag \Leftrightarrow \\
			x^{2}+y^{2}&=100
		\end{align}
		Παραγωγίζοντας την σχέση~\eqref{eq:pyth} ως προς το χρόνο $t$ έχουμε
		\[
			2x\dv{x}{t} + 2y\dv{y}{t} = 0
		\]
		Λύνουμε ως προς το ζητούμενο ρυθμό μεταβολής και έχουμε
		\begin{equation}\label{eq:rateofy}
			\dv{y}{t} = -\frac{x}{y}\dv{x}{t}
		\end{equation}

		Όταν $x=6$, τότε από τη σχέση~\eqref{eq:pyth} έχουμε ότι $y=8$. Με αντικατάσταση αυτών των τιμών στη σχέση~\eqref{eq:rateofy} έχουμε
		\[
			\dv{y}{t} = -\frac{6}{8}(1) = -\SI[quotient-mode=fraction]{3/4}{m/s}
		\]

		Το γεγονός ότι ο ρυθμός μεταβολής $\dv*{y}{t}$ είναι αρνητικός σημαίνει
		ότι η απόσταση $y$ της κορυφής της σκάλας από το έδαφος,
		\emph{μειώνεται} με ρυθμό $\sfrac{3}{4}$ $\si{m\per s}$. Με άλλα λόγια η
		κορυφή της σκάλας πέφτει με ρυθμό $\sfrac{3}{4}$ $\si{m\per s}$.


	\item {\bfseries \boldmath Μια δεξαμενή νερού έχει το σχήμα ενός ανάποδου
		κυκλικού κώνου με ακτίνα βάσης $2$ $\si{m}$ και ύψος $4$ $\si{m}.$ Αν
	στη δεξαμενή εισέρχεται ποσότητα νερού με ρυθμό $2$ $\si{m^{3}/min}$, να
υπολογίσετε το ρυθμό με τον οποίο ανέρχεται η στάθμη του νερού στο εσωτερικό της
δεξαμενής, όταν το νερό έχει βάθος $3$ $\si{m}$.}

		\fbox{\bfseries Λύση:}

		\vspace{\baselineskip}

		Ο όγκος του κυκλικού κώνου δίνεται από τη σχέση
		\begin{equation} \label{eq:cylvol}
			V=\frac{1}{3}\pi r^{2} h
\end{equation}

όπου $r$ είναι η ακτίνα της βάσης και $h$ το ύψος του κώνου. Από την ισότητα των τριγώνων έχουμε ότι
\[
	\frac{r}{h} = \frac{2}{4} \Rightarrow r=\frac{h}{2}
\]
και έτσι η σχέση~\eqref{eq:cylvol} γίνεται
\[
	V=\frac{1}{3}\pi\Bigl(\frac{h}{2}\Bigr)^{2}h = \frac{\pi}{12}h^{3}
\]
Παραγωγίζοντας, τώρα αυτή τη σχέση ως προς $t$ έχουμε
\begin{align}\label{eq:volrate}
	\dv{V}{t}&=\frac{\pi}{4}h^{2}\dv{h}{t}  \notag \Leftrightarrow \\
	\dv{h}{t}&=\frac{4}{\pi h^{2}}\dv{V}{t}
\end{align}
και αντικαθιστώντας $h=3$ και $\dv{V}{t}=2$ έχουμε τελικά
\[
	\dv{h}{t}=\frac{4}{\pi (3)^{2}}\cdot 2 = \frac{8}{9\pi}
\]
Επομένως η στάθμη του νερού στο εσωτερικό της δεξαμενής ανεβαίνει με ρυθμό
$\frac{8}{9\pi}$ $\si{m/min}$.



\item  {\bfseries \boldmath Αυτοκίνητο $A$ ταξιδεύει δυτικά με ταχύτητα $50$
	$\si{km\per h}$ και αυτοκίνητο $B$ ταξιδεύει βόρεια με ταχύτητα $60$
$\si{km/h}$. Και τα δύο αυτοκίνητα κινούνται προς την διαστάυρωση των δύο δρόμων. Με τι ρυθμό τα δύο αυτοκίνητα πλησιάζουν το ένα το άλλο, όταν το $A$ βρίσκεται $0.3$ $\si{km}$ και το $B$ $0.4$ $\si{km}$ από τη διασταύρωση?}

	\fbox{\bfseries Λύση:}

	\vspace{\baselineskip}

	Εστω $C$ η διασταύρωση των δύο δρόμων. Έστω επίσης $x=x(t)$ και $y=y(t)$ η απόσταση των $A$ και $B$ αντίστοιχα, από τη διασταύρωση  και $z=z(t)$ η απόσταση των δύο αυτοκινήτων κατά την χρονική στιγμή $t$.
	Από τα δεδομένα του προβλήματος έχουμε ότι $\dv{x}{t}=-50$ \si{km/h} και
	$\dv{y}{t}=-60$ $\si{km/h}$. Οι παράγωγοι είναι αρνητικές γιατί οι αποστάσεις $x$ και $y$ συνεχώς μειώνονται. Ζητούμενο είναι ο ρυθμός $\dv{z}{t}$

	Από το Πυθαγόρειο θεώρημα έχουμε
	\[
		z^{2}=x^{2}+y^{2}
	\]
	και παραγωγίζοντας ως προς $t$
	\begin{align} \label{eq:distrate}
2z\dv{z}{t} &= 2x\dv{x}{t} + 2y\dv{y}{t} \notag \Rightarrow \\
\dv{z}{t} &= \frac{1}{z}\left(x\dv{x}{t}+y\dv{y}{t}\right)
\end{align}
Όταν $x=0.3$ και $y=0.4$ το Πυθαγόρειο θεώρημα δίνει $z=0.5$, οπότε με αντικατάσταση στη σχέση~\eqref{eq:distrate} έχουμε
\[
	\dv{z}{t}=\frac{1}{0.5}[0.3(-50)+0.4(-60)] = \SI{-78}{km/h}
\]

Επομένως τα δύο αυτοκίνητα πλησιάζουν το ένα το άλλο με ρυθμό $\SI{78}{km/h}$.



\item {\bfseries \boldmath Ένας άνδρας περπατά σε ευθύγραμμο μονοπάτι με
	ταχύτητα $\SI{4}{m/s}$. Μια δέσμη \textlatin{leiser} η οποία είναι τοποθετημένη σε απόσταση $\SI{20}{m}$ από το μονοπάτι, τον σημαδεύει συνεχώς. Με ποιο ρυθμό περιστρέφεται η δέσμη \textlatin{leiser} οταν ο άνδρας βρίσκεται σε απόσταση $\SI{15}{m}$ από εκείνο το σημείο στο μονοπάτι το οποίο απέχει ελάχιστη απόσταση από την πηγή \textlatin{leiser}?}

	\fbox{\bfseries Λύση:}

	\vspace{\baselineskip}
	Έστω $O$ το σημείο στο μονοπάτι που απέχει λιγότερο από την πηγή \textlatin{leiser}.
	Έστω $x$ η απόσταση του άνδρα από το σημείο $O$ και $\phi$ η γωνία που σχηματίζει η δέσμη που σημαδεύει τον άνδρα με την κάθετη ευθεία στο μονοπάτι.

	Έχουμε
	\[
		\frac{x}{20}=\tan\phi \Rightarrow x=20\tan\phi
	\]
	Παραγωγίζοντας ως προς $t$ έχουμε
	\begin{align*}
		\dv{x}{t} &= 20\cdot \frac{1}{\cos^{2}\phi}\dv{\phi}{t} \notag \Rightarrow \\
		\dv{\phi}{t}&=\frac{1}{20}\cos^{2}\phi \\
					&=\frac{1}{20}\cos^{2}\phi\cdot 4 \\
					&=\frac{1}{5}\cos^{2}\phi
	\end{align*}
	Άρα
	\begin{equation} \label{eq:anglerate}
		\dv{\phi}{t}=\frac{1}{20}\cos^{2}\phi
\end{equation}
Όταν $x=15$ τότε το μήκος της δέσμης είναι $25$, όποτε $\cos\phi = \frac{4}{5}$ και επομένως
\[
	\dv{\phi}{t} = \frac{1}{5}\Bigl(\frac{4}{5}\Bigr)^{2} = \frac{16}{125} = 0.128
\]
Επομένως η δέσμη περιστρέφεται με ρυθμό $\SI{0.128}{rad/s}$.


	\item {\bfseries \boldmath Πλοίο Α ξεκινάει από ένα λιμάνι στις 12 μ.μ. και κατευθύνεται δυτικά
		με ταχύτητα  $\SI{9}{km/h}$. Πλοίο Β ξεκινάει από το ίδιο λιμάνι στη 1
		μ.μ. και κατευθύνεται νότια με ταχύτητα $\SI{12}{km/h}$. Με τι ρυθμό
	απομακρύνονται μεταξύ τους τα δύο πλοία στις 3 μ.μ.?}

		\hfill $ \dv{s}{t} = \SI{14,7}{km/h} $
\end{enumerate}


	\fbox{\bfseries Λύση:}

	\[
		s^{2} = x^{2} + y^{2}
	\]
	Παραγωγίζοντας ως προς $t$ έχουμε
	\begin{align*}
		2s \dv{s}{t} & = 2x\dv{x}{t} + 2y\dv{y}{t} \Rightarrow \\
		s \dv{s}{t} & = x\dv{x}{t} + y\dv{y}{t} \Rightarrow \\
	\end{align*}
	Οι αποστάσεις που έχουν διανύσει τα δύο πλοία ως τις $3$ μ.μ. είναι
	\begin{align*}
		x &= v\cdot t = 9 \cdot 3 = 27 \\
		y &= v\cdot t = 12 \cdot 2 = 24
	\end{align*}
	Έχουμε ότι $ s = \sqrt{x^{2} + y^{2}} = \sqrt{27^{2} + 24^{2}} =
	\sqrt{729 + 576} = \sqrt{1305} = 36,12$.
	Άρα 
	\begin{align}
		36,12 \cdot \dv{s}{t} &= 27\cdot 9 + 24 \cdot 12 \\
		\dv{s}{t} &= \frac{243 + 288}{36,12} = \frac{531}{34,12} = 14,70
	\end{align}
	Επομένως τα δύο πλοία απομακρύνονται με ρυθμό $\SI{14,70}{km/h}$.





   \end{document}
