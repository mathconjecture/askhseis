\input{$HOME/Desktop/preamble/preamble.tex}
\input{$HOME/Desktop/preamble/definitions.tex}


\thispagestyle{empty}

\begin{document}

\begin{center}
    \fbox{\large\bfseries Ασκήσεις στα Διανύσματα}
\end{center}

\vspace{\baselineskip}



\begin{enumerate}
    \item Να εξετάσετε αν τα διανύσματα είναι γραμμικώς ανεξάρτητα.
        \begin{enumerate}[i)]
            \item $ \mathbf{u} = (1,-2) $, $ \mathbf{v} = (2,1) $ \hfill Απ: ναι 
            \item $ \mathbf{u} = (0,-1) $, $ \mathbf{v} = (-1,1) $ \hfill Απ: ναι 
            \item $ \mathbf{u} = (2,4) $, $ \mathbf{v} = (1,2) $ \hfill Απ: οχι 
            \item $ \mathbf{u} = (3,-1) $, $ \mathbf{v} = (-6,2) $ \hfill Απ: οχι 
            \item $ \mathbf{u} = (1,2,5) $, $ \mathbf{v} = (2,5,5) $, $ \mathbf{w} =
                (5,1,2) $ \hfill Απ: ναι 
            \item $ \mathbf{u} = (1,1,0) $, $ \mathbf{v} = (1,3,2) $, $ \mathbf{w} =
                (4,9,5) $ \hfill Απ: οχι 
            \item $ \mathbf{u} = (1,2,4) $, $ \mathbf{v} = (1,3,5) $, $ \mathbf{w} =
                (2,1,5) $ \hfill Απ: οχι 
            \item $ \mathbf{u} = (1,3,-2) $, $ \mathbf{v} = (2,1,0) $ \hfill Απ: ναι 
        \end{enumerate}

    \item Να υπολογισετε το εσωτερικό γινόμενο των παρακάτω διανυσμάτων.
        \begin{enumerate}[i)]
            \item $ \mathbf{u} = (1,2) $, $ \mathbf{v} = (-2,4) $ \hfill Απ: 6 
            \item $ \mathbf{u} = (0,-3) $, $ \mathbf{v} = (7,1) $ \hfill Απ: -3 
            \item $ \mathbf{u} = (1,-3,2) $, $ \mathbf{v} = (2,0,5) $ \hfill Απ: 12 
            \item $ \mathbf{u} = (3,1,-4) $, $ \mathbf{v} = (0,2,1) $ \hfill Απ: -2 
            \item $ \mathbf{u} = (1,1,1) $, $ \mathbf{v} = (2,0,4) $ \hfill Απ: 6 
            \item $ \mathbf{u} = (1,-2,0,4) $, $ \mathbf{v} = (0,1,3,0) $ \hfill Απ: -2 
            \item $ \mathbf{u} = (0,1,1,-3) $, $ \mathbf{v} = (5,1,-2,3) $ \hfill Απ: -8 
        \end{enumerate}

    \item Αν $ \mathbf{u} = (-1,3) $ και $ \mathbf{v} = (2,5) $, τότε να υπολογίσετε τα παρακάτω
        εσωτερικά γινόμενα.
        \begin{enumerate}[i)]
            \item $ \mathbf{u} \cdot \mathbf{v} $ \hfill Απ: 13 
            \item $ (2 \mathbf{u}) \cdot (-3 \mathbf{v}) $ \hfill Απ: -78 
            \item $ (\mathbf{u} - \mathbf{v}) \cdot (3 \mathbf{u}+ \mathbf{v}) $ \hfill Απ: -25 
        \end{enumerate}

    \item Να υπολογίσετε την γωνία που σχηματίζουν τα παρακάτω διανύσματα.
        \begin{enumerate}[i)]
            \item $ \mathbf{u} = \left(1,2\right) $, $ \mathbf{v} = \left(3,1\right) $ \hfill Απ: $
                        \SI{45}{\degree} $ 
            \item $ \mathbf{u} = \left(\sqrt{3}, 1\right) $, $ \mathbf{v} = \left(- \sqrt{3} , 
                1\right) $ \hfill Απ: $ \SI{120}{\degree}$ 
        \end{enumerate}

    \item Έστω δυο διανύσματα $ \mathbf{u} $ και $ \mathbf{v} $ που έχουμε μέτρα $ \norm{\mathbf{u}}
        = \sqrt{3} $ και $ \norm{\mathbf{v}} = 1 $ και σχηματίζουν γωνία $ \theta =
        \frac{\pi}{6} $. Να υπολογιστεί το εσωτερικό τους γινόμενο. 

        \hfill Απ: $ \frac{3}{2} $ 

    \item Να υπολογισετε την τιμή της μεταβλητής $ \lambda $ ωστε τα παρακάτω διανύσματα να είναι 
        ορθογώνια.
        \begin{enumerate}[i)]
            \item $ \mathbf{u} = (1, \lambda, -1) $, $ \mathbf{v} = (2,-1,2) $ \hfill 
                Απ: $ \lambda = 0 $
            \item $ \mathbf{u} = (- \lambda, 2 \lambda, 3) $, $ \mathbf{v} = (1,-4,-1) $ \hfill Απ:
                $ \lambda = 3 $ 
            \item $ \mathbf{u} = (1, \lambda, 2,-1) $, $ \mathbf{v} = (-1,1,0,3) $ \hfill Απ: $
                \lambda = 4 $ 
        \end{enumerate}
\end{enumerate}


\end{document}

