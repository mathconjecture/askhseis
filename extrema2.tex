\input{preamble.tex}
\input{definitions_ask.tex}

\pagestyle{askhseis}

\renewcommand{\vec}{\mathbf}

\begin{document}

\begin{center}
  \minibox{\large \bfseries \textcolor{Col1}{Ασκήσεις στα Ακρότατα}}
\end{center}

\vspace{\baselineskip}

\section*{Τοπικά Ακρότατα}

\begin{enumerate}
  \item Να βρεθούν και να χαρακτηριστούν τα κρίσιμα σημεία  των παρακάτω συναρτήσεων:
    \begin{enumerate}[i)]
      \item $ f(x,y) = x^{3} + y^{3} + 3xy $ 
        \hfill Απ: max: $(-1,-1)  $, σάγμα: $ (0,0) $
      \item $ f(x,y) = x^{2}+y^{4} $ 
        \hfill Απ: min: $ (0,0) $ 
      \item $ f(x,y) = x^{3} + y^{3} - 3x -12y + 50 $ 
        \hfill Απ: max: $ (-1,-2)$, min: $ (1,2) $, 
        σάγμα: $ (1,-2), (-1,2) $
      \item $ f(x,y) = x^{3} + y^{3} -3x -3y + 1 $ 
        \hfill Απ: max: $(-1,-1)  $, min: $ (1,1) $,
        σάγμα: $ (1,-1), (-1,1) $
      \item $ f(x,y) = x^{3} + 4xy -4y^{2} $ 
        \hfill Απ: max: $ (-2/3, -1/3)  $, σάγμα: $ (0,0) $
      % \item $ f(x,y) = x^{4} + y^{4} -2(x-y)^{2}$  
      %   \hfill Απ: min: $ (\sqrt{2} , -\sqrt{2}), (-\sqrt{2} , \sqrt{2}) $, 
      %   σάγμα: $ (0,0) $
      \item $ f(x,y) = \mathrm{e}^{2x} - 2x + 2y^{2} +3 $ \hfill Απ: max: $ (0,0) $  
      \item $ f(x,y) = (x^{2}-3y^{2})e^{1-x^{2}-y^{2}} $ 
        \hfill Απ: max: $ (1,0), (-1,0) $, min: $ (0,1), (0,-1) $, 
        σάγμα: $ (0,0) $
      \item $ f(x,y,z) = 2-x^{2}+2xy-3y^{2}-2z^{2} $ \hfill Απ:  max $ (0,0,0) $ 
      \item $ f(x,y,z) = x^{2}+y^{2}+z^{2}-2x-1 $ \hfill Απ:  min: $ (1,0,0) $ 
      \item $ f(x,y,z) = 2x^{2}+xy+4y^{2}+xz+z^{2}+2 $ \hfill Απ: min: $ (0,0,0) $ 
    \end{enumerate}

  \item Να βρεθούν και να χαρακτηριστούν τα κρίσιμα σημεία της συνάρτησης 
    $ f(x,y,z) = x^{3} + y^{3}+z^{3} + 3xy +3yz + 3xz $

    \hfill Απ: σάγμα: $(0,0,0)$, max: $(-2,-2,-2)$  
\end{enumerate}


\section*{Ακρότατα Υπό Συνθήκη}

\begin{enumerate}

  \item 
    \begin{enumerate}[i)]
      \item $ z=xy $ με περιορισμό $ x+2y=2 $ \hfill Απ: max: $ (1,1/2) $ 
      \item $ z=7-y+x^{2} $ με περιορισμό $ x+y=0 $ \hfill Απ: min: $ (-1/2,1/2) $ 
      \item $ z=x-3y-xy $ με περιορισμό $ x+y=6 $ \hfill Απ: min: $ (1,5) $ 
    \end{enumerate}

  \item \textbf{(θέμα 2021)} 
    Να υπολογιστούν τα τοπικά ακρότατα της συνάρτησης $ f(x,y,z) = x^{2}+y^{2}+z^{2}
    $ που ικανοποιούν τον περιορισμό $ x+y+z+1=0 $.
    \hfill Απ: min: $ (-1/3,-1/3,-1/3) $ 

  \item Να υπολογιστούν τα τοπικά ακρότατα της συνάρτησης 
    $ f(x,y,z) = x^{2}+y^{2}+z^{2}-2x-2y-z+ \frac{5}{4} $ που ικανοποιούν τον 
    περιορισμό $ x^{2}+y^{2}-z=0  $.
    \hfill Απ: min: $ (1/ \sqrt[3]{4} , 1/ \sqrt[3]{4}) $ 

  \item Να υπολογιστούν τα τοπικά ακρότατα της συνάρτησης 
    $f(x,y,z)=xyz$ που ικανοποιούν την εξίσωση $x+y+z-1=0$.  
    \hfill Απ: max: $ (1/3,1/3,1/3) $ 

    %Thomas 12th 14.8 ex.1 
  \item Να βρείτε τα ακρότατα της συνάρτησης $ f(x,y) = xy $ πάνω στην έλλειψη 
    $ x^{2}+2y^{2}=1 $.

    \hfill Απ: 
    \begin{tabular}{l}
      max: $ f(\sqrt{2} /2, 1/2) = f(- \sqrt{2} /2, -1/2) = \frac{\sqrt{2}}{2} $ \\
      min $ f(\sqrt{2} /2, -1/2) = f(- \sqrt{2} /2, 1/2) = -\frac{\sqrt{2}}{2} $ \\
    \end{tabular}

    %Thomas 12th 14.8 ex.14 
  \item Να βρείτε τα ακρότατα της συνάρτησης $ f(x,y) = 3x-y+6 $ υπό τον περιορισμό 
    $ x^{2}+y^{2}=4 $.

    \hfill Απ:  
    \begin{tabular}{l}
      max: $ f(\frac{6}{\sqrt{10}} , - \frac{2}{\sqrt{10}}) = 2 \sqrt{10} +6 $ \\
      min $ f(-\frac{6}{\sqrt{10}} , + \frac{2}{\sqrt{10}}) = -2 \sqrt{10} +6 $ \\
    \end{tabular}
\end{enumerate}





\end{document}

