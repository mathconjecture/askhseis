\documentclass[a4paper,12pt]{article}


\usepackage[english,greek]{babel}
\usepackage[utf8]{inputenc}

\usepackage{amsmath}
\usepackage{amssymb}
\usepackage{amsfonts}
\usepackage[top=2cm,bottom=2cm,left=2cm,right=2cm]{geometry}
\usepackage{graphicx}

\begin{document}
\thispagestyle{empty}
\begin{center}
\fbox{\Large\bfseries{Αόριστα Ολοκληρώματα}}
\end{center}

\vspace{2\baselineskip}

{\Large \bfseries Ασκήσεις}
\begin{enumerate}
\item Να υπολογιστούν τα παρακάτω αόριστα ολοκληρώματα (με ριζικά).
\begin{enumerate}
\item $\int\frac{\sqrt{x+4}}{x}dx$\hfill Απ: $2\sqrt{x+4}+2\ln\left|\frac{\sqrt{x+4}-2}{\sqrt{x+4}+2}\right|+c$
\item $\int\frac{1}{(x-1)^2}\sqrt{\frac{1-x}{1+x}}dx$\hfill Απ: $\frac{1}{\sqrt{\frac{1-x}{1+x}}}+c$.
\item $\int\frac{1}{\sqrt{x^2+3x+2}}dx$\hfill Απ: $\ln\left|\frac{t+1}{t-1}\right|+c$, όπου $t=\frac{\sqrt{x^2+3x+2}}{x+1}$
\item $\int\sqrt[3]{\frac{x-1}{x}}\frac{1}{x(x-1)}dx$\hfill Απ: $3^3\sqrt[3]{\frac{x-1}{x}} +c$
\item $\int\frac{1}{\sqrt{2x-1}-\sqrt[4]{2x-1}}dx$\hfill Απ: $(t+1)^2 +2\ln|t-1|+c$, όπου $t=\sqrt[4]{2x-1}$
\item $\int\frac{1}{x\sqrt{x^2+x+1}}dx$\hfill Απ: $\ln\left|\frac{t+1}{t-1}+c\right|$, όπου $t=x-\sqrt{x^2+x+1}$
\end{enumerate}

\item Να υπολογιστούν τα παρακάτω αόριστα ολοκληρώματα (διωνυμικά).
\begin{enumerate}
\item $\int x^{-\frac{1}{2}}(x^{\frac{1}{4}}+1)^{-10}dx$\hfill Απ: $-\frac{1}{2}\frac{1}{(t+1)^8}+\frac{4}{9}\frac{1}{(x+1)^9}+c$, όπου $t=\sqrt[4]{x}$
\item $\frac{x}{\sqrt{4-x^2}^3}dx$\hfill Απ: $\frac{8-x^2}{\sqrt{4-x^2}}+c$
\item $\int x^{-3}(2-x^3)^{-\frac{1}{3}}$\hfill Απ: $-\frac{\sqrt[3]{(2-x^3)^2}}{4x^2}+c$
\item $\int\frac{\sqrt[3]{1+\sqrt[4]{x}}}{\sqrt{x}}dx$\hfill Απ: $\frac{12}{7}t^7-3t^4+c$, όπου $t=\sqrt[3]{1+x^{\frac{1}{4}}}$
\item $\int \frac{\sqrt{1+\sqrt[3]{x}}}{\sqrt[3]{x^2}}dx$\hfill Απ: $2t^3+c$, όπου $t=(1+x^{\frac{1}{3}})^{\frac{1}{2}}$
\item $\sqrt{x}(1+\sqrt[3]{x})^4dx$\hfill Απ: $6(\frac{t^9}{9}+\frac{4t^{11}}{11}+\frac{6t^{13}}{13}+\frac{4t^{15}}{15}+\frac{t^{17}}{17})+c$, όπου $t=\sqrt[6]{x}$
\end{enumerate}

\item Να υπολογιστούν τα παρακάτω αόριστα ολοκληρώματα (τριγωνομετρικά)
\begin{enumerate}
\item $\int\sin^3x\cos^2xdx$\hfill Απ: $-\frac{\cos^3x}{3}+\frac{\cos^5x}{5}+c$
\item $\int\sin^4x\cos^2xdx$\hfill Απ: $\frac{1}{16}x-\frac{1}{64}\sin 4x-\frac{1}{48}\sin^32x+c$
\item $\int\frac{\cos^2x}{\sin^6x}dx$\hfill Απ: $-\frac{1}{5\tan^5x}-\frac{1}{\tan^3x}+c$
\item $\int\frac{1}{\sin x}dx$\hfill Απ: $\ln|\tan\frac{x}{2}|+c$
\item $\int\frac{1}{\cos x}dx$\hfill Απ: $\ln|\frac{\tan\frac{x}{2}+1}{\tan\frac{x}{2}-1}|+c$
\item $\frac{\cos x}{1+\cos x}dx$\hfill Απ: $x-\tan\frac{x}{2}+c$
\end{enumerate}
\end{enumerate}

{\Large\bfseries Παρατηρήσεις:}
\begin{enumerate}
\item Για την $2$η άσκηση, όσα ολοκληρώματα περιέχουν ριζικά πρέπει να γραφούν στην κανονική μορφή των διωνύμων. Μετά συνεχίζουμε κατά τα γνωστά.

\item Για την $3$η άσκηση, με τα τριγωνομετρικά ολοκληρώματα, στην $1$ και στην $2$ θέτουμε όπου $t=\tan\frac{x}{2}\Rightarrow x=2\arctan t \Rightarrow dx=2\frac{1}{t^2+1}dt$, όπότε ελπίζω να σου έδειξα ότι τότε τα $\sin x$ και $\cos x$ γράφονται ως συναρτήσεις του $t$ ως εξής: $\sin x=\frac{2t}{1+t^2}$ και $\cos x =\frac{1-t^2}{1+t^2}$
\item Η $3$ είναι άρτια ως προς $\sin x$ και $\cos x$ και άρα θέτω $t=\tan x\Rightarrow dx=\frac{1}{t^2+1}dt$. Σε αυτήν χρησιμοποιώ τις ταυτότητες $\cos^2x=\frac{1}{1+\tan^2x}=\frac{1}{1+t^2}$ και $\sin^2x=1-cos^2x=1-\frac{1}{1+t^2}=\frac{t^2}{1+t^2}$
\item Η $4$ είναι περιττή ως προς $\sin x$
\item Για την $5$ είναι καλύτερα να χρησιμοποιήσεις τους τύπους αποτετραγωνισμού $\sin^2x=\frac{1-\cos 2x}{2}$ και $\cos^2x=\frac{1+\cos 2x}{2}$ και να αντικαταστήσεις με την διαφορά τετραγώνων που εμφανίζεται. 
\item Για την $5$ επειδή είναι και αυτή η γενική περίπτωση όπου η συνάρτηση είναι πηλίκο τριγωνομετρικών, θέτουμε $t=\tan\frac{x}{2}$. 
\end{enumerate}

\end{document}