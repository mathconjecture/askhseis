\input{preamble_ask.tex}
\input{definitions_ask.tex}


\pagestyle{askhseis}
\everymath{\displaystyle}


\begin{document}

\begin{center}
  \minibox{\large\bfseries \textcolor{Col1}{Αόριστα Ολοκληρώματα}}
\end{center}

\vspace{\baselineskip}

\begin{enumerate}
\item Να υπολογιστούν τα παρακάτω αόριστα ολοκληρώματα (με ριζικά).
  \begin{enumerate}[i)]
\item $\int\frac{\sqrt{x+4}}{x} \; \; dx$
  \hfill Απ: $2\sqrt{x+4}+2\ln\left|\frac{\sqrt{x+4}-2}{\sqrt{x+4}+2}\right|$
\item $\int\frac{1}{(x-1)^2}\sqrt{\frac{1-x}{1+x}}\; dx$
  \hfill Απ: $\frac{1}{\sqrt{\frac{1-x}{1+x}}}$
\item $\int\frac{1}{\sqrt{x^2+3x+2}}\; dx$
  \hfill Απ: $\ln\left|\frac{t+1}{t-1}\right|+c$, \; $t=\frac{\sqrt{x^2+3x+2}}{x+1}$
\item $\int\sqrt[3]{\frac{x-1}{x}}\frac{1}{x(x-1)}\; dx$
  \hfill Απ: $3\sqrt[3]{\frac{x-1}{x}} $
\item $\int\frac{1}{\sqrt{2x-1}-\sqrt[4]{2x-1}}\; dx$
  \hfill Απ: $(t+1)^2 +2\ln|t-1|$, \; $t=\sqrt[4]{2x-1}$
\item $\int\frac{1}{x\sqrt{x^2+x+1}}\; dx$
  \hfill Απ: $\ln\left|\frac{t+1}{t-1}+c\right|$, \; $t=x-\sqrt{x^2+x+1}$
\end{enumerate}

\item Να υπολογιστούν τα παρακάτω αόριστα ολοκληρώματα (διωνυμικά).
  \begin{enumerate}[i)]
\item $\int x^{-\frac{1}{2}}(x^{\frac{1}{4}}+1)^{-10}\; dx$
  \hfill Απ: $-\frac{1}{2}\frac{1}{(t+1)^8}+\frac{4}{9}\frac{1}{(x+1)^9}$, \; 
  $t=\sqrt[4]{x}$
\item $\int\frac{x^{3}}{\sqrt{(4-x^2)^3}}\; dx$\hfill Απ: $\frac{8-x^2}{\sqrt{4-x^2}}$
\item $\int x^{-3}(2-x^3)^{-\frac{1}{3}} \; dx$
  \hfill Απ: $-\frac{\sqrt[3]{(2-x^3)^2}}{4x^2} $
\item $\int\frac{\sqrt[3]{1+\sqrt[4]{x}}}{\sqrt{x}}\; dx$
  \hfill Απ: $\frac{12}{7}t^7-3t^4$, \; $t=\sqrt[3]{1+x^{\frac{1}{4}}}$
\item $\int \frac{\sqrt{1+\sqrt[3]{x}}}{\sqrt[3]{x^2}}\; dx$
  \hfill Απ: $2t^3$, \; $t=(1+x^{\frac{1}{3}})^{\frac{1}{2}}$
\item $\int\sqrt{x}(1+\sqrt[3]{x})^4\; dx$
  \hfill Απ: $6\left(\frac{t^9}{9}+\frac{4t^{11}}{11}+\frac{6t^{13}}{13}+
    \frac{4t^{15}}{15}+ \frac{t^{17}}{17}\right)$, \; $t=\sqrt[6]{x}$
\end{enumerate}

\item Να υπολογιστούν τα παρακάτω αόριστα ολοκληρώματα (τριγωνομετρικά)
  \begin{enumerate}[i)]
\item $\int\sin^3x\cos^2x\; dx$\hfill Απ: $-\frac{\cos^3x}{3}+\frac{\cos^5x}{5}$
\item $\int\sin^4x\cos^2x\; dx$
  \hfill Απ: $\frac{1}{16}x-\frac{1}{64}\sin 4x-\frac{1}{48}\sin^32x$
\item $\int\frac{\cos^2x}{\sin^6x}\; dx$
  \hfill Απ: $-\frac{1}{5\tan^5x}-\frac{1}{\tan^3x}$
\item $\int\frac{1}{\sin x}\; dx$\hfill Απ: $\ln\left|\tan\frac{x}{2}\right|$
\item $\int\frac{1}{\cos x}\; dx$
  \hfill Απ: $\ln\left|\frac{\tan\frac{x}{2}+1}{\tan\frac{x}{2}-1}\right|$
\item $ \int\frac{\cos x}{1+\cos x}\; dx$\hfill Απ: $x-\tan\frac{x}{2}$
\end{enumerate}
\end{enumerate}

\begin{center}
  \minibox{\large\bfseries \textcolor{Col1}{Παρατηρήσεις-Υποδείξεις}}
\end{center}

\vspace{\baselineskip}

\begin{enumerate}
  \item Για τα τριγωνομετρικά ολοκληρώματα, θυμάμαι τη γενική περίπτωση αντικατάστασης: 
    \begin{center}
      Θέτουμε $ \boldsymbol{t = \tan{\frac{x}{2}}} \Rightarrow \frac{x}{2} = \arctan{t} 
      \Rightarrow x = 2 \arctan {t} $ οπότε $ \boldsymbol{dx = \frac{2}{1 + t^{2}} dt} $ 
      και ισχύουν οι τύποι:
    \end{center}
    \[
      \boxed{\sin x=\frac{2t}{1+t^2}} \quad \text{και} \quad \boxed{\cos x
      =\frac{1-t^2}{1+t^2}}
    \] 
\end{enumerate}

\end{document}
