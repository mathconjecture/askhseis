\input{preamble.tex}
\input{definitions.tex}

\usepackage{anyfontsize}
\pagestyle{vangelis}

\begin{document}

\begin{center}
  \minibox[frame]{\Large \bfseries Ασκήσεις στους Διανυσματικούς Χώρους}
\end{center}

\vspace{\baselineskip}

\begin{enumerate}
  \item Να εξετάσετε αν τα παρακάτω υποσύνολα είναι υπόχωροι. 
    \begin{enumerate}[(i)]
      \item $ W = \{(x,y)\in \mathbb{R}^{2} \; \mid \; x,y \geq 0 \} $ 
        \hfill Απ: όχι 
      \item $ W = \{ (x,y) \in \mathbb{R}^{2} \; \mid \; x = 3y \} $ \hfill Απ: ναι
      \item $ W = \{ (x,y) \in \mathbb{R}^{2} \; \mid \; x+y=1 \} $ 
        \hfill Απ: όχι 
      \item $ W = \{ (x,y) \in \mathbb{R}^{2} \; \mid \; y = x^{2} \} $ 
        \hfill Απ: όχι 
      \item $ W = \{ (x,y,z) \in \mathbb{R}^{3} \; \mid \; x\geq 0 \} $ \hfill Απ:  όχι
      \item $ W = \{ (x,y,z) \in \mathbb{R}^{3} \; \mid \; x + y + z = 0 \} $ 
        \hfill Απ: ναι
      \item $ W = \{ (x,y,z) \in \mathbb{R}^{3} \; \mid \; x + y + z = 1 \} $ 
        \hfill Απ: όχι
      \item $ W = \{ (x,y,z) \in \mathbb{R}^{3} \; \mid \; x^{2} + y^{2} + z^{2} 
        \leq 1  \} $ \hfill Απ: όχι
      \item $ W = \{ (x,y,z) \in \mathbb{R}^{3} \; \mid \; x = y = z \} $ \hfill Απ: ναι
        % \item $ W = \{ (x,y,z) \in \mathbb{R}^{3} \mid x \leq y \leq q \} $ 
        % \hfill Απ: όχι
      \item $ W = \{ (x,y,z) \in \mathbb{R}^{3} \; \mid \; x = 2y = 3z \} $ 
        \hfill Απ: ναι 
      \item $ W = \{ A \in M_{2}(\mathbb{R}) \; \mid \; \det{A}=1 \} $ 
        \hfill Απ: όχι 
      \item $ W = \{ A \in M_{2}(\mathbb{R}) \; \mid \; \det{A}=0 \} $ 
        \hfill Απ: όχι 
    \end{enumerate}

  \item Να δείξετε ότι τα παρακάτω υποσύνολα είναι υπόχωροι
    \begin{enumerate}[(i)]
      \item $ W = \{ (x,y) \in \mathbb{R}^{2} \mid y = 3x \} $ 
      \item $ W = \{(x,y,z)\in \mathbb{R}^{3} \; : \; 2x-3y+z=0 \} $
    \end{enumerate}

  \item\label{ask:lineks} Να εξετάσετε αν τα παρακάτω διανύσματα είναι γραμμικώς 
    ανεξάρτητα.
    \begin{enumerate}[(i)]
      \item $ \mathbf{u} = (1,2) $, $ \mathbf{v} = (3,-5) $ \hfill Απ: ναι
      \item $ \mathbf{u} = (1,2,-3) $ $ \mathbf{v} = (4,5,-6) $ \hfill Απ: ναι
      \item $ \mathbf{u} = (1,1,0)$, $ \mathbf{v} = (1,3,2)$, $ \mathbf{w} = (4,9,5) $ 
        \hfill Απ: όχι 
      \item $ \mathbf{u} = (1,2,3)$, $ \mathbf{v} = (2,5,7)$, $ \mathbf{w} = (1,3,5) $ 
        \hfill Απ: ναι 
      \item $ \mathbf{u} = (1,2,-3,1) $, $ \mathbf{v} = (3,7,1,-2) $, $ \mathbf{w} =
        (1,3,7,-4) $ \hfill Απ: όχι
      \item $ \mathbf{u} = (1,3,1,-2) $, $ \mathbf{v} = (2,5,-1,3) $, $ \mathbf{w} =
        (1,3,7,-2) $ \hfill Απ: ναι
    \end{enumerate}

  \item Αν $ \mathbf{u} $, $ \mathbf{v} $, $ \mathbf{w} $ είναι γραμμικώς 
    ανεξάρτητα διανύσματα τότε να δείξετε και ότι τα παρακάτω σύνολα διανυσμάτων 
    είναι επίσης γραμμικώς ανεξάρτητα.
    \begin{enumerate}[(i)]
      \item $ S = \{ \mathbf{u} + \mathbf{v} - 2 \mathbf{w}, \mathbf{u} - \mathbf{v} -
        \mathbf{w}, \mathbf{u} + \mathbf{w} \} $
      \item $ S = \{ \mathbf{u} + \mathbf{v} - 3 \mathbf{w}, \mathbf{u} + 3 \mathbf{v} -
        \mathbf{w}, \mathbf{v} + \mathbf{w}\}  $
    \end{enumerate}

  \item\label{ask:eksart} Να εκφράσετε το διάνυσμα $ \mathbf{b} = (1,-2,5) $ ως γραμμικό 
    συνδυασμό των διανυσμάτων $ \mathbf{u} = (1,1,1)$, $ \mathbf{v} = (1,2,3)$, 
    $ \mathbf{w} = (2,-1,1) $.

    \hfill Απ: $ \mathbf{b} = -6 \mathbf{u} + 3 \mathbf{v} +2 \mathbf{w} $ 	

  \item\label{ask:eksart2} Να εξετάσετε αν το διάνυσμα $ \mathbf{b} = (2,5,-3) $ 
    γράφεται ως γραμμικός συνδυασμός των διανυσμάτων $ \mathbf{u} = (1,-3,2)$, 
    $ \mathbf{v} = (2,4,1)$, $ \mathbf{w} = (1,-5,7) $.

    \hfill Απ: ναι 

  \item Να εξετάσετε αν το διάνυσμα $ \mathbf{b} = (-3,0,12,15) $ μπορεί να γραφεί 
    ως γραμμικός συνδυασμός των διανυσμάτων $ \mathbf{u}_{1} = (1,2,-2,1) $, 
    $ \mathbf{u}_{2} = (1,3,-1,4) $ και $ \mathbf{u}_{3} = (2,1,-7,-7) $. 
    Αν ναι, τότε να βρεθεί ένας τέτοιος γραμμικός συνδυασμός.

    \hfill Απ: $ \mathbf{b} = -9 \mathbf{u}_{1} + 6 \mathbf{u}_{2} $ 

  \item\label{ask:eksart3} Έστω τα διανύσματα $ \mathbf{u}_{1} = (1,2,0) $, 
    $ \mathbf{u}_{2} = (-1,1,2) $ 
    και $ \mathbf{u}_{3} = (3,0,-4) $. Να βρείτε ποια συνθήκη πρέπει να ικανοποιούν τα 
    $a$, $b$, $c$, ώστε το διάνυσμα $ \mathbf{b} = (a,b,c) $ να ανήκει στον υπόχωρο 
    $ \Span \{ \mathbf{u}_{1}, \mathbf{u}_{2}, \mathbf{u}_{3} \} $.

    \hfill Απ: $ 4a -2b + 3c = 0 $ 

  \item\label{ask:synd} Να δείξετε ότι τα διανύσματα $ \mathbf{u}_{1} = (1,3,0,5) $, 
    $ \mathbf{u}_{2} = (1,2,1,4) $, $ \mathbf{u}_{3} = (1,1,2,3)$, 
    $ \mathbf{u}_{4} = (1,1,2,3) $, $ \mathbf{u}_{5} = (1,-3,6,-1) $ 
    είναι γραμμικώς εξαρτημένα και να βρεθεί μια σχέση που τα συνδέει. 

    \hfill Απ: $ \mathbf{u}_{1} - 2 \mathbf{u}_{2} + \mathbf{u}_{3} = 0 $ 

  \item\label{ask:baseeks} Να εξετάσετε αν τα διανύσματα $ \mathbf{u} = (1,1,1)$, 
    $ \mathbf{v} = (1,2,3)$, $ \mathbf{w} = (1,5,3) $ είναι βάση του $\mathbb{R}^{3}$. 

    \hfill Απ: ναι 

  \item\label{ask:parag} Να εξετάσετε αν τα διανύσματα $ \mathbf{u_{1}} = (1,1,3), 
    \mathbf{u_{2}} = (1,0,2), \mathbf{u_{3}}= (1,-2,0)$ παράγουν τον $ \mathbb{R}^{3} $.

    \hfill Απ: όχι 

  \item Να δείξετε ότι τo σύνολο $ B = \{ (-1,1,0), (1,1,0), 
    (0,0,1) \} $ είναι βάση του 
    $ \mathbb{R}^{3} $, και να βρεθούν οι συντεταγμένες του διανύσματος 
    $ \mathbf{v} = (2,3,5) $ ως προς αυτή τη βάση.

    \hfill Απ: $ [\mathbf{v}]_{B} = (-2,0,5) $ 

  \item\label{ask:isoi} Έστω τα διανύσματα $ \mathbf{u}_{1} = (1,2,-1,3)$, 
    $\mathbf{u}_{2} = (2,4,1,-2)$, $ \mathbf{u} _{3} = (3,6,3,-7) $ και 
    $ \mathbf{w}_{1} = (1,2,-4,11)$, $ \mathbf{w}_{2} = (2,4,-5,14) $. 
    Αν $ U = \Span \{ \mathbf{u}_{1}, \mathbf{u}_{2}, \mathbf{u}_{3} \} $ και 
    $ W = \Span \{ \mathbf{w}_{1}, \mathbf{w}_{2} \} $, τότε να δείξετε ότι $ U=W $.

  \item\label{ask:parag2} Για τα παρακάτω διανύσματα να βρεθεί μια βάση και η διάσταση 
    του χώρου που παράγουν.
    \begin{enumerate}[(i)]
      \item $ \mathbf{u}_{1} = (1,1,1,2,3) $, $ \mathbf{u}_{2} = (1,2,-1,-2,1) $, 
        $ \mathbf{u} _{3} = (3,5,-1,-2,5) $, $ \mathbf{u}_{4} = (1,2,1,-1,4) $

        \hfill Απ: $ B = \{ \mathbf{u}_{1}, \mathbf{u}_{2}, \mathbf{u}_{3}\} $ 

      \item $ \mathbf{u}_{1} = (1,-2,1,3,-1) $, $ \mathbf{u}_{2} = (-2,4,-2,-6,2) $, $
        \mathbf{u}_{3} = (1,-3,1,2,1) $, $ \mathbf{u}_{4} = (3,-7,3,8,-1) $

        \hfill Απ: $ B = \{ \mathbf{u}_{1}, \mathbf{u}_{2}, \mathbf{u}_{3} \} $ 

      \item $ \mathbf{u}_{1} = (1,0,1,0,1) $, $ \mathbf{u}_{2} = (1,1,2,1,0) $, 
        $ \mathbf{u} _{3} = (1,2,3,1,1,) $, $ \mathbf{u}_{4} = (1,2,1,1,1) $

        \hfill Απ: $ B = \{ \mathbf{u}_{1}, \mathbf{u}_{2}, \mathbf{u}_{3}, 
        \mathbf{u}_{4} \} $ 
      \item $ \mathbf{u}_{1} = (1,0,1,1,1) $, $ \mathbf{u}_{2} = (2,1,2,0,1) $, 
        $ \mathbf{u} _{3} = (1,1,2,3,4) $, $ \mathbf{u}_{4} = (4,2,5,4,6) $

        \hfill Απ: $ B = \{ \mathbf{u}_{1}, \mathbf{u}_{2}, \mathbf{u}_{3} \} $ 
    \end{enumerate}

  \item Θεωρήστε τους υποχώρους 
    $ U = \{ (x,y,z,w) \in \mathbb{R}^{3} \mid y - 2z + w = 0 \} $ και 
    $ W = \{ (x,y,z,w) \in \mathbb{R}^{3} \mid x = w, y = 2z \} $ του 
    $\mathbb{R}^{4}$. Να βρείτε μια βάση και τη διάσταση των υποχώρων:
    \begin{enumerate}[(i)]
      \item $ U $ \hfill Απ: $ B_{U} = \{ (1,0,0,0), (0,2,1,0), (0,-1,0,1) \} $ 
      \item $ W $ \hfill Απ: $ B_{W} = \{ (0,2,1,0), (1,0,0,1) \} $ 
      \item $ U \cap W $ \hfill Απ: $ B_{U\cap W} = \{ (0,2,1,0) \} $ 
    \end{enumerate}	

    % \item Να βρείτε ένα ομογενές σύστημα του οποίου ο χώρος λύσεων να παράγεται από τα διανύσματα:
    % 	\begin{enumerate}[(i)]
    % 		\item (1,-2,0,3,-1), (2,-3,2,5,-3), (1,-2,1,2,-2) \hfill Απ: $ \begin{cases}
    % 				5 x_{1} + x_{2} - x_{3} - x_{4} = 0 \\
    % 				x_{1} + x_{2} - x_{3} - x_{5} = 0
    % 			\end{cases} $ 
    % 		\item (1,1,2,1,1), (1,2,1,4,3), (3,5,4,9,7) \hfill Απ: $ \begin{cases}
    % 				2 x_{1} - x_{3} = 0 \\
    % 				2 x_{1} - 3 x_{2} + x_{4} = 0 \\
    % 				x_{1} - 2 x_{2} + x_{5} = 0
    % 			\end{cases} $ 
    % 	\end{enumerate}

  \item Έστω 
    \begin{enumerate*}[itemjoin=\hspace{1cm},label=(\alph*)]
      \item $ A = 
        \begin{pmatrix*}[r]
          0 & 0 & 3 & 1 & 4 \\
          1 & 3 & 1 & 2 & 1 \\
          3 & 9 & 4 & 5 & 2 \\
          4 & 12 & 8 & 8 &7
        \end{pmatrix*} $
      \item $ B = 
        \begin{pmatrix*}[r]
          1 & 2 & 1 & 0 & 1 \\
          1 & 2 & 2 & 1 & 3 \\
          3 & 6 &	5 & 2 & 7 \\
          2 & 4 & 1 & -1 & 0 
        \end{pmatrix*} $
    \end{enumerate*}

    Για κάθε πίνακα να βρείτε
    \begin{enumerate}[(i)]
      \item τον αντίστοιχο ανηγμένο κλιμακωτό πίνακα. 
      \item Βάση και διάσταση του χώρου στηλών.
      \item Βάση και διάσταση του χώρου γραμμών.
      \item Βάση και διάσταση του Μηδενόχωρου.
      \item Βάση και διάσταση του αριστερού Μηδενόχωρου.
    \end{enumerate}

    \hfill Απ: 
    \begin{enumerate*}[itemjoin=\hspace{1cm},label=(\alph*)]
      \item $ A \sim 
        \begin{pmatrix*}[r]
          1 & 3 & 0 & 0 & -\sfrac{13}{4} \\
          0 & 0 & 1 & 0 & \sfrac{3}{4} \\
          0 & 0 & 0 & 1 & \sfrac{7}{4} \\
          0 & 0 & 0 & 0 & 0 
        \end{pmatrix*} $
      \item $ B \sim 
        \begin{pmatrix*}[r]
          1 & 2 & 0 & -1 & -1 \\
          0 & 0 & 1 & 1 & 2 \\
          0 & 0 & 0 & 0 & 0 \\
          0 & 0 & 0 & 0 & 0 
        \end{pmatrix*} $
    \end{enumerate*}

  \item\label{ask:mhden} Έστω 
    \begin{enumerate*}[itemjoin=\hspace{1cm},label=(\alph*)]
      \item 	$ A =  \begin{pmatrix*}[r]
          1 & 2 & 1 & 3 & 1 & 6 \\
          2 & 4 & 3 & 8 & 3 & 9 \\
          1 & 2 & 2 & 5 & 3 & 11 \\
          4 & 8 & 6 & 16 & 7 & 26
        \end{pmatrix*}$
      \item $ B = \begin{pmatrix}
          1 & 2 & 2 & 1 & 2 & 1 \\
          2 & 4 & 5 & 4 & 5 & 5 \\
          1 & 2 & 3 & 4 & 4 & 6 \\
          3 & 6 & 7 & 7 & 9 & 10
        \end{pmatrix}$
    \end{enumerate*}

    Για κάθε πίνακα να βρείτε
    \begin{enumerate}[(i)]
      \item τον αντίστοιχο ανηγμένο κλιμακωτό πίνακα. 
      \item Βάση και διάσταση του χώρου στηλών.
      \item Βάση και διάσταση του χώρου γραμμών.
      \item Βάση και διάσταση του Μηδενόχωρου.
      \item Βάση και διάσταση του αριστερού Μηδενόχωρου.
    \end{enumerate}	

    \hfill Απ:  
    \begin{enumerate*}[itemjoin=\hspace{1cm},label=(\alph*)]
      \item $ A \sim \begin{pmatrix*}[r]
          1 & 2 & 0 & 1 & 0 & 9 \\
          0 & 0 & 1 & 2 & 0 & -11 \\
          0 & 0 & 0 & 0 & 1 & 8 \\
          0 & 0 & 0 & 0 & 0 & 0 
        \end{pmatrix*} $
      \item $ B \sim \begin{pmatrix*}[r]
          1 & 2 & 0 & 0 & 3 & 1 \\
          0 & 0 & 1 & 0 & -1 & -1 \\
          0 & 0 & 0 & 1 & 1 & 2 \\
          0 & 0 & 0 & 0 & 0 & 0 
        \end{pmatrix*}$
    \end{enumerate*}

  \item Να βρείτε την πλήρη λύση των παρακάτω συστημάτων με τη μέθοδο της απαλοιφής 
    Gauss.
    \begin{enumerate}[(i)]
      \item $ 
        \begin{pmatrix*}[r]
          1 & 2 & 1 & 0 \\
          2 & 4 & 4 & 8 \\
          4 & 8 & 6 & 8 
        \end{pmatrix*}
        \cdot 
        \begin{pmatrix*}[r]
          x_{1} \\ x_{2} \\ x_{3} \\ x_{4}
        \end{pmatrix*}
        = 
        \begin{pmatrix*}[r]
          4 \\ 2 \\ 10
        \end{pmatrix*} $ \hfill Απ: $ X = x_{2} 
        \begin{pmatrix*}[r]
          -2 \\ 1 \\ 0 \\ 0
        \end{pmatrix*}
        + x_{4} 
        \begin{pmatrix*}[r]
          4 \\ 0 \\ -4 \\ 1
        \end{pmatrix*}
        + 
        \begin{pmatrix*}[r]
          7 \\ 0 \\ -3 \\ 0
        \end{pmatrix*} $ 

      \item $ 
        \begin{pmatrix*}[r]
          2 & 4 & 6 & 4 \\
          2 & 5 & 7 & 6 \\
          2 & 3 & 5 & 2 
        \end{pmatrix*}
        \cdot
        \begin{pmatrix*}[r]
          x_{1} \\ x_{2} \\ x_{3} \\ x_{4}
        \end{pmatrix*} 
        =
        \begin{pmatrix*}[r]
          4 \\ 3 \\ 5 
        \end{pmatrix*}
        $ \hfill Απ: $ X = x_{3} 
        \begin{pmatrix*}[r]
          -1 \\ -1 \\ 1 \\ 0 
        \end{pmatrix*}
        + x_{4} 
        \begin{pmatrix*}[r]
          2 \\ -2 \\ 0 \\ 1
        \end{pmatrix*}
        + 
        \begin{pmatrix*}[r]
          4 \\ -1 \\ 0 \\ 0
        \end{pmatrix*} $ 
      \item $ 
        \begin{pmatrix*}[r]
          1 & 3 & 1 & 2 \\
          2 & 6 & 4 & 8 \\
          0 & 0 & 2 & 4 
        \end{pmatrix*}
        \cdot 
        \begin{pmatrix*}[r]
          x \\ y \\ z \\ w
        \end{pmatrix*}
        = 
        \begin{pmatrix*}[r]
          1 \\ 3 \\ 1
        \end{pmatrix*}
        $ \hfill Απ: $ X = 
        y
        \begin{pmatrix*}[r]
          -3 \\ 1 \\ 0 \\ 0
        \end{pmatrix*}
        + w
        \begin{pmatrix*}[r]
          0 \\ 0 \\ -2 \\ 1
        \end{pmatrix*}
        + 
        \begin{pmatrix*}[r]
          \sfrac{1}{2} \\ 0 \\ \sfrac{1}{2} \\ 0
        \end{pmatrix*} $
    \end{enumerate}	

  \item Να βρείτε μια βάση και τη διάσταση του Μηδενόχωρου για τους παρακάτω πίνακες  
    \begin{enumerate}[(i)]
      \item $
        \begin{pmatrix*}[r]
          1 & 1 & 5 & 1 & 4 \\
          2 & -1 & 1 & 2 & 2 \\
          3 & 0 & 6 & 0 & -3 
        \end{pmatrix*}$

        \hfill Απ: $ B = \{ (-2,-3,1,0,0), (1,-2,0,-3,1) \} $ 

      \item $ 
        \begin{pmatrix*}[r]
          0 & 0 & 1 & 2 \\
          0 & 0 & 1 & 2 \\
          1 & 1 & 1 & 0
        \end{pmatrix*} $ 

        \hfill Απ: $ B = \{ (-1,1,0,0), (2,0,-2,1) \} $ 
    \end{enumerate}

  \item Έστω $ W $ ο υπόχωρος του $\mathbb{R}^{4}$ που παράγεται από τα διανύσματα 
    $ \mathbf{u} _{1} = (1,-2,5,-3), \mathbf{u}_{2} = (2,3,1,-4) $, 
    $ \mathbf{u}_{3} = (3,8,-3,-5) $.
    \begin{enumerate}[(i)]
      \item Να βρείτε μια βάση και τη διάσταση του $ W $.
      \item Να επεκτείνετε τη βάση του $ W $ σε μια βάση του $ \mathbb{R}^{4} $.
    \end{enumerate}

    \hfill Απ:  \begin{tabular}{l}
      $B = \{ (1,-2,5,-3), (0,7,-9,2) \}$ \\
      $ B = \{ (1,-2,5,-3), (0,7,-9,2), (0,0,1,0), (0,0,0,1) \} $
    \end{tabular} 

  \item Να βρείτε τον αντίστροφο του πίνακα $ A = 
    \begin{pmatrix*}[r]
      1 & 0 & 2 \\
      2 & -1 & 3 \\
      4 & 1 & 8
    \end{pmatrix*}$ με τη μέθοδο της απαλοιφής Gauss.

    \hfill Απ: $ A^{-1} = 
    \begin{pmatrix*}[r]
      -11 & 2 & 2 \\
      -4 & 0 & 1 \\
      6 & -1 & -1
    \end{pmatrix*} $ 

  \item Να βρείτε τον αντίστροφο του πίνακα $ A = 
    \begin{pmatrix*}[r]
      1 & 0 & 0 \\
      -2 & 1 & 0 \\
      -1 & -1 & 1
    \end{pmatrix*}$ με τη μέθοδο της απαλοιφής Gauss.

    \hfill Απ: $ A^{-1} = 
    \begin{pmatrix*}[r]
      1 & 0 & 0 \\
      2 & 1 & 0 \\
      1 & 1 & 1
    \end{pmatrix*}$
\end{enumerate}


\end{document}
