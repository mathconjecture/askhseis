\documentclass[a4paper,table]{report}
\input{$HOME/Desktop/preamble/preamble.tex}
\input{$HOME/Desktop/preamble/definitions.tex}


\everymath{\displaystyle}

\thispagestyle{empty}


\begin{document}

\begin{center}
    \fbox{\large\bfseries Ασκήσεις στο πεδίο ορισμού }
\end{center}

\vspace{\baselineskip}


\begin{enumerate}
    \item  Να βρείτε το πεδίο ορισμού των παρακάτω συναρτήσεων.
        \begin{enumerate}[i)]
            \item $ f(x) = x^{3} - 5x^{2} + 2x -3 $ \hfill Απ: $ \mathbb{R} $
            \item $ f(x) = \frac{x-2}{x-3} $ \hfill Απ: $ \mathbb{R} \setminus 
                \{ 3 \} $ 
            \item $ f(x) = \frac{x}{2x-1} $ \hfill Απ: 
                $ \mathbb{R} \setminus \left\{ \frac{1}{2} \right\} $ 
            \item $ f(x) = 2x^{5} - \frac{1}{x} + \frac{x^{3}}{x+3}  $ \hfill 
                Απ: $ \mathbb{R} \setminus \{ 0, -3 \} $
            \item $ f(x) = \frac{x-1}{x^{2} - 5x + 6} $ \hfill Απ: 
                $ \mathbb{R} \setminus \{ 2,3 \}  $ 
            \item $ f(x) = \frac{4}{x^{2}+x+1} $ \hfill Απ: $ \mathbb{R} $ 
            \item $ f(x) = \frac{2x-1}{x^{3}-8} $ \hfill Απ: 
                $ \mathbb{R} \setminus \{ 1 \}  $ 
            \item $ f(x) = \frac{x+4}{x^{3}-4x} -2 + \frac{1}{x^{2}+2x} $ 
                \hfill Απ: $ \mathbb{R} \setminus \{ 0,-2,2 \} $ 
        \end{enumerate}

    \item  Να βρείτε το πεδίο ορισμού των παρακάτω συναρτήσεων.
        \begin{enumerate}[i)]
            \item $ f(x) = \sqrt{x-2} $ \hfill Απ: $[2,+\infty)$
            \item $ f(x) = \sqrt{4-x} -3x \sqrt{x+2} $ \hfill Απ: $ [-2,4] $ 
            \item $ f(x) = \sqrt{x^{2}-2x-3} $ \hfill Απ: $ (-\infty,-1] 
                \cup [3,+\infty) $  
            \item $ f(x) = \frac{\sqrt{x+3}}{x+1} $ \hfill Απ: 
                $ [-3,-1) \cup (-1,+\infty)  $ 
            \item $ f(x) = \frac{\sqrt{x-2}}{\sqrt{x-3}} $ \hfill Απ:
                $ ( 3, +\infty ) $  
            \item $ f(x) = \frac{\sqrt{\abs{x}-2}}{x-3} $ \hfill 
                Απ: $ ( -\infty, -2 ] \cup [2,3) \cup (3,+\infty) $ 
            \item $ f(x) = \frac{\sqrt{\abs{x}-x}}{\abs{x}-2} $ \hfill Απ: 
                $ (-\infty,-2) \cup (-2,0] $
        \end{enumerate}

    \item Να υπολογίσετε τις τιμές του πραγματικού αριθμού $ \lambda $, για τις 
        οποίες η συνάρτηση $ f(x) = \frac{x^{3}-8}{x^{2}-4x- \lambda} $ έχει πεδίο 
        ορισμού το $ \mathbb{R} $.

        \hfill Απ: $ \lambda < -4 $ 

    \item  Να βρείτε το πεδίο ορισμού των παρακάτω συναρτήσεων.
        \begin{enumerate}[i)]
            \item $ f(x) = \ln{(x-3)} $ \hfill Απ: $(3,+\infty)$
            \item $ f(x) = \ln{(-x^{2}+5x-6)} $ \hfill Απ: $ (2,3) $
            \item $ f(x) = \ln{(4x-x^{2})} $ \hfill Απ: $ (0,4) $
            \item $ f(x) = \ln{(-x^{2}+3x-2)} $ \hfill Απ: $ (1,2) $ 
            \item $ f(x) = \frac{1}{\ln{(x-1)}} $ \hfill Απ: $ (1,2) \cup (2,+\infty) $ 
            \item $ f(x) = \frac{1+e^{x}}{1-e^{x}} $ \hfill Απ: 
                $ \mathbb{R} \setminus \{ 0 \} $ 
        \end{enumerate}


\end{enumerate}






\end{document}
