\input{$HOME/Desktop/preamble/preamble.tex}
\input{$HOME/Desktop/preamble/definitions.tex}


\everymath{\displaystyle}
\thispagestyle{empty}

\begin{document}

\begin{center}
\fbox{\Large \bfseries Ασκήσεις Όρια - Συνέχεια - Παραγωγισιμότητα}
\end{center}


\vspace{\baselineskip}

\begin{enumerate}
  \item Να υπολογιστούν τα παρακάτω όρια:
  \begin{enumerate}[i)]

    \item $\lim\limits_{(x,y)\to (1,1)}\frac{x+y}{\sqrt{\abs{x-1}}+\sqrt{\abs{y+1}}}$\hfill Απ: $0$
    \item $\lim\limits_{(x,y)\to (0,0)}\frac{\sin(xy)}{x}$ \hfill Απ: $0$
    \item $\lim\limits_{(x,y)\to (0,0)}\frac{x^{2}y^{2}}{x^{2}+y^{2}+(x-y)^{2}}$ \hfill Απ: Δεν υπάρχει
  \end{enumerate}

  \item Δίνεται η συνάρτηση $f(x,y)=\frac{x^{2}y^{2}}{x^{2}y^{2}+(x+y)^{2}}$, για κάθε $(x,y)\in \mathbb{R}-\{0,0\}$. Να επεκταθεί κατάλληλα ώστε να είναι συνεχής στο $(0,0)$.

  \item Να εξεταστούν πλήρως ως προς τη συνέχεια οι συνάρτησεις:
  \begin{enumerate}[i)]
    \item   \(
      f(x,y) = \begin{cases}
        \sqrt{x^{2}+y^{2}}\ln(x^{2}+y^{2}) &, (x,y)\neq (0,0) \\
        0 &, (x,y)=(0,0)
    \end{cases}
      \) \hfill Απ: συνεχής
      \item   \(
        f(x,y)=\begin{cases}
          e^{-\frac{x^{2}}{y}} &, (x,y)\in\mathbb{R}^{2}-\{(x,y)\mid y=0\} \\
          0 &, (x,0)
      \end{cases}
        \) \hfill Απ: οχι συνεχής για $y=0$
        \item   \(
          f(x,y)=\begin{cases}
            y^{2}x\sin \frac{1}{x^{2}+y^{2}} &, (x,y)\neq (0,0)\\
            0 &, (x,y)=(0,0)
        \end{cases}
          \) \hfill Απ: συνεχής
  \end{enumerate}

  \item Δείξτε ότι η συνάρτηση $Y=f(x+at)+g(x-at)$ ικανοποιεί την εξίσωση $\pdv[2]{Y}{t}=a^{2}\pdv[2]{Y}{x}$, όπου $f,g$ είναι τουλάχιστον $2$ φορές παραγωγίσιμες συναρτήσεις και $a=$ σταθ.

  \item Η εξίσωση $F(x+y-z,x^{2}+y^{2})=0$ ορίζει μια πεπλεγμένη εξίσωση για τη συνάρτηση $z=z(x,y)$. Να δείξετε ότι η συνάρτηση ικανοποιεί την εξίσωση $xz_{y}-yz_{x}=x-y$

  \item Έστω η συνάρτηση $z=f(x,y)$ με $x=x(u,v)$ και $y=y(u,v)$ όπου ισχύει ότι $x_{u}=y_{v}$ και $x_{v}=-y_{u}$. Τότε να δειχθεί ότι: $F_{uu}+F_{vv}=(F_{xx}+F_{yy})({x_{u}}^{2}+{x_{v}}^{2})$

  \item Να δείξετε ότι η εξίσωση $xyz+x+y-z=0$ μπορεί να λυθεί ως προς $z$ σε μια περιοχή του σημείου $(0,0,0)$. Να βρεθεί η $z_{xy}$.
  \end{enumerate}



\end{document}
