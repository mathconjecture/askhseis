\documentclass[a4paper,table]{report}
\input{preamble.tex}
\input{definitions.tex}

\usepackage{systeme}

\pagestyle{askhseis}


\setlength{\itemsep}{\baselineskip}

\begin{document}
\begin{center}
  \minibox{\large \bfseries \textcolor{Col1}{Ασκήσεις στα Γραμμικά Συστήματα}}
\end{center}

\vspace{\baselineskip}

\begin{enumerate}

  \item Να λυθούν τα παρακάτω γραμμικά συστήματα $ 2 \times 2 $ με αντικατάσταση ή με 
    μέθοδο αντίθετων συντελεστών.

    \begin{enumerate}[i)]
      \item $ 
      \sysdelim.\}
      \systeme{
        x + 2y = 5, 
        4x + y = 6
      } $ 
      \hfill Απ: 
      \begin{tabular}{l}  
        $x=1 $ \\ 
        $y=2 $ 
      \end{tabular}

    \item $ 
    \sysdelim.\}
    \systeme{
      4x+3y=11,
      5x+7y=17
    } $ 
    \hfill Απ: 
    \begin{tabular}{l}  
      $x=2 $ \\ 
      $y=1 $ 
    \end{tabular}
\end{enumerate}

  \item Να λυθούν τα παρακάτω γραμμικά συστήματα.

    \begin{enumerate}[i)]
      \item $ 
      \sysdelim.\}
      \systeme{
        x+y+z=0,
        x-2y-2z=-3,
        2x+y+z=-1
      } $ 
      \hfill Απ: \begin{tabular}{l}  
        $x=-1 $ \\ 
        $ y=1-z $ \\
        $z \in \mathbb{R}  $
      \end{tabular}

    \item $ 
    \sysdelim.\}
    \systeme{
      x+y-3z=-1,
      4x-2y+6z=8,
      15x-3y+9z=21
    } $ 
    \hfill Απ: \begin{tabular}{l}  
      $x=1 $ \\ 
      $ y=-2+3z $ \\
      $z \in \mathbb{R}  $
    \end{tabular}

  \item $ 
  \sysdelim.\}
  \systeme{
    x+2y+z=2,
    x+y+2z=-1,
    x+3y=6
  } $ 
  \hfill Απ: αδύνατο 

\item $ 
\sysdelim.\}
\systeme{
  -y+z=0,
  x-z=0,
  x-y=-1
} $ 
\hfill Απ: αδύνατο 

\end{enumerate}
\end{enumerate}

\end{document}
