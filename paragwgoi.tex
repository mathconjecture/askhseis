\input{$HOME/Desktop/preamble/preamble.tex}
\input{$HOME/Desktop/preamble/definitions.tex}



\begin{document}

\begin{center}
	\fbox{\Large \bfseries Ασκήσεις στις Παραγώγους}
\end{center}

\vspace{\baselineskip}

\begin{enumerate}
	\item Δίνεται η σχέση $ x^{2} + xy + y^{3} -2x + 3y = 0 $, $ y=y(x) $. Να βρεθεί η
		παράγωγος της $y$ στο $ x=0 $. 

		\hfill Απ: $ y'= \frac{2}{3} $
	\item Δίνεται η σχέση $ x^{2} - xy + y^{2} = 3 $, $ y=y(x) $. Να βρεθεί η 1η
		και η 2η παράγωγος της $y$ ως προς $x$ στο σημείο $ (1,-1) $.

		\hfill Απ: $ y' = 1$, $ y'' = \frac{2}{3} $

	\item Δίνεται η σχέση $ 4x^{3} - 3xy^{2} + 6x^{2} - 5xy - 8 y^{2} + 9x + 14
		= 0$. Να βρείτε τις εξισώσεις της εφαπτομένης και της κάθετης ευθείας
		της καμπύλης στο σημείο $ (-2,3) $.

		\hfill Απ: $\varepsilon\colon y = \frac{9}{2} x - 6 $, $\kappa\colon y = \frac{2}{9} x +
		\frac{31}{9} $.

	\item Δίνονται οι παραμετρικές εξισώσεις $ x = 3 \cos^{3}{t} $, $ y = 4
		\sin^{3}{t}	$. Να βρεθεί η 1η και η 2η παράγωγος της συνάρτησης $y$.

		\hfill Απ: $ y' = -\frac{4}{3} \tan{t} $, $ y'' = \frac{4}{27}
		\frac{1}{\cos^{4}t \sin{t}} $ 

	\item Να βρεθεί η παράγωγος της συνάρτησης $ y = \left(1 +
		\frac{1}{x} \right)^{x} $.

		\hfill Απ: $ y' = \left(1 + \frac{1}{x}\right)\left[\ln(1 + \frac{1}{x}) -
		\frac{1}{x+1}\right] $
	\item {\bfseries (Σεπ 2017)}
		\begin{enumerate}[i)]
			\item Να δοθεί ο ορισμός καθώς και η γεωμετρική
				ερμηνεία του διαφορικού πρώτης τάξης της συνάρτησης $ y = g(x) $ στο
				τυχαίο σημείο $x$. 
			\item Να βρεθεί το διαφορικό δεύτερης τάξης της σύνθετης συνάρτησης $ z(x) =
				f(u(x))	$.
		\end{enumerate}
	\item Να υπολογιστούν κατά προσέγγιση οι τιμές:
		\begin{enumerate}[i)]
			\item $\sqrt{50}$ \hfill Απ: $7+\frac{1}{14}$
			\item $\sqrt[4]{17}$ \hfill Απ: $\frac{1}{4}17^{-\frac{3}{4}}+2$
		\end{enumerate}
	\item Να βρεθεί η παράγωγος της συνάρτησης $ y= \left[(\sin{x}) \cdot
		x^{2}\right]^{(25)}$, χρησιμοποιώντας τον τύπο \textlatin{Leinbiz}.

		\hfill Απ: $ y' = (x^{2} - 600) \cos{x} + 50 x \sin{x} $

	\item {\bfseries (Ιαν 2018)} Να αποδείξετε ότι η εξίσωση $ x^{2} = x \sin{x} + \cos{x} $ έχει δύο ακριβώς
		πραγματικές ρίζες $ x_{1} $, $ x_{2} $, με $ x_{1} \in (-\pi, 0) $, και
		$x_{2} \in (0, \pi) $.

	\item Να αποδείξετε την παρακάτω ανισότητα   
		\[
			\frac{a - b}{\cos^{2}{b}} \leq \tan{a} - \tan{b}\leq \frac{a -
			b}{\cos^{2}{a}}, \qq{με}  0 < b \leq a < \frac{\pi}{2}
		\]
	\item{\bfseries (Ιαν 2016)} Να βρείτε μια πολυωνυμική προσέγγιση μέχρι και όρους 3ης τάξης της
		συνάρτησης που ορίζεται πεπλεγμένα από την εξίσωση $ x^{2} - xy + y^{2}
		= 3$ στο σημείο $ (1,-1) $.

		\hfill Απ: $f(x) \cong -1 + (x-1) + \frac{(x-1)^{2}}{3} +
		\frac{(x-1){3}}{9}$

	\item Να δειχθεί ότι η συνάρτηση $ f(x) = \frac{x^{2} + 6x + 12}{x^{2} - 6x
		+ 12} $ είναι καλή προσέγγιση της συνάρτησης $ e^{x} $ για μικρές τιμές
		του $x$ διότι τα αναπτύγματα των δύο συναρτήσεων συμπίπτουν στους 5
		πρώτους όρους. 

	\item Να δείξετε ότι 
		\[
			\sin{x} = \sin{a} + \cos{a} (x-a) - \frac{\sin{a}}{2!} (x-a)^{2} -
			\frac{\cos{\xi} (x-a)^{3}}{3!}
		\]

		όπου $\xi$ μεταξύ $a$ και $x$. Στη συνέχεια να υπολογίσετε το $
		\sin{\ang{51}}$ καθώς και το διαπραττόμενο σφάλμα.
\end{enumerate}


\end{document}
