\documentclass[a4paper,12pt]{article}



%%%%%%%%%%%%%%%%%%%%%%%%%%%%%%%%%%%%%%
% Babel language package
\usepackage[english,greek]{babel}
% Inputenc font encoding
\usepackage[utf8]{inputenc}
%%%%%%%%%%%%%%%%%%%%%%%%%%%%%%%%%%%%%%





\begin{document}

\begin{center}
\fbox{\large \textbf{Μαθηματικά 1}}
\end{center}

\vspace{\baselineskip}

\begin{description}
    \item[Διάλεξη 2] Να διαβάσεις τα παρακάτω:
  \begin{itemize}
    \item Μαθηματική Επαγωγή
    \item Διωνυμικο Θεώρημα
  \end{itemize}

  \item[Διάλεξη 3] Να διαβάσεις τα παρακάτω. Περισσότερη προσοχή
  στις παραγράφους που έχω τονίσει (αν έχεις απορίες, με ρωτάς):
  \begin{itemize}
    \item \textbf{Ακολουθίες} (Ορισμόί και παραδείγματα)
    \item \textbf{Φράγμα Ακολουθίας} (Θεώρημα και παραδείγματα)
    \item \textbf{Μονοτονία Ακολουθιών} (Ορισμοί και παραδείγματα)
    \item \textbf{Όριο Ακολουθίας} (Ορισμός, ιδιότητες, θεώρημα παρεμβολής,
    Θεώρημα
    Μονότονης σύγκλισης και παραδείγματα)
    \item {Όριο Ακολουθίας στο Άπειρο}
  \end{itemize}
  \item[Διάλεξη 4] Να διαβάσεις τα παρακάτω:
  \begin{itemize}
    \item \textbf{Σειρές Αριθμών} (Ορισμός και Ορισμός Σύγκλισης σειράς,
    \textbf{Γεωμετρική, Τηλεσκοπική, Αρμονική, Εκθετική Σειρά,
    Θεώρημα ν-οστού όρου})
    \item \textbf{Απόλυτη Σύγκλιση}
    \item \textbf{Κριτήρια Σύγκλισης} (Απόλυτης Σύγκλισης, Σύγκρισης, Λόγου,
    Ρίζας, Εναλλασσόμενης Σειράς)
  \end{itemize}
  \item[Διάλεξη 5--6] Πρέπει να ξέρεις τις γραφικές παραστάσεις των βασικών
  συναρτήσεων και επίσης:
  \begin{itemize}
    \item \textbf{Υπερβατικές Συναρτήσεις} (Τριγωνομετρικές, Αντίστροφες
    Τριγωνομετρικές, Εκθετικές, Λογαριθμικές καθώς και βασικές ταυτότητες
    και ιδιότητες αυτών)

    Από τις αντίστροφες τριγωνομετρικές συναρτήσεις
    (επειδή είναι καινούριες, τις μαθαίνετε στο 1ο ετος για πρωτη φορα) το
    μόνο που χρειάζεται να ξέρεις ειναι οι γραφικές τους παραστάσεις, οι
    παράγωγοί τους και τα ολοκληρώματα που συνδέονται με αυτές.
  \end{itemize}
  \item[Διάλεξη 7] Δεν χρειάζεται να διαβασεις τον ορισμό του ορίου (ε-δ)

\begin{itemize}
  \item Ιδιότητες Ορίων
  \item Θεώρημα Παρεμβολής (Παραδείγματα)
  \item Συνέχεια Συναρτήσεων (Τα βασικά, όπως τα ξέρεις από το Λύκειο, μην
  ασχοληθεις με τις αποδειξεις)
  \item Θεωρήματα \textlatin{Bolzano} και Ενδιάμεσης Τιμής
\end{itemize}

\item[Διάλεξη 8] Να διαβάσεις τα παρακάτω:
\begin{itemize}
  \item Πεπλεγμένη Παραγώγιση (Θα χρειαστεί για τα μαθηματικά 2)
  \item Εξίσωση Εφαπτομένης (σε γραφική παράσταση συνάρτησης)
\end{itemize}

\item[Διάλεξη 9] Να διαβάσεις τα παρακάτω:
\begin{itemize}
  \item \textbf{Θεωρήματα \textlatin{Rolle} και Μέσεης Τιμής}
  \item \textbf{Τύπος \textlatin{Taylor} (σχέση 5) και Πολυώνμο
   \textlatin{Taylor}} και παραδείγματα.
\end{itemize}

\item[Διάλεξη 10] Να θυμηθείς τα ακρότατα συνάρτησης 1 μεταβλητής...
και κυρίως το κριτήριο της 2ης παραγώγου... (όχι τοσο τα πινακακια
μονοτονίας)
\begin{itemize}
  \item \textbf{Μέγιστα και Ελάχιστα} και παραδείγματα
  \item \textbf{\textlatin{L' Hospital}}
\end{itemize}
\item[Διάλεξη 10--11] Ρίξε μια ματιά... αν κ μπορώ να σου δώσω
δικές μου σημειώσεις και λυμένα παραδείγματα.
\begin{itemize}
  \item Γενικευμένα Ολοκληρώματα (Είναι καινουρια, τα μαθαινετε στο
  1ο ετος, μοιαζουν με τα ορισμενα, αλλα παιρνουμε το οριο, οταν δεν
  μπορουμε να κανουμε αντικατασταση τα ακρα ολοκληρωσης)
\end{itemize}
\item[Διάλεξη 13] Ρίξε μια ματια... αν και θα τα δουμε στο μάθημα καλύτερα!
\begin{itemize}
  \item \textbf{Κριτήριο Ολοκληρώματος} (Συνδέει το Γενικευμένο Ολοκλήρωμα με τις
  Σειρές)
  \item \textbf{Δυναμοσειρές} (Ορισμός, Ακτίνα και Διάστημα Σύγκλισης)
  \item \textbf{Παράγωγος και Ολοκλήρωμα Δυναμοσειράς}
  \item \textbf{Τύποι \textlatin{Mac Laurin, Taylor}}
\end{itemize}

\end{description}

\end{document}
