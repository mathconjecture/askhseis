\input{preamble_ask.tex}
\input{definitions_ask.tex}

\usepackage{systeme}

\pagestyle{askhseis}

\setlength{\itemsep}{\baselineskip}



\begin{document}

\begin{center}
  \minibox{\large \bfseries \textcolor{Col1}{Ασκήσεις στα παραμετρικά  συστήματα}}
\end{center}

\vspace{\baselineskip}

\begin{enumerate}
  \setlength{\itemsep}{\baselineskip}
\item Για ποιές τιμές της παραμέτρου $ \lambda $ τα παρακάτω συστήματα έχουν 
  μοναδική λύση.

  \begin{enumerate}[i)]
    \setlength{\itemsep}{\baselineskip}
  \item $\sysdelim.\} \systeme{4x-2\lambda y + 6z = 8, \lambda x + y -3z = -\lambda, 15x -3y + 9z = 21
  } $ \hfill Απ: \begin{tabular}{l}
    $\lambda \neq 1$ και $ \lambda \neq -5 $
  \end{tabular}


\item $ \sysdelim.\} \systeme{
  x + 2y + z = 2, 
  x + y + (\lambda  \+\ 1)z = -1, 
  x + (\lambda  \+\ 2)y + (1 \-\ \lambda)z = 4 \lambda +2
} $ \hfill Απ: \begin{tabular}{l}
    $\lambda \neq 0$ και $ \lambda \neq 1 $
\end{tabular} 
\end{enumerate}
\end{enumerate}


\end{document}
