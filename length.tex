\input{preamble/preamble.tex}
\input{preamble/definitions.tex}

\everymath{\displaystyle}
\thispagestyle{empty}

\begin{document}


\begin{center}
    \minibox[frame]{\large\bfseries Ασκήσεις στο μήκος καμπύλης}
\end{center}

\vspace{\baselineskip}

\begin{enumerate}

  \item Να υπολογιστεί το μήκος της καμπύλης $y=x^{\frac{3}{2}}-1$ στο διάστημα 
      $x\in[0,3]$.

  \hfill Απ: $\frac{1}{27}(31\sqrt{31}-8)$

  \item Να υπολογιστεί το μήκος της καμπύλης $y=\ln(\cos x)$ όταν 
      $\frac{\pi}{6}\leq x \leq \frac{\pi}{3}$.

  \hfill Απ: $\ln\frac{2+\sqrt{3}}{\sqrt{3}}$

  \item Να υπολογιστεί το μήκος της καμπύλης $y=\frac{e^{x}+e^{-x}}{2}$, από 
      $x=0$ έως $x=1$.

  \hfill Απ: $\frac{e^{2}-1}{2e}$

  \item Να υπολογιστεί το μήκος του τόξου της καμπύλης $x=e^{t}\sin t$, 
      $y=e^{t}\cos t$, όταν $t\in [0,\frac{\pi}{2}]$.

  \hfill Απ: $\sqrt{2}(e^{\frac{\pi}{2}}-1)$

  \item Η θέση ενός κινητού τη χρονική στιγμή $t$ δίνεται από τις σχέσεις $x=1+t^{3}$, 
      $y=2-t^{2}$. Να βρεθεί η απόσταση που θα διανύσει αν ταξιδεύει από 
      $t=0$ έως $t=2$.

  \hfill Απ: $\frac{8}{27}(10\sqrt{10}-1)$

  \item Να υπολογιστεί το μήκος της καμπύλης $x=1+\sin t$, $y=2+\cos t$, όταν 
      $0\leq t\leq \pi$.

  \hfill Απ: $4$

  \item Να υπολογιστεί το μήκος της καμπύλης $x=2-\ln(1+t^{2})$, $y=1+\arccos t$ για 
      $0\leq t\leq 1$.

  \hfill Απ: $\ln(\sqrt{2}+1)$

\end{enumerate}

\end{document}
