\input{$HOME/Desktop/preamble/preamble.tex}
\input{$HOME/Desktop/preamble/definitions.tex}


\everymath{\displaystyle}

\begin{document}

\begin{center}
	\fbox{\Large \bfseries Προβλήματα Ακροτάτων}
\end{center}

\vspace{\baselineskip}

\begin{enumerate}

	\item {\bfseries \boldmathΝα προσδιοριστούν οι διαστάσεις μιας πισίνας $ \SI{32}{m^{3}} $ με
			τετραγωνική βάση έτσι ώστε η επιφάνεια των εσωτερικών τοίχων και του
		πυθμένα να είναι ελάχιστη.}

		\fbox{\bfseries Λύση}

		Το εμβαδό της πισίνας δίνεται από τη σχέση
		\inlineequation[eq:sur1]{E = x^{2} + 4xy}

		Ο όγκος της πισίνας είναι 
		\begin{align}
			V &= 32 \notag \Leftrightarrow \\
			x^{2}y &= 32 \notag \Leftrightarrow \\
			y &= \frac{32}{x^{2}} \label{eq:vol1} 
		\end{align}
		Άρα με αντικατάσταση της σχέσης~\eqref{eq:vol1} στη
		σχέση~\eqref{eq:sur1} έχουμε
		\begin{align*}
			E(x) &= x^{2} + 4x\cdot \frac{32}{x^{2}} \Rightarrow \\
			E(x) &= x^{2} +	\frac{128}{x}
		\end{align*}
		Αναζητάμε ελάχιστο για αυτή τη συνάρτηση.
		\begin{itemize}
			\item $ E'(x) = 2x - \frac{128}{x^{2}} $ 
			\item $ E'(x) = 0 \Leftrightarrow 2x - \frac{128}{x^{2}} = 0
				\Leftrightarrow \frac{2x^{3} - 128}{x^{2}} = 0 \Leftrightarrow
				2x^{3} = 128 \Leftrightarrow x^{3} = 64 \Leftrightarrow$
				\inlineequation[eq:sta1]{\boxed{x = 4}}
			\item $ E''(x) = 2 + \frac{256}{x^{3}} $
			\item $ E''(4) = 2 + \frac{256}{64} = 6 > 0 $, επομένως πρόκειται για
				ολικό ελάχιστο.
		\end{itemize}
		Με αντικατάσταση της σχέσης~\eqref{eq:sta1} στη σχέση~\eqref{eq:vol1} έχουμε $ y = 2 $.

		Επομένως οι διαστάσεις της πισίνας πρέπει να είναι $ x = 4 $ και $ y = 2 $.

	\item {\bfseries \boldmath Να βρεθεί η εξίσωση της ευθείας $ (\varepsilon)
			$ που περνάει από γνωστό σημείο $ P(a,b) $ και σχηματίζει με τους άξονες
		συντεταγμένων τρίγωνο ελάχιστου εμβαδού $ (a>0,\, b>0) $.}

		\fbox{\bfseries Λύση}

		Το εμβαδό του τριγώνου δίνεται από τη σχέση \inlineequation[eq:sur2]{E = \frac{1}{2}
		xy}. 

		Η εξίσωση ευθείας που διέρχεται από το σημείο $ P(a,b) $ είναι η 
		\inlineequation[eq:line2]{\varepsilon: y-b = \lambda(x-a)}

		Για να βρούμε τα σημεία τομής της ευθείας με τους άξονες, έχουμε

		Για $ x = 0 $ η σχέση~\eqref{eq:line2} δίνει $ y = b - a\lambda $.

		Για $ y = 0 $ η ίδια σχέση δίνει $ x = a - \frac{b}{\lambda} $. 

		Με αντικατάσταση των $x$ και $y$ που μόλις βρήκαμε στη
		σχέση~\eqref{eq:sur2} έχουμε
		\begin{align}
			E &= \frac{1}{2} \abs{a - \frac{b}{\lambda}} \cdot \abs{b -
			a\lambda} \Leftrightarrow \notag \\
			2E &= \abs{\frac{a\lambda - b}{\lambda}} \cdot \abs{b - a\lambda}
			\Leftrightarrow \notag \\
			2E \abs{\lambda} &= \abs{a\lambda - b}\cdot \abs{b - a\lambda}
			\notag \\
			2E \abs{\lambda} &= (a\lambda - b)^{2} \label{eq:sur2a}
		\end{align}
		Διακρίνουμε τις παρακάτω δύο περιπτώσεις.
		\begin{itemize}
			\item Αν $ \lambda > 0 $ τότε από τη σχέση~\eqref{eq:sur2a} έχουμε
				\begin{align*}
					&2E\lambda  = (a\lambda - b)^{2} \Leftrightarrow \\
					&a^{2}\lambda^{2} - 2ab\lambda + b^{2} - 2\lambda E = 0
					\Leftrightarrow \\
					&a^{2}\lambda^{2} - 2(ab + E)\lambda + b^{2} = 0 \qq{(τριώνυμο
					ως προς $\lambda$)}
				\end{align*}
				Υπολογίζουμε τη Διακρίνουσα του τριωνύμου.
				\[
					\Delta = 4(ab+E)^{2} - 4a^{2}b^{2} = 4a^{2}{b}^{2} + 8abE +
					4E^{2} - 4a^{2}b^{2} = 4E(2ab + E) 
				\]
				Θέλουμε να έχουμε πραγματικές λύσεις για το $\lambda$, οπότε
				\[
					\Delta \geq 0 \Leftrightarrow 4E(2ab+E) \geq 0
					\Leftrightarrow E\leq -2ab \qq{(απορ.) ή} E\geq 0  	
				\]
				Σε αυτή την περίπτωση η μικρότερη τιμή για το εμβαδό είναι $
				E = 0 $ και είναι η τετριμμένη περίπτωση. Από τη
				σχέση~\eqref{eq:sur2a} προκύπτει ότι το
				$\lambda = b/a $ και με αντικατάσταση στη
				σχέση~\eqref{eq:line2} έχουμε ότι $ \varepsilon: y =
				(b/a)x $ που σημαίνει ότι η ευθεία περνά από την αρχή των
				αξόνων.

			\item Αν $ \lambda < 0 $ τότε από τη σχέση~\eqref{eq:sur2a} έχουμε
				\begin{align*}
					&2E(-\lambda)  = (a\lambda - b)^{2} \Leftrightarrow \\
					&a^{2}\lambda^{2} - 2ab\lambda + b^{2} + 2\lambda E = 0
					\Leftrightarrow \\
					&a^{2}\lambda^{2} - 2(ab - E)\lambda + b^{2} = 0 \qq{(τριώνυμο
					ως προς $\lambda$)}
				\end{align*}
				Υπολογίζουμε τη Διακρίνουσα του τριωνύμου.
				\[
					\Delta = 4(ab-E)^{2} - 4a^{2}b^{2} = 4a^{2}{b}^{2} - 8abE +
					4E^{2} - 4a^{2}b^{2} = 4E(2ab - E) 
				\]
				Θέλουμε να έχουμε πραγματικές λύσεις για το $\lambda$, οπότε
				\[
					\Delta \geq 0 \Leftrightarrow 4E(2ab-E) \geq 0
					\Leftrightarrow E\leq 0 \qq{(απορ.) ή} E\geq 2ab  	
				\]
				Σε αυτή την περίπτωση η μικρότερη τιμή για το εμβαδό είναι $
				E = 2ab $. Από τη σχέση~\eqref{eq:sur2a} προκύπτει ότι το
				$\lambda = -b/a $ και με αντικατάσταση στη
				σχέση~\eqref{eq:line2} προκύπτει ότι $ bx+ay-2ab=0 $ που σημαίνει ότι η
				ευθεία περνά από την αρχή των αξόνων.

				Επομένως η εξίσωση της ζητούμενης ευθείας είναι
				\[
					\boxed{y=-\frac{b}{a}x +2b}	
				\]
		\end{itemize}






\end{enumerate}

\end{document}
