\documentclass[a4paper,table]{report}
\input{preamble.tex}
\input{definitions.tex}

\pagestyle{askhseis}

\begin{document}

\begin{center}
  \minibox{\large\bfseries \textcolor{Col1}{Ασκήσεις στα Σύνολα}}
\end{center}

\vspace{\baselineskip}

\begin{enumerate}

\item Έστω τα σύνολα $A=\{1,2,3\}, B=\{1,2\}, C=\{1,3\}, D=\{2,3\}, E=\{1\}, F=\{2\}, G=\{3\}$. Να βρεθούν τα παρακάτω σύνολα.

\begin{enumerate}[i)]

\item $A\cup B$
\item $A\cap B$
\item $A\cap (B\cap C)$
\item $(C\cup A)\cap B$
\item $A\setminus B$
\item $C\setminus A$
\item $D\triangle F = (D\setminus F)\cup(F\setminus D)$
\end{enumerate}

\item Να βρεθεί η ένωση και η τομή των συνόλων $A$ και $B$ στις παρακάτω περιπτώσεις.

\begin{enumerate}[i)]

\item Αν είναι $A=\mathbb{R}\setminus\{1,2\}$ και $B=\mathbb{R}\setminus\{1,3\}$
\item Αν είναι $A=\mathbb{R}\setminus\{1,2\}$ και $B=[1,+\infty)/$
\item Αν είναι $A=(3,+\infty)$ και $B=(-\infty,5]$
\end{enumerate}

\item Δίνεται το γενικό σύνολο $S=\{x\in \mathbb{N} : 0<x<10\}$ και τα υποσύνολά του $A=\{x\in \mathbb{N} : 1\leq x\leq 5\}$ και $B=\{x\in \mathbb{N} : 3\leq x\leq 8\}$. Να παραστήσετε αναλυτικά τα παρακάτω σύνολα:

\begin{enumerate}[i)]

\item $A\cup B$
\item $A\cap B$
\item $A^c$
\item $B^c$
\item $(A\cup B)^c$
\item $(A\cap B)^c$
\item $A^c\cap B^c$
\item $A^c\cup B^c$
\end{enumerate}



\end{enumerate}



\end{document}

