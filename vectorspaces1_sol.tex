\input{preamble_ask.tex}
\input{definitions_ask.tex}


\zexternaldocument*{vectorspaces1}

\pagestyle{vangelis}


\begin{document}


\chapter*{Υποδείξεις}

\section*{Γραμμική Ανεξαρτησία}

\begin{dfn}
  Τα διανύσματα $ \mathbf{u_{1}}, \mathbf{u_{2}, \ldots, \mathbf{u_{n}}} $ 
  θα είναι \textcolor{Col1}{γραμμικώς ανεξάρτητα} αν το ομογενές σύστημα
  \[
    \lambda _{1} \mathbf{u_{1}}+ \lambda _{2} \mathbf{u_{2}} + \cdots + \lambda _{n} 
    \mathbf{u_{n}} = \mathbf{0}
  \] 
  έχει μοναδική λύση $ \lambda _{1} = \lambda _{2} = \cdots = \lambda _{n} = 0 $, τη  
  μηδενική, \textbf{αλλιώς} λέγονται \textcolor{Col1}{γραμμικώς εξαρτημένα} (και κατά 
  συνέπεια το ομογενές σύστημα θα έχει άπειρες λύσεις. \textbf{Θυμάμαι:} ένα ομογενές 
  σύστημα δεν είναι ποτέ αδύνατο. Έχει πάντα λύση. Είτε μοναδική, τη μηδενική, 
  είτε άπειρες)
\end{dfn}

\subsection*{Άσκηση~\ref{ask:lineks}}
Τα διανύσματα $ \mathbf{u} = (1,3,1,2), \mathbf{v} = (2,5,-1,3) $ και $ \mathbf{w} = 
(1,3,7,-2) $ θα είναι \textbf{γραμμικώς ανεξάρτητα}, αν και μόνον αν 
το (\textbf{ομογενές σύστημα}) 
\[
  \lambda _{1} \mathbf{u} + \lambda _{2} \mathbf{v} + \lambda _{3} \mathbf{w} =
  \mathbf{0} \Leftrightarrow \cdots \Leftrightarrow 
  \begin{pmatrix*}[r]
    1 & 2 & 1 \\
    3 & 5 & 3 \\
    1 & -1 & 7 \\
    \undermat{A}{2 & 3 & -2}
  \end{pmatrix*} 
  \cdot 
  \begin{pmatrix*}[r] \lambda _{1} \\ \lambda _{2} \\ \lambda _{3} \end{pmatrix*} 
  = 
  \begin{pmatrix*}[r] 0 \\ 0 \\ 0 \\ 0 \end{pmatrix*} 
\] 
έχει \textcolor{Col1}{μοναδική λύση} την $ \lambda _{1} = \lambda _{2} = 
\lambda _{3} = 0 $ (τη μηδενική). 

Δηλαδή, ισοδύναμα αν $ \color{Col1}{rank(A) = 3} $ όσα και τα διανύσματα 
τα οποία εξετάζω (γιατί, \textbf{θυμάμαι ότι ένα σύστημα $ A_{m \times n} \cdot X = b $ 
  έχει μοναδική λύση αν και μόνον αν $ \rank(A) = \rank(E) = n $, όπου $n$ είναι το 
πλήθος των αγνώστων})

Άρα αρκεί να βρω το $ \rank(A) $ ή το $ \rank(A^{T}) $ (γιατί \textbf{θυμάμαι} ότι 
$ \rank(A) = \rank(A^{T})) $ 

Οπότε βάζω τα διανύσματα σε έναν πίνακα $A$ ως γραμμές (ή ως στήλες) και βρίσκω το 
$ \rank(A) $
\[
  A = 
  \begin{pmatrix*}[r]
    1 & 3 & 1 &-2 \\
    2 & 5 & -1 & 3 \\
    1 & 3 & 7 &  -2
  \end{pmatrix*} \sim \cdots \sim 
  \begin{pmatrix*}[c]
    \Circle{1} & 0 & 0 & 19 \\
    0 & \Circle{1} & 0 & -7 \\
    0 & 0 & \Circle{1} & 0
  \end{pmatrix*}
\] 
Άρα $ \rank(A) = 3 $, όσα και τα διανύσματα, επομένως γραμμικώς ανεξάρτητα.

\begin{concl}
  Για να εξετάσω αν $\textcolor{Col1}{n}$ το πλήθος διανύσματα είναι γραμμικώς 
  ανεξάρτητα, τότε ξεκινάω με τον ορισμό, γράφω το σύστημα με μορφή 
  πινάκων και στη συνέχεια:
  \begin{myitemize}
    \item Αν $A_{n \times n}$ τετραγωνικός, δηλαδή αν τα διανύσματα σχηματίζουν 
      ορίζουσα, τότε: 
      \begin{enumerate}[i)]
        \item Αν $ \abs{A} \neq 0 $ τότε τα διανύσματα είναι γραμμικώς ανεξάρτητα.
        \item Αν $ \abs{A} = 0 $ τότε τα διανύσματα είναι γραμμικώς εξαρτημένα.
      \end{enumerate}
    \item Αν $A_{m\times n}$ μη-τετραγωνικός, δηλαδή τα διανύσματα δεν σχηματίζουν
      ορίζουσα, τότε:
      \begin{enumerate}[i)]
        \item Αν $ \rank(A) = n $ όσα και τα διανύσματα, τότε είναι γραμμικώς ανεξάρτητα.
        \item Αν $ \rank(A) < n $, τότε τα διανύσματα είναι γραμμικώς εξαρτημένα.
      \end{enumerate}
  \end{myitemize}
\end{concl}

\subsection*{Άσκηση~\ref{ask:baseeks}}

Ξέρω ότι $ \dim(\mathbb{R}^{3}) = 3 $, οπότε κάθε βάση του $ \mathbb{R}^{3} $ θα έχει 
3 διανύσματα. Οπότε τα διανύσματα $ \mathbf{u}, \mathbf{v} $ και $ \mathbf{w} $ θα 
είναι βάση του $ \mathbb{R}^{3} $ αν και μόνον αν είναι γραμμικώς ανεξάρτητα. Οπότε
\[
  \lambda _{1} \mathbf{u} + \lambda _{2} \mathbf{v} + \lambda _{3} \mathbf{w} = 
  \mathbf{0}
  \Leftrightarrow \cdots \Leftrightarrow 
  \begin{pmatrix*}[r]
    1 & 1 & 1 \\
    1 & 2 & 5 \\
    \undermat{A}{1 & 3 & 3}
  \end{pmatrix*} \cdot 
  \begin{pmatrix*}[r] \lambda _{1} \\ \lambda _{2} \\ \lambda _{3} \end{pmatrix*} = 
  \begin{pmatrix*}[r] 0 \\ 0 \\ 0 \end{pmatrix*}
\] 
Οπότε 
\[
  \abs{A} = 
  \begin{vmatrix*}[r]
    1 & 1 & 1 \\
    1 & 2 & 5 \\
    1 & 3 & 5
  \end{vmatrix*} = \cdots
\]

\subsection*{Άσκηση~\ref{ask:synd}} 

Αφού τα διανύσματα είναι γραμμικώς εξαρτημένα, τότε το σύστημα
\[
  \lambda _{1} \mathbf{u_{1}}+ \lambda _{2} \mathbf{u}_{2} + \lambda _{3} \mathbf{u_{3}}+
  \lambda _{4} \mathbf{u_{4}} + \lambda _{5} \mathbf{u_{5}} = \mathbf{0} 
  \Leftrightarrow \cdots \Leftrightarrow 
  \begin{pmatrix*}[r]
    1 & 1 & 1 & 1 & 1  \\
    3 & 2 & 1 & 1 & -3 \\
    0 & 1 & 2 & 2 & 6  \\
    5 & 4 & 3 & 3 & -1
  \end{pmatrix*} \cdot 
  \begin{pmatrix*}[r] \lambda _{1} \\ \lambda _{2} \\ \lambda _{3} \\ \lambda _{4} \\
  \lambda _{5} \end{pmatrix*} = 
  \begin{pmatrix*}[r] 0 \\ 0 \\ 0 \\ 0 \end{pmatrix*}
\] 
θα έχει \textbf{άπειρες λύσεις}. Τις βρίσκω με Gauss:
\[
  \begin{pmatrix*}[r]
    1 & 1 & 1 & 1 & 1  \\
    3 & 2 & 1 & 1 & -3 \\
    0 & 1 & 2 & 2 & 6 \\
    5 & 4 & 3 & 3 & -1
  \end{pmatrix*} \sim \cdots \sim
  \begin{pmatrix*}
    \Circle{1} & 0 & -1 & -1 & -5 \\
    0 & \Circle{1} & \phantom{-}2 & \phantom{-}2 & \phantom{-}6 \\
    0 & 0 & \phantom{-}0 & \phantom{-}0 & \phantom{-}0 \\
    0 & 0 & \phantom{-}0 & \phantom{-}0 & \phantom{-}0 
  \end{pmatrix*} \Leftrightarrow 
  \begin{pmatrix*}[r] \lambda _{1} \\ \lambda _{2} \\ \lambda _{3} \\ \lambda _{4} \\
  \lambda _{5}  \end{pmatrix*} = \lambda _{3} 
  \begin{pmatrix*}[r] 1 \\ -2 \\ 1 \\ 0 \\ 0  \end{pmatrix*} + \lambda _{4} 
  \begin{pmatrix*}[r] 1 \\ -2 \\ 0 \\ 1 \\ 0  \end{pmatrix*} + \lambda _{5} 
  \begin{pmatrix*}[r] 5 \\ -6 \\ 0 \\ 0 \\ 1  \end{pmatrix*}
\] 



\textbf{Θυμάμαι} ότι αν τα διανύσματα είναι γραμμικώς εξαρτημένα, τότε κάποιο ή 
κάποια από αυτά θα είναι γραμμικός συνδυασμός των υπολοίπων, οπότε αυτή τη σχέση 
πρέπει να βρω, τον γραμμικό συνδυασμό που τα συνδέει. Δηλαδή, \textbf{μία} από τις 
άπειρες λύσεις του συστήματος. Οπότε, επιλέγω \textbf{τυχαία} μια τιμή για τα 
$ \lambda _{3}, \lambda _{4}, \lambda _{5} $, (τις ελεύθερες μεταβλητές) έστω: 
\[
  \lambda _{3} = 1 \quad \text{και} \quad \lambda _{4} = \lambda _{5} = 0 
\] 
Άρα 
\[
  \begin{pmatrix*}[r] \lambda _{1} \\ \lambda _{2} \\ \lambda _{3} \\ \lambda _{4} \\
  \lambda _{5}  \end{pmatrix*} = 1
  \begin{pmatrix*}[r] 1 \\ -2 \\ 1 \\ 0 \\ 0  \end{pmatrix*} + 0
  \begin{pmatrix*}[r] 1 \\ -2 \\ 0 \\ 1 \\ 0  \end{pmatrix*} + 0
  \begin{pmatrix*}[r] 5 \\ -6 \\ 0 \\ 0 \\ 1  \end{pmatrix*} \Leftrightarrow 
  \begin{pmatrix*}[r] \lambda _{1} \\ \lambda _{2} \\ \lambda _{3} \\ \lambda _{4} \\
  \lambda _{5}  \end{pmatrix*} = 
  \begin{pmatrix*}[r] 1 \\ -2 \\ 1 \\ 0 \\ 0  \end{pmatrix*} \Leftrightarrow 
  \left.
    \begin{matrix*}[l]
      \lambda _{1} = \phantom{-} 1 \\
      \lambda _{2} = -2 \\
      \lambda _{3} = \phantom{-}1 \\
      \lambda _{4} = \phantom{-}0 \\
      \lambda _{5} = \phantom{-}0 
    \end{matrix*} 
  \right\} 
\] 
Άρα από τη σχέση
\[
  \lambda _{1} \mathbf{u_{1}}+ \lambda _{2} \mathbf{u}_{2} + \lambda _{3} \mathbf{u_{3}}+
  \lambda _{4} \mathbf{u_{4}} + \lambda _{5} \mathbf{u_{5}} = \mathbf{0} 
\]
παίρνουμε για τις παραπάνω τιμές των $ \lambda _{1}, \lambda _{2}, \lambda _{3}, \lambda
_{4}, \lambda _{5} $ ότι
\[
  1 \mathbf{u_{1}} -2 \mathbf{u}_{2} + 1\mathbf{u_{3}}+
  0 \mathbf{u_{4}} + 0 \mathbf{u_{5}} = \mathbf{0} \Leftrightarrow 
  \boxed{\mathbf{u_{1}} - 2\mathbf{u_{2}} + \mathbf{u_{3}} = \mathbf{0}}
\]

\pagebreak

\section*{Γραμμικοί Συνδυασμοί}

\subsection*{Άσκηση~\ref{ask:eksart}}
Το διάνυσμα $ \mathbf{b} $ θα είναι γραμμικός συνδυασμός των $ \mathbf{u}, \mathbf{v}, 
\mathbf{w}$, αν υπάρχουν $ x_{1}, x_{2}, x_{3} \in \mathbb{R} $ τέτοιοι ώστε 
\[ 
  \mathbf{b} = x_{1} \mathbf{u} + x_{2} \mathbf{v} + x_{3} \mathbf{w} \Leftrightarrow 
  \cdots \Leftrightarrow 
  \begin{pmatrix*}[r]
    1 & 1 & 2 \\
    1 & 2 & -1 \\
    1 & 3 & 1
  \end{pmatrix*} \cdot 
  \begin{pmatrix*}[r]  x_{1} \\ x_{2} \\ x_{3} \end{pmatrix*} = 
  \begin{pmatrix*}[r] 1 \\ -2 \\ 5 \end{pmatrix*}
\]
δηλαδή ισοδύναμα, αν το σύστημα έχει λύση. Επειδή θέλει να εκφράσουμε το $ \mathbf{b} $
ως γραμμικό συνδυασμό των $ \mathbf{u}, \mathbf{v}, \mathbf{w} $, λύνουμε το σύστημα.
\[
  \begin{pmatrix*}[r]
    1 & 1 & 2 & \vrule &  1 \\
    1 & 2 & -1 & \vrule & -2 \\
    1 & 3 & 1 & \vrule & 5
  \end{pmatrix*} \sim \cdots \sim 
  \begin{pmatrix*}[c]
    \Circle{1} & 0 & 0 & \vrule & -6 \\
    0 & \Circle{1} & 0 & \vrule & \phantom{-}3 \\
    0 & 0 & \Circle{1} & \vrule & \phantom{-}2
  \end{pmatrix*} \Leftrightarrow 
  \left.
    \begin{matrix*}[l]
      x_{1}= -6 \\
      x_{2} = \phantom{-}3 \\
      x_{3} = \phantom{-}2
    \end{matrix*} 
  \right\} 
\] 
Άρα, έχουμε 
\[
  \boxed{\mathbf{b} = -6 \mathbf{u} + 3 \mathbf{v}+ 2 \mathbf{w}}
\]  

\subsection*{Άσκηση~\ref{ask:eksart2}}

Το διάνυσμα $ \mathbf{b} $ θα είναι γραμμικός συνδυασμός των $ \mathbf{u}, \mathbf{v}, 
\mathbf{w}$, αν υπάρχουν $ x_{1}, x_{2}, x_{3} \in \mathbb{R} $ τέτοιοι ώστε 
\[ 
  \mathbf{b} = x_{1} \mathbf{u} + x_{2} \mathbf{v} + x_{3} \mathbf{w} \Leftrightarrow 
  \cdots \Leftrightarrow 
  \begin{pmatrix*}[r]
    1 & 2 & 1 \\
    -3 & 4 & -5 \\
    2 & 1 & 7
  \end{pmatrix*} \cdot 
  \begin{pmatrix*}[r]  x_{1} \\ x_{2} \\ x_{3} \end{pmatrix*} = 
  \begin{pmatrix*}[r] 2 \\ 5 \\ -3 \end{pmatrix*}
\]
δηλαδή ισοδύναμα, αν το σύστημα έχει λύση. Επειδή θέλει απλώς να εξετάσουμε αν 
το $ \mathbf{b} $ γράφεται ως γραμμικός συνδυασμός των 
$ \mathbf{u}, \mathbf{v}, \mathbf{w} $, δεν χρειάζεται να λύσουμε το σύστημα απλώς να 
το \textcolor{Col1}{διερευνήσουμε}. Άρα
\[
  \begin{pmatrix*}[r]
    1 & 2 & 1 & \vrule & 2 \\
    -3 & 4 & -5 & \vrule & 5 \\
    2 & 1 & 7 & \vrule & -3
  \end{pmatrix*}
  \sim \cdots \sim 
  \begin{pmatrix*}[c]
    \Circle{1} & 0 & 7/5 & \vrule & -1/5 \\
    0 & \Circle{1} & -1/5 & \vrule & \phantom{-}11/10 \\
    \undermat{\text{κλιμακωτός}}{0 & 0 & \Circle{22/5} & \vrule & \phantom{-}37/10}
  \end{pmatrix*} 
\] 

\vspace{\baselineskip}

Έχουμε 
\[
  \rank(A) = \rank(E) = 3 \quad \text{(όσοι οι άγνωστοι)}
\] 
οπότε από το θεώρημα Gauss το σύστημα έχει μοναδική λύση. Επομένως το $ \mathbf{b} $ 
γράφεται ως γραμμικός συνδυασμός των $ \mathbf{u}, \mathbf{v}, \mathbf{w} $. 

\section*{Παραγόμενοι Υπόχωροι, Βάσεις, Διάσταση}

Έστω $ S = \{ \mathbf{u_{1}}, \mathbf{u_{2}}, \ldots, \mathbf{u_{n}} \} $ 
υποσύνολο ενός $ \mathbb{K}- $χώρου. Τότε το σύνολο \textcolor{Col1}{όλων} 
των γραμμικών συνδυασμών των στοιχείων του $S$, συμβολίζεται με $ < S >  $ και 
είναι ο \textcolor{Col1}{μικρότερος} υπόχωρος που παράγεται από τα στοιχεία του $S$. 
Δηλαδή 
\[
  < S > = \{ \lambda _{1} \mathbf{u_{1}} + \lambda _{2} \mathbf{u_{2}} + \cdots + 
  \lambda _{n} \mathbf{u_{n}} \; : \; \lambda _{i} \in \mathbb{K} \}   
\] 

\begin{myitemize}[leftmargin=*]
  \item Ισχύει ότι $ \mathbf{b} \in < S > $, αν το $ \mathbf{b} $ είναι 
    γραμμικός συνδυασμός των στοιχείων του $S$, δηλαδή αν υπάρχουν 
    $ \lambda _{1}, \ldots, \lambda _{n} \in \mathbb{K} $, τέτοιοι ώστε 
    \[ 
      \mathbf{b} = \lambda _{1} \mathbf{u_{1}} + \cdots, 
      \lambda _{n} \mathbf{u_{n}} 
    \]
  \item Ισχύει ότι ένα σύνολο $ S = \{ \mathbf{u_{1}}, \mathbf{u_{2}}, 
    \ldots, \mathbf{u_{n}} \} $ \textcolor{Col1}{παράγει} έναν διανυσματικό χώρο $V$, 
    αν \textcolor{Col1}{κάθε} στοιχείο του $V$ γράφεται ως γραμμικός συνδυασμός 
    των στοιχείων του $S$. 
\end{myitemize}


\subsection*{Άσκηση~\ref{ask:eksart3}} 
Το διάνυσμα $ \mathbf{b} $ θα ανήκει στο χώρο που παράγουν τα $ \mathbf{u_{1}},
\mathbf{u_{2}}, \mathbf{u_{3}} $, δηλαδή $ \mathbf{b} \in < \mathbf{u_{1}},
\mathbf{u_{2}}, \mathbf{u_{3}}> $, αν το $ \mathbf{b} $ γράφεται ως γραμμικός 
συνδυασμός των $ \mathbf{u_{1}}, \mathbf{u_{2}}, \mathbf{u_{3}} $, δηλαδή ισοδύναμα, 
αν υπάρχουν $ x_{1}, x_{2}, x_{3} \in 
\mathbb{R} $ τέτοιοι ώστε 
\[
  \mathbf{b} = x_{1} \mathbf{u_{1}} + x_{2} \mathbf{u_{2}} + x_{3} \mathbf{u_{3}} 
  \Leftrightarrow \cdots \Leftrightarrow 
  \begin{pmatrix*}[r]
    1 & -1 & 3 \\
    2 & 1 & 0 \\
    0 & 2 & -4
  \end{pmatrix*} \cdot 
  \begin{pmatrix*}[r] x_{1} \\ x_{2} \\ x_{3} \end{pmatrix*} = 
  \begin{pmatrix*}[r] a \\ b \\ c \end{pmatrix*}
\] 
δηλαδή ισοδύναμα, αν το σύστημα έχει λύση. Δηλαδή, αρκεί να βρούμε ποιες συνθήκες 
απαιτούνται ώστε το σύστημα να έχει λύση. Έχουμε
\[
  \begin{pmatrix*}[r]
    1 & -1 & 3 & \vrule & a \\
    2 & 1 & 0 & \vrule & b \\
    0 & 2 & -4 & \vrule & c
  \end{pmatrix*}
  \sim 
  \begin{pmatrix*}[r]
    1 & -1 & 3 & \vrule & a \\
    0 & 3 & -6 & \vrule & b-2a \\
    0 & 2 & -4 & \vrule & c
  \end{pmatrix*} \sim 
  \begin{pmatrix*}[c]
    \Circle{1} & -1 & \phantom{-}3 & \vrule & a \\
    0 & \Circle{1} & -2 & \vrule & \frac{b-2a}{3} \\
    0 & 0 & \phantom{-}0 & \vrule & c -2\frac{b-2a}{3}  
  \end{pmatrix*}
\]
Παρατηρούμε ότι προκειμένου να έχει λύση το σύστημα, θα πρέπει, σύμφωνα με το θεώρημα 
Gauss να ισχύει:
\[
  \textcolor{Col1}{\rank(A) = \rank(E)} \Leftrightarrow c - 2 \frac{b-2a}{3} = 0 \Leftrightarrow 
  \frac{3c-2b+4a}{3} = 0 \Leftrightarrow \boxed{4a -2b +3c = 0}
\] 

\subsection*{Άσκηση~\ref{ask:parag}} 

\begin{description}
  \item[1ος τρόπος] 
    Σύμφωνα με τον ορισμό, τα $ \mathbf{u_{1}}, \mathbf{u_{2}}, \mathbf{u_{3}} $ θα 
    παράγουν τον $ \mathbb{R}^{3} $ αν και μόνον αν \textcolor{Col1}{κάθε} στοιχείο του 
    $ \mathbb{R}^{3} $ είναι γραμμικός συδνυασμός τους.

    Δηλαδή αν για κάθε $ \mathbf{x} = (x,y,z) \in \mathbb{R}^{3} $  υπάρχουν 
    $ x_{1}, x_{2}, x_{3} \in \mathbb{R} $, τέτοιοι ώστε
    \[
      \mathbf{x} = x_{1} \mathbf{u_{1}}+ x_{2} \mathbf{u_{2}}+ x_{3} \mathbf{u_{3}}
      \Leftrightarrow \ldots \Leftrightarrow 
      \begin{pmatrix*}[r]
        1 & 1 & 1 \\
        1 & 0 & -2 \\
        3 & 2 & 0
      \end{pmatrix*} \cdot 
      \begin{pmatrix*}[r] x_{1} \\ x_{2} \\ x_{3} \end{pmatrix*} = 
      \begin{pmatrix*}[r] x \\ y \\ z \end{pmatrix*}
    \] 
    δηλαδή, ισοδύναμα, αν το σύστημα έχει λύση για \textcolor{Col1}{κάθε} 
    $x,y,z \in \mathbb{R}$. Έχουμε
    \[
      \begin{pmatrix*}[r]
        1 & 1 & 1 & \vrule & x \\
        1 & 0 & -2 & \vrule & y \\
        3 & 2 & 0 & \vrule & z
      \end{pmatrix*} \sim \cdots \sim 
      \begin{pmatrix*}[r]
        1 & 0 & -2 & \vrule & x \\
        0 & 1 & 3 & \vrule & x - y \\
        0 & 0 & 0 & \vrule & z - 2x -y 
      \end{pmatrix*}
    \] 
    Παρατηρούμε ότι το σύστημα θα έχει λύση, αν
    \[
      \rank(A) = \rank(E) \Leftrightarrow z-2x-y=0
    \] 
    Άρα, αν $ z-2x-y \neq 0 $ τότε το σύστημα είναι αδύνατο. Κατά συνέπεια, δεν ισχύει 
    ότι καθε $ (x,y,z) \in \mathbb{R}^{3} $ είναι συνδυασμός των $ \mathbf{u_{1}},
    \mathbf{u_{2}}, \mathbf{u_{3}} $ και επομένως τα $ \mathbf{u_{1}}, \mathbf{u_{2}},
    \mathbf{u_{3}} $ δεν παράγουν τον $ \mathbb{R}^{3} $.

  \item[2ος τρόπος] 
    Επειδή $ \dim(\mathbb{R}^{3}) = 3 $, έχουμε ότι 3 οποιαδήποτε διανύσματα του $
    \mathbb{R}^{3} $, που είναι γραμμικώς ανεξάρτητα, θα είναι βάση του, και επομενως
    θα τον παράγουν. Άρα, ελέγχουμε αν τα 
    $ \mathbf{u_{1}}, \mathbf{u_{2}}, \mathbf{u_{3}} $ είναι γραμμικώς ανεξάρτητα.
    \[
      \begin{vmatrix*}[r]
        1 & 1 & 1 \\
        1 & 0 & -2 \\
        3 & 2 & 0 
      \end{vmatrix*} = \cdots = 0
    \] 
    Άρα τα διανύσματα $ \mathbf{u_{1}}, \mathbf{u_{2}}, \mathbf{u_{3}}$ είναι 
    γραμμικώς ανεξάρτητα και επομένως δεν παράγουν τον $ \mathbb{R}^{3}$.
\end{description}



\subsection*{Άσκηση~\ref{ask:isoi}}

Για να δείξουμε ότι οι υπόχωροι που παράγονται από δύο διαφορετικά σύνολα διανυσμάτων
είναι ίσοι, ακολουθούμε τα παρακάτω βήματα:
\begin{myitemize}
  \item Βάζουμε τα διανύσματα του 1ου συνόλου ως \textcolor{Col1}{γραμμές} σε έναν 
    πίνακα $A$.
  \item Ομοίως, βάζουμε τα διανύσματα του 2ου συνόλου ως \textcolor{Col1}{γραμμές} 
    σε έναν πίνακα $Β$.
  \item Βρίσκουμε τους αντίστοιχους \textcolor{Col1}{ανηγμένους κλιμακωτούς} πίνακες 
    των πινάκων $A$ και $B$.
  \item Δείχνουμε ότι οι \textbf{βάσεις} που προκύπτουν από τις ανεξάρτητες γραμμές του 
    \textcolor{Col1}{ανηγμένου κλιμακωτού}, πίνακα, για τον χώρο γραμμών των 2 πινάκων, 
    είναι \textcolor{Col1}{ίδιες}.
\end{myitemize}

\subsection*{Άσκηση~\ref{ask:parag2}}

Ζητάμε τον υπόχωρο $ U = < \mathbf{u_{1}}, \mathbf{u_{2}}, \mathbf{u_{3}},
\mathbf{u_{4}} >  $. Οπότε, βάζουμε τα διανύσματα σε έναν πίνακα, ως γραμμές 
(ή ως στήλες, θέλω ο πίνακας που θα προκύψει να έχει όσο το δυνατόν λιγότερες γραμμές) 
και βρίσκουμε το χώρο γραμμών (ή στηλών αντίστοιχα) αυτού του πίνακα. 
\[
  A = \begin{pmatrix*}[r]
    1 & 1 & 1 & 2 & 3 \\
    1 & 2 & -1 & -2 & 1 \\
    3 & 5 & -1 & -2 & 5 \\
    1 & 2 & 1 & -1 & 4
  \end{pmatrix*} \sim \cdots \sim
  \begin{pmatrix*}[c]
    \Circle{1} & 0 & 0 & 9/2 & 1/2 \\
    0 & \Circle{1} & 0 & -3 & 1 \\
    0 & 0 & \Circle{1} & 1/2 & 3/2 \\
    0 & 0 & 0 & 0 & 0 \\
  \end{pmatrix*}
\] 
Άρα, ο χώρος γραμμών του $A$, και κατά συνέπεια, ο 
$ U = < \mathbf{u_{1}}, \mathbf{u_{2}}, \mathbf{u_{3}}, \mathbf{u_{4}}, 
\mathbf{u_{5}} >  $ θα είναι ο χώρος που παράγουν οι ανεξάρτητες γραμμές του $A$, 
δηλαδή εκείνες οι γραμμές του $A$, οι οποίες έχουν οδηγό στον αντίστοιχο κλιμακωτό (ή
ανηγμένο κλιμακωτό). Δηλαδή
\[
  R(A) = U = < \mathbf{u_{1}}, \mathbf{u_{2}}, \mathbf{u_{3}} > = 
  < (1,1,1,2,3), (1,2,-1,-2,1),(3,5,-1,-2,5) >  
\] 
Μια βάση για τον $ R(A) = U $ θα είναι η 
\[
  B = \{ \mathbf{u_{1}}, \mathbf{u_{2}}, \mathbf{u_{3}} \} = 
  \{  (1,1,1,2,3), (1,2,-1,-2,1),(3,5,-1,-2,5)  \} \Rightarrow 
  \dim(U) = 3
\]
Μια \textbf{2η βάση} για το χώρο γραμμών ενός πίνακα, δίνεται από τις
\textcolor{Col1}{ίδιες} τις γραμμές του αντίστοιχου κλιμακωτού 
(ή του ανηγμένου κλιμακωτού) οι οποίες έχουν οδηγό.  Έτσι το σύνολο 
\[
  B = \{ (1,0,0,9/2,1/2), (0,1,0,-3,1), (0,0,1,1/2,3/2)\}  
\] 
είναι μια ακόμη βάση του $U$.

\section*{Χώρος Γραμμών, Χώρος Στηλών, Μηδενόχωρος}

\subsection*{Άσκηση~\ref{ask:mhden}}

Για να βρούμε τον χώρο Γραμμών, τον χώρο Στηλών και τον Μηδενόχωρο ενός πίνακα, 
χρειαζόμαστε τον αντίστοιχο κλιμακωτό ή (ανηγμένο κλιμακωτό) πίνακα.
\[
  \begin{pmatrix*}[r]
    1 & 2 & 2 & 1 & 2 & 1 \\
    2 & 4 & 5 & 4 & 5 & 5 \\
    1 & 2 & 3 & 4 & 4 & 6 \\
    3 & 6 & 7 & 7 & 9 & 10 
  \end{pmatrix*} \sim \cdots \sim 
  \begin{pmatrix*}[c]
    \Circle{1} & 2 & 0 & 0 & \phantom{-}3 & \phantom{-}1 \\
    0 & 0 & \Circle{1} & 0 & -1 & -1 \\
    0 & 0 & 0 & \Circle{1} & \phantom{-}1 & \phantom{-}2 \\
    0 & 0 & 0 & 0 & \phantom{-}0 & \phantom{-}0 
  \end{pmatrix*}
\]
Οπότε, για τον $ m \times n $ πίνακα $A$, έχουμε:
\begin{myitemize}[leftmargin=*]
  \item \textcolor{Col1}{Χώρος Στηλών} του Α είναι ο χώρος που παράγεται 
    από τις \textbf{γραμμικώς ανεξάρτητες} \textcolor{Col1}{στήλες} του $A$. 
    Αυτές θα είναι η 1η η 3η και η 4η στήλη \textcolor{Col1}{του Α} 
    (αυτές, για τις οποίες οι αντίστοιχες στήλες του ανηγμένου κλιμακωτού έχουν οδηγό).
    Επομένως 
    \[
      C(A) = \left< 
      \begin{pmatrix*}[r] 1 \\ 2 \\ 1 \\ 3 \end{pmatrix*} , 
      \begin{pmatrix*}[r] 2 \\ 5 \\ 3 \\ 7 \end{pmatrix*} , 
      \begin{pmatrix*}[r] 1 \\ 4 \\ 4 \\ 7 \end{pmatrix*} \right> \Rightarrow 
      B_{C(A)} = \left\{  
        \begin{pmatrix*}[r] 1 \\ 2 \\ 1 \\ 3 \end{pmatrix*} , 
        \begin{pmatrix*}[r] 2 \\ 5 \\ 3 \\ 7 \end{pmatrix*} , 
      \begin{pmatrix*}[r] 1 \\ 4 \\ 4 \\ 7 \end{pmatrix*} \right\} \Rightarrow 
      \dim(C(A)) = 3
    \] 
    Θυμάμαι ότι $ \textcolor{Col1}{C(A) \leq \mathbb{R}^{m}} $, αν 
    $ A_{m\times n} $. Άρα $ C(A) \leq \mathbb{R}^{4} $
  \item \textcolor{Col1}{Χώρος Γραμμών} του Α είναι ο χώρος που παράγεται 
    από τις \textbf{γραμμικώς ανεξάρτητες} \textcolor{Col1}{γραμμές} του $A$. 
    Αυτές θα είναι η 1η η 2η και η 3η γραμμή \textcolor{Col1}{του Α} 
    (αυτές, για τις οποίες οι αντίστοιχες γραμμές του ανηγμένου κλιμακωτού έχουν οδηγό).
    Επομένως 
    \begin{gather*}
      R(A) = \left<(1,2,2.1,2,1), (2,4,5,4,5,5), (1,2,3,4,4,6) \right> \Rightarrow \\
      B_{R(A)} = \{ (1,2,2.1,2,1), (2,4,5,4,5,5), (1,2,3,4,4,6) \} \Rightarrow 
      \dim(R(A)) = 3
    \end{gather*} 
    Θυμάμαι, ότι μια ακόμη βάση για το χώρο Γραμμών του $A$, δίνεται από τις 
    γραμμικώς ανεξάρτητες γραμμές του αντίστοιχου κλιμακωτού (ή ανηγμένου κλιμακωτού).
    Οπότε 
    \[
      B'_{R(A)} = \{ (1,2,0,0,3,1), (0,0,1,0,-1,-1), (0,0,0,1,1,2) \}  
    \] 
    Επιβεβαιώνουμε ότι $ \dim(C(A)) = \dim(R(A)) = \rank(A) = 3 $
  \item Ο \textcolor{Col1}{Μηδενόχωρος} του $A$, είναι ο χώρος λύσεων του ομογενούς 
    συστήματος $ A \cdot X = \mathbf{0} $. Οπότε από τον ανηγμένο κλιμακωτό πίνακα του 
    $A$, παίρνουμε:
    \[
      \left.
        \begin{aligned}
          x_{1}+2 x_{2}+ 3 x_{5}+ x_{6}&= 0 \\
          x_{3}- x_{5}- x_{6}&= 0 \\
          x_{4}+ x_{5}+2 x_{6}&= 0 \\
          x_{2}, x_{5}, x_{6} \in \mathbb{R} 
        \end{aligned} 
      \right\}\!\! \Rightarrow 
      \left.
        \begin{aligned}
          x_{1} &= -2 x_{2} - 3 x_{5}- x_{6} \\
          x_{3} &= x_{5}+ x_{6} \\
          x_{4} &= - x_{5}- 2 x_{6} \\
          x_{2} &, x_{5}, x_{6} \in \mathbb{R} 
        \end{aligned} 
      \right\}\!\!\Rightarrow 
      \begin{pmatrix*}[r] 
        x_{1} \\ x_{2} \\ x_{3} \\ x_{4} \\ x_{5} \\ x_{6}  
      \end{pmatrix*} = x_{2} 
      \begin{pmatrix*}[r] -2 \\ 1 \\ 0 \\ 0 \\ 0 \\ 0 \end{pmatrix*} 
      + x_{5} 
      \begin{pmatrix*}[r] -3 \\ 0 \\ 1 \\ -1 \\ 1 \\ 0  \end{pmatrix*} 
      + x_{6} 
      \begin{pmatrix*}[r] -1 \\ 0 \\ 1 \\ -2 \\ 0 \\ 1  \end{pmatrix*} \!\!
    \] 
    Επομένως ο Μηδενόχωρος του πίνακα $A$ είναι:
    \[
      N(A) = \left< 
      \begin{pmatrix*}[r] -2 \\ 1 \\ 0 \\ 0 \\ 0 \\ 0  \end{pmatrix*}, 
      \begin{pmatrix*}[r] -3 \\ 0 \\ 1 \\ -1 \\ 1 \\ 0  \end{pmatrix*}, 
      \begin{pmatrix*}[r] -1 \\ 0 \\ 1 \\ -2 \\ 0 \\ 1 \end{pmatrix*} \right> 
      \Rightarrow B_{N(A)} = \left\{  
        \begin{pmatrix*}[r] -2 \\ 1 \\ 0 \\ 0 \\ 0 \\ 0  \end{pmatrix*}, 
        \begin{pmatrix*}[r] -3 \\ 0 \\ 1 \\ -1 \\ 1 \\ 0  \end{pmatrix*}, 
      \begin{pmatrix*}[r] -1 \\ 0 \\ 1 \\ -2 \\ 0 \\ 1 \end{pmatrix*} \right \}
      \Rightarrow \dim(N(A)) = 3
    \]
    Επιβεβαιώνουμε ότι $ \dim(N(A)) = n - \rank(A) = 6-3 = 3 $
  \item Ο \textcolor{Col1}{Αριστερός Μηδενόχωρος} του $A$, είναι ο χώρος λύσεων του 
    ομογενούς συστήματος $ A^{T} \cdot Y = \mathbf{0} $. Δηλαδή ο Αριστερός Μηδενόχωρος 
    του $A$ είναι ουσιαστικά ο Μηδενόχωρος του $ A^{T} $. Οπότε χρειαζόμαστε τον 
    ανηγμένο κλιμακωτό πίνακα του $A^{T}$.
    \[
      \begin{pmatrix*}[r]
        1 & 2 & 1 & 3 \\
        2 & 4 & 2 & 6 \\
        2 & 5 & 3 & 7 \\
        1 & 4 & 4 & 7 \\
        2 & 5 & 4 & 9 \\
        1 & 5 & 6 & 10
      \end{pmatrix*} \sim \cdots \sim 
      \begin{pmatrix*}[c]
        \Circle{1} & 0 & 0 & \phantom{-}3 \\
        0 & \Circle{1} & 0 & -1 \\
        0 & 0 & \Circle{1} & \phantom{-}2 \\
        0 & 0 & 0 & \phantom{-}0 \\
        0 & 0 & 0 & \phantom{-}0 \\
        0 & 0 & 0 & \phantom{-}0 \\
      \end{pmatrix*}
    \]
    Άρα όπως κάναμε και για τον Μηδενόχωρο του $A$, έχουμε:
    \[
      N(A^{T}) = \left< 
      \begin{pmatrix*}[r] -3 \\ 1 \\ -2 \\ 1 \\ 0 \\ 0  \end{pmatrix*} \right>   
      \Rightarrow B_{N(A^{T})} = \left\{ 
        \begin{pmatrix*}[r] -3 \\ 1 \\ -2 \\ 1 \\ 0 \\ 0  
      \end{pmatrix*} \right\} \Rightarrow \dim(N(A^{T})) = 1
    \] 
    Επιβεβαιώνουμε ότι $ \dim(N(A^{T})) = m - \rank(A) = 4-3 = 1 $
\end{myitemize}

\end{document}

