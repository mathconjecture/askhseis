\input{$HOME/Desktop/preamble/preamble.tex}
\input{$HOME/Desktop/preamble/definitions.tex}

\everymath{\displaystyle}

\begin{document}

\begin{center}
\fbox{\Large \bfseries Ασκήσεις Όρια - Συνέχεια - Παραγωγισιμότητα}
\end{center}


\vspace{\baselineskip}

\begin{enumerate}
  \item Να υπολογιστούν κατά προσέγγιση οι τιμές:
  \begin{enumerate}[i)]
    \item $\sqrt{50}$ \hfill Απ: $7+\frac{1}{14}$
    \item $\sqrt[4]{17}$ \hfill Απ: $\frac{1}{4}17^{-\frac{3}{4}}+2$
  \end{enumerate}

  \item Να υπολογιστούν τα παρακάτω όρια:
  \begin{enumerate}[i)]
    \item $\lim\limits_{x\to 1}\left(\frac{1}{\ln x}-\frac{1}{x-1}\right)$ \hfill Απ: $\frac{1}{2}$
    \item $\lim\limits_{x\to 1}(1-x)\tan \frac{\pi x}{2}$ \hfill Απ: $\frac{2}{\pi}$
    \item $\lim\limits_{x\to 0^{+}}\left(\frac{1}{x}\right)^{\sin x}$ \hfill Απ: $1$
    \item $\lim\limits_{x\to 0}(\cos 2x)^{\frac{3}{x^{2}}}$ \hfill Απ: $e^{-6}$
  \end{enumerate}

  \item Να δείξετε ότι για κάθε $x>0$ ισχύει $\ln x\leq \frac{x}{e}$. 

  \item Να δείξετε ότι για τη συνάρτηση 
	  \[
		  f(x) = \begin{cases}
			  (x+1)^{2}, & -2\leq x\leq -1 \\
			  (x+1)^{3}, & -1<x\leq 0
		  \end{cases}  
	  \]
	  υπάρχει $ x_{0} \in (-2,0) $, τέτοια ώστε $ f'(x_{0}) = 0 $.

	  \hfill Απ: $ x_{0} = -1 $

  \item Να δείξετε ότι η $ e^{x} = x+1 $ έχει μόνο μια πραγματική ρίζα.
	  
	  \hfill Απ: $ f(0) = 0 $

  \item Αν μια συνάρτηση $f$ είναι συνεχής στο $ [a,b] $, παραγωγίσιμη στο $
	  (a,b) $ και ισχύει $ f(a) = f(b) $, να δείξετε ότι υπάρχουν $ x_{1}, x_{2}
	  \in (a,b)$, τέτοια ώστε $ f'(x_{1}) + f'(x_{2}) = 0 $.

\end{enumerate}

\end{document}
