\documentclass[a4paper,table]{report}
\input{preamble_ask.tex}
\input{definitions_ask.tex}


\geometry{top=2cm}
% \DeclareMathOperator{\r}{r}

\pagestyle{askhseis}

\begin{document}




\section*{Μέθοδος Gauss}


\begin{enumerate}
  \setlength{\itemsep}{\baselineskip}
  \item Να λυθεί τo παρακάτω συστήμα με τη μέθοδο \textbf{απαλοιφής} (\textbf{Gauss}).

    \begin{enumerate}[i)]
      \setlength{\itemsep}{\baselineskip}
    \item $\sysdelim.\} \systeme{\phantom{-}x+4y-2z=1, y+z=3, 2x+11y-z= 11}$ 
      \hfill Απ: \begin{tabular}{l}
        $ x=6z-11 $ \\
        $ y=3-z $ \\
        $z \in \mathbb{R}$
      \end{tabular}
  \end{enumerate}
\end{enumerate}


\section*{Παραμετρικά Συστήματα}


\begin{enumerate}


  \item Να λυθούν για τις διάφορες τιμές των παραμέτρων τα παρακάτω συστήματα.


    \begin{enumerate}[i)]
    \setlength{\itemsep}{\baselineskip}
    \item $\sysdelim.\} \systeme{{(1-\lambda)} x - 2y + 2z = 0, x + {(2 \lambda-2)} y + 2z = 0, x - 2y + {(2-2 \lambda)} z = 0 }$ 
      \hfill Απ: \begin{tabular}{l}
        $\lambda\neq 0, 3 \; $, μοναδική λύση \\
        $\lambda=0 \; $, άπειρες λύσεις \\
        $\lambda=3 \; $, άπειρες λύσεις
      \end{tabular}

    \item $\sysdelim.\} \systeme{ \lambda x+ \mu y+z=1,x+{\lambda \mu }y+z=b,x+ \mu y+
      \lambda z=1}$ με $ \mu \neq 0 $ 
      \hfill Απ: \begin{tabular}{l} 
        $ \lambda \neq 1, -2 \; $, μοναδική λύση \\
        $ \lambda =-2 \;, \mu \neq -2 $, αδύνατο \\
        $ \lambda =-2 \;, \mu = -2 $, άπειρες λύσεις
      \end{tabular}
  \end{enumerate}
\end{enumerate}


\section*{Ιδιοτιμές - Ιδιοδιανύσματα - Διαγωνιοποίηση}


\begin{enumerate}
  \item Να βρεθεί ο \textbf{ορθογώνιος} πίνακας $Q$ που διαγωνιοποιεί τον πίνακα $A$. 

    \begin{enumerate}[i)]
      \item $ A = 
        \begin{pmatrix*}[r]
          4 & 0 & -1 \\
          0 & 4 & -1 \\
          -1 & -1 & 5
        \end{pmatrix*} $ \hfill Απ: $ Q = 
        \begin{pmatrix*}[r]
          -1/ \sqrt{2} & 1 / \sqrt{3} & - 1 / \sqrt{6} \\ 
          1/ \sqrt{2} & 1 / \sqrt{3} & - 1 / \sqrt{6} \\ 
          0 & 1 / \sqrt{3} &  2 / \sqrt{6} 
        \end{pmatrix*} $ 
    \end{enumerate}

\end{enumerate}


\end{document}
