\input{preamble.tex}
\input{definitions_ask.tex}

\pagestyle{askhseis}

\renewcommand{\vec}{\mathbf}

\begin{document}

\begin{center}
  \minibox{\large \bfseries \textcolor{Col1}{Ασκήσεις στα Διανύσματα}}
\end{center}

\vspace{\baselineskip}

\begin{enumerate}

	\item Να υπολογιστεί η γωνία $ \theta $, $ 0\leq \theta \leq \pi $ μεταξύ των διανυσμάτων $
		\vec{a} = (3,0,3) $ και $ \vec{b} = (0,6,6) $.

		\hfill Απ: $ \frac{\pi}{3} $

	\item Να δειχθεί με τη βοήθεια του εσωτερικού γινομένου ότι
		\begin{enumerate}[i)]
			\item τα διανύσματα $ \vec{a}_1 = (5,-2,1) $ και $ \vec{b}_1 = (-15,6,-3) $ είναι παράλληλα.
			\item τα διανύσματα $ \vec{a}_2 = (4,-3,2) $ και $ \vec{b}_2 = (-1,-2,1) $ είναι κάθετα.
		\end{enumerate}

	\item Να υπολογιστεί το εξωτερικό γινόμενο των διανυσμάτων $ \vec{a} = (2,-1,1) $ και $ \vec{b} = (-1,2,-3) $.

		\hfill Απ: $ (1,5,3) $

	\item Να υπολογιστεί το εμβαδό του παραλληλογράμμου που σχηματίζουν τα διανύσματα $ \vec{a} = (3,-2,-2) $ και $ \vec{b} = (-1,0,-1) $.

		\hfill Απ: $3 $ 
	\item Να γραφεί το διάνυσμα $ \vec{a}=(-3,10,10) $ ως γραμμικός συνδυασμός των διανυσμάτων $\vec{b}=(1,-1,2)$, $ \vec{c}=(2,0,3) $ και $ \vec{d}=(-1,4,3) $.  

		\hfill Υπόδειξή: Λύση $ 3\times 3 $ σύστημα. 

		\hfill Απ: $ \vec{a}=2\vec{b}-\vec{c}+3\vec{d} $

	\item Να βρεθεί διάνυσμα $ \vec{b} $ που έχει μέτρο $3$ και είναι αντίρροπο του διανύσματος $ \vec{a}=(4,2,-4)$.

		\hfill Υπ: $ \vec{b} = \lambda \vec{a} $		

		\hfill Απ: $ \vec{b}=(2,-1,2) $

	\item Έστω διανύσματα $ \vec{a}=(3,2,1) $ και $ \vec{b}=(1,-1,2) $.
		\begin{enumerate}[(i)]
			\item Να βρεθούν όλα τα δινύσματα του $ \mathbb{R}^{3} $ που είναι κάθετα στα $ \vec{a} $ και $ \vec{b} $.
			\item Ποια από τα παραπάνω έχουν μέτρο ίσο με $3$; 
		\end{enumerate}

		\hfill Υπ: Έστω $ \vec{c} = (x,y,z) $

		\hfill Απ: \begin{tabular}{l}
			(i) $ \vec{c} = (x,-x,-x), x\in \mathbb{R}$, $a\times b = (5,-5,-5)$ \\
			(ii) $ (\sqrt{3}, - \sqrt{3}, - \sqrt{3}), (-\sqrt{3}, \sqrt{3}, \sqrt{3})$ 
		\end{tabular}

	\item Να δείξετε ότι 
		\begin{enumerate}[(i)]
			\item $(\vec{a}\cdot \vec{b})^{2} = \vec{a}^{2}\cdot \vec{b}^{2}$
			\item $|\vec{a}\cdot \vec{b}| = ||\vec{a}|| \cdot ||\vec{b}||$
		\end{enumerate}
		
	\item Να δείξετε ότι $ (\vec{a}\times \vec{b})^{2} + (\vec{a}\cdot \vec{b})^{2} =
		\vec{a}^{2}\cdot \vec{b}^{2} $  

	\item Να δείξετ ότι
		\begin{enumerate}[i)]
			\item $ ( \vec{a} - \vec{b} ) \times ( \vec{a} + \vec{b} ) = 2 (\vec{a} \times \vec{b}) $
			\item Πως ερμηνεύεται γεωμετρικά το παραπάνω αποτέλεσμα αν τα διανύσματα $ \vec{a} $ και $ \vec{b} $ είναι γραμμικώς ανεξάρτητα;
		\end{enumerate}
		
	\item Αν $ \vec{a} = (1,1,0) $ και $ \vec{b} = (1,2,-1) $ και η γωνία που σχηματίζουν είναι $
		\phi\neq \frac{\pi}{2} $ να υπολογιστεί η $ \tan{\phi} $ 

		\hfill Υπ: $ \tan{\phi} = \frac{\sin{\phi}}{\cos{\phi}} $

		\hfill Απ: $ \tan{\phi} = \frac{\sqrt{3}}{3}  $

	\item Να δείξετε ότι τα σημεία $ A(1,2,3), B(4,2,4), C(2,4,0) $ και $ D(2,1,3) $ είναι συνεπίπεδα.

	\item Να βρείτε το $x$ και $y$ ώστε τα διανύσματα $ \vec{a} = (x^{2}+y^{2},2,13) $, $ \vec{b} = (1,2x-3y,1) $ να είναι μεταξύ τους κάθετα. 

		\hfill Απ: $ x=-2 $, $ y=3 $

	\item Να δείξετε ότι τα διανύσματα $ \vec{a} = (1,2,3) $, $ \vec{b} = (4,2,4) $, $ \vec{c} = (2,4,0) $ και $ \vec{d} = (-1,1,5) $ κορυφές τετραέδρου και να υπολογιστεί ο όγκος του.
		
		\hfill Απ: $ V=1 $

	\item Αν $ \vec{a} = (1,2,3) $ και $ \vec{b} = (2,0,1) $ να βρεθεί το διάνυσμα $ \vec{v}$ για το
		οποίο ισχυει $ \vec{v} = \vec{a} + \vec{b} \times \vec{v} $ 

		\hfill Υπ: $ \vec{v} (x,y,z) $

		\hfill Απ: $ \vec{v} = (\frac{3}{2}, - \frac{1}{2}, 2) $.

	\item Να βρεθεί μοναδιαίο διάνυσμα, συνεπίπεδο με τα $ \vec{a} = (1,2,3) $ και $ \vec{b} =
		(-1,0,2) $ και κάθετο στο $ \vec{b} $.


		\hfill Απ: $ (\frac{2}{3}, \frac{2}{3}, \frac{1}{3}) $,$ (-\frac{2}{3},- \frac{2}{3}, -\frac{1}{3})$  
\end{enumerate}

\end{document}
